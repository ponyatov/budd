\secrel{Filters}

An entirely different program can solve the same task as the prime number
generators described in the last section. It uses another programing technique,
filters, that is frequently useful in conjunction with generators.
Extemally(that is, examining only the messages to which an object responds) a
filter looks just like a generator. Unlike a "true" generator, however, a filter
does not produce new· values in.response to first or next but takes values
produced by a previously defined generator and modifies them or filters but
values.

The class FactorFilter exemplifies some of the essential features of a filter.
Instances of FactorFilter are initialized by giving them a generator and a
specific nonnegative value. In .response to next (the message first is in this
case replaced by the initialization protocol), values from the underlying
generator are requested andretumed, except values for which the given number is
a factor are repressed. Thus the sequence returned by an instance of
FactorFiiteris exactly the same as that given by the underlying generator, with
the exception that values for which the given number is a factor are filtered
out.

\begin{lstlisting}
    Class FactorFilter
I myFactor generator I
[
remove: factorValue from: generatorValue
myFactor factorValue.
generator generatorValue
next I possible I
[ (possible generator next) notNil ]
whileTrue:
[ (possible" " myFactor -=0)
ifTrue: [ t possible]].
i nil
\end{lstlisting}

Using FactorFilter, you can construct a simple generator for prime numbers.
First an instance of Interval that will generate all numbers from 2 to some
fixed limit is constructed. As each value is removed, a filter is inserted in
front of the generator to insure that all subsequent multiples of the value will
be eliminated. A new value is then requested from the updated generator.

\begin{lstlisting}
    Class Primes
    I primeGenerator lastFactor I
    [
    first
    primeGenerator 2 to: 100.
    lastFactor primeGenerator first.
    i lastFactor
    next
    primeGenerator (FactorFilter new;
    remove: lastFactor
    from: primeGenerator).
    i lastFactor primeGenerator next    
\end{lstlisting}

Pictorially, the underlying generator constructed by the first occurrence of the
message next can be viewed as follows:

\fig{language/filter_1.png}{height=.2\textheight}

When asked for the next prime, the generator is modified by adding a filter,
this time for the last prime value returned, the number 3.

\fig{language/filter_2.png}{height=.2\textheight}

The program continues. Each time a new prime is requested, a filter is
constructed to remove all factors of the previous prime. in this fashion, all
the primes are eventually generated.

\fig{language/filter_3.png}{width=\textwidth}

Of course, like the first two programs in the last section, the storage required
for the chain of filters is proportional to the number of primes generated so
far. Despite this, timings of actual programs show that the filter program is
the fastest of the prime number generating programs described in this chapter.
