\secrel{Class Definition}\secdown

В этой главе представлен синтаксис, используемый для определения классов, и пример определения.

\bigskip

The last chapter introduced some of the standard classes in the Little
Smalltalk system. This chapter will show how the user can define new
classes to provide additional functionality for the Smalltalk language.

Class descriptions cannot be entered directly at the keyboard during
a Little Smalltalk session. Instead, class descriptions must be prepared
beforehand in a file, using a standard editor. There can be any number of
class descriptions in a file, although it is convenient to keep each class
description in a separate file for ease in program development. The textual
representation of a class description consists of two main parts, the class
Heading, followed by a sequence of one or more m.ethod descriptions. We
will discuss each portion in turn. The syntax for class descriptions is given
in detail in Appendix 2.

Figure 4.1 shows a prototypical class description, in this case the description for the class Set. The initial part of the description, the class
header, consists of the keyword Class (the first letter capitalized) followed
by a class name (also capitalized). Following the class name, the superclass
name preceded by a colon, is given. The superclass name is optional and,
if not given, the class Object will be assumed.

After the class and superclass names, the class description lists instance
variable names. Instaqce variables provide the local memory for instances
of the class. Each class instance is given its own set of instance variables,
and no other object can modify or change an instance variable in another
object. The list of instance variable names is surrounded by a pair of
vertical bars. Note that Smalltalk has no notion of variables being declared
to be of a specific type, thus any instance variable can contain any accessible
object or expression. Although the syntax is free form, it is conventional
to place the instance variables on a line separate from the class name.
Instance variables are initially given the value nil. Ifa class does not contain
any instance variables, the entire list, including the vertical bars, can be
omitted. Following the heading, a pair ofsquare braces surround the methods that comprise the protocol for the class

Each method in the class protocol defines how instances of the class
will respond.to a single message. The particular message being defined by
the method is given by the message pattern. The pattern is the first of the
three major portions ofa method; the other two portions are a list of
temporary variables and the message body.

\fig{language/fig_4_1.png}{height=.6\textheight}

A message pattern defines both the name of the message for which the
method is providing protocol and the names to be used when referring to
parameters associated with the message. There are three forms of message
pattern, corresponding to the three forms of messages (unary, binary, and
keyword). Example methods using two of those forms are given in Figure
4.2. These methods are from class Collection. Note that method descriptions are separated by vertical bars. The one exception to the free form
notation for class descriptions is that this vertical bar must appear in the
first column.

An optional list of temporary identifier names can appear following a
message pattern. For example, the message description for the message
do: in Figure 4.2 lists a temporary identifier named item. Temporary identifiers are created when the method is initiated in response to a message,
and they exist as long as the message body is being executed or there is a
block in existence that can refer to the temporary value. Like instance
variables, temporary variables are initially given the value nil.

A method body consists of a sequence of one or more Smalltalk expressions separated by periods. Note that the period is an expression separator,
not an expression terminator, and does not follow the final expression in
the method body. The final expression in a method body, or the final
expression in any block created within the method body, can be preceded
by an up arrow to indicate that the following expression is to be returned
as the response to the message. (On some terminals the up arrow looks
like a <lcaret" or".) If no explicit value is returned, the default action at the
end of a method is to return the receiver as the response to the messae.

\fig{language/fig_4_2.png}{width=\textwidth}

\clearpage\noindent
Note that the up arrow indicates an immediate return from the current
method description even ifthe arrow occurs within a block. This is different
from a block returning a value, which, as we saw in Chapter 2, is implicitly
the value of the last expression in the block.

Within a method body there are four types of variables that can be
accessed, namely instance variables, argument variables, temporary variables, and pseudo variables. In addition to the pseudo variables discussed
in the last chapter (true, false, nil and smalltalk), the pseudo variables
self, super, and selfProcess can be used. Both the variables self and super
refer to the receiver of the current message. When a message is passed to
self, the search for a method begins with the class description associated
with the class of the receiver. On the other hand, when a message is passed
to super, the search begins with the class description of the superclass of
the class in which the current message is defined. For example, if an
expression in class Set (Figure 4.1) passed a message to super, the search
for a matching method would begin in class Collection, the superclass of
class Set. Amessage passed to selfwould initiate a search for an associated
method in class Set.

To illustrate the actions of self, suppose variable "x" is an instance of
class Set. The methods shown in Figure 4.2 are from class Collection, a
superclass of class Set. In response to the message "x isEnzpty, " the method
isEmpty shown in Figure 4.2 is initiated. This method, in turn, passes the
message size to the pseudo variable self, which in this case represents x.
Thus the search for a nlethod matching the message size would begin in
class Set, the class of x. If, on the other hand, the method for isEmpty had
passed the message size to the pseudo variable super, the search for a
corresponding method would have begun in the class Object, the super
class of Collection.

Two messages are singled out for special treatment. If either of the
messages new or new: is defined in a class protocol, then, when an instance
of that class is created, the associated message (either new or new:) will
automatically be passed to the new instance before any further processing.
Thus, these messages can be used to provide automatic initialization of
instance variables. The next section describes in detail one class, the class
Set, and illustrates the use of this feature.

\secrel{An Illustrative Example}

Figure 4.3 gives the complete class description of the class Set. A set is a
collection of objects; each object occurs no more than once in the collection. (See Exercise 2.) The data structure used to implement the set will
be a list. Recall that a list is also a collection but one that maintains values
in order.

\fig{language/fig_4_3_a.png}{width=.7\textwidth}

\fig{language/fig_4_3_b.png}{width=.7\textwidth}

As we have already seen, the class Set maintains one instance variable,
list. Each instance of class Set will have a separate instance variable that
cannot be accessed by any other object. The method for message new will
automatically create an instance of class List in the variable list whenever
an instance of class Set is created.

When an element is added to a set, a test is first performed to see if
the element is already in the set. If so, no further processing is done; if
not, the element is added to the underlying Jist. Similarly, to remove an
element, the actions in the method in Set merely pass the removal message
on to the underlying list. Note that the long form of the remove: message
is defined; it includes a second argument, indicating what actions should
be taken if the given item is not found. As we saw in Figure 4.2, the class
Collection defines the short message (re1110ve:) in terms of this longer
message. Thus either form ofthe message can be used on a Set.

The messages first and next are used to produce generators. Generators
will be the topic of a later chapter; it is sufficient to say here that first and
next provide a means to enumerate all elements in a collection. The message first can be interpreted informally as flinitialize the enumeration of
items in the collection and return the first item, or nil if there are no items
in the collection." Similarly, next can be defined as "return the next item
in the collection, or nil if there are no remaining items." The protocol for
do: (Figure 4.2) illustrates how these messages can be used to construct a
loop to access every item in a collection. In Class Set, the generation of
items in response to these messages is provided by passing the same messages to the underlying list.

\secrel{Processing a Class Definition}

Once you have created a class description, the next step is to add the class
to the collection ofstandard classes provided by the Little Smalltalk system.
Suppose, for example, you have defined a class Test in the file test.st; to
add this class definition to a running Little Smalltalk execution you would
type the following command in place of a Smalltalk expression:
\begin{lstlisting}
    )i test.st
\end{lstlisting}

The i can be thought of as mnemonic for "include." At this point the
class description will be parsed, and, if there are syntax errors, a number
of descriptive error messages will be produced. Ifthere are no syntax errors,
the class definition will be added to the collection of classes known to the
Little Smalltalk system. Whether there are syntax errors or not, the cursor
should advance by one tab stop when the system is ready to accept the
next command.

If there are syntax errors, the class description can be edited without
leaving the Little Smalltalk system. To do this, type the following command:
\begin{lstlisting}
    )e test.st
\end{lstlisting}

The )e command will place the user in an editor mode to view the
designated file\note{On UNIX Systems the editor selected can be changed by modifying the EDITOR
environment variable.}. When the user exits the editor, another attempt will be
made to include the file, as if the )i command were typed.

Once a class definition has been successfully included, the defined class
can be used the same as any other class in the Little SmalltaIk system. For
example, a new instance can be created using the command new.
\begin{lstlisting}
    i~Te5t new
\end{lstlisting}

There are other system commands, similar to )i and k, described in
Appendix 1. These commands are provided with an alternative syntax,
rather than as messages passed to smalltalk, because they are used during
the bootstrapping process before any objects have been defined. The bootstrapping of the Little Smalltalk system will be discussed in the second
section of this book.


\secup

\secly{EXERCISES}

\begin{enumerate}

\item A Bag is similar to a Set; however, each entry may occur any number
of times. One way to implement the class Bag would be to use a
dictionary, similar to the way a List was used to implement the class
Set in Figure 4.3. The value contained in the dictionary for a given
entry would represent the number of times the entry occurs in the bag.
The framework for such an implementation is shown below. Change
the name of the class from Bag to MyBag, and complete the implementation of class Bag.
\begin{lstlisting}
    Class Bag :Colleetion
    I diet count I
    [
    new
    dict~Dictionarynew
    several missing methods
    first
    (count diet first) isNil ifTrue: [ t nil].
    count count - 1.
    t diet currentKey
    next
    [count notNil]whileTrue:
    [(eount>O)
    ifTrue:[eount~eount-1. t diet currentKey]
    ifFaIse: [eou nt~ iet next]].
    t nil
\end{lstlisting}
Keep in mind that instances of Dictionary respond to first and next
with values and not keys. However, the current key is accessible if you
use the message currentKey

\item The collections described in Chapter 3 were all linear, meaning they
could all be represented by a linearly written list. A common nonlinear
data structure in computer science is the binary tree. Implement the
class BinaryTree. Instances of BinaryTree contain a left subtree (or
nil), a right subtree (or nil), and a value. Instances of the class should
respond to the following messages:

\begin{tabular}{l p{8cm}}
    left & Return the left subtree, usually either nil or other instances of BinaryTree. \\
    left: & Set the left subtree to the argument value. \\
    right & Return the right subtree. \\
    right: & Set the right subtree to the argument value. \\
    value & Return the value of the current node. \\
    value: & Set the value of the current node to the argument. \\
\end{tabular}

\noindent
How should instances ofBinaryTree respond to the enumeration messages first and next? How about print or printString? What should be
.the superclass of BinaryTree?

\end{enumerate}