\secrel{Finding Out About Objects}

There are various messages that can be used to discover facts about an
object. The message class, for example, will tell you the class of an object.
Try typing
\begin{lstlisting}
7 class
\end{lstlisting}
The message superClass, when passed to an instance of Class, will return
the immediate superclass of that class. Try typing
\begin{lstlisting}
Integer superClass
7 class superClass
\end{lstlisting}
What is the superclass of Object?

The keyword message respondsTo: can be used to discover if an object
will respond to a particular message. The argument must be a symbol,
representing the message. Try typing
\begin{lstlisting}
 7 respondsTo: #+
$A respondsTo: #between:and:
$A respondsTo: #sqrt
\end{lstlisting}
When passed to a ciass, the message respondTo: inquires whetherinstances
of the class respond to the given message. For example,
\begin{lstlisting}
Integer respondsTo: #+
\end{lstlisting}

You can discover if two objects are the same using the binary message
==. The message \verb|~~| is the logical inverse of ==. Try typing
\begin{lstlisting}
i <- 17
i == 17
17~~17
\end{lstlisting}

One way to tell if an object is an instance of a particular class is to
connect the unary message class and the binary message ==. Try typing
\begin{lstlisting}
i class == Integer
\end{lstlisting}

A simple abbreviation for this is the message isMemberOf:. For example, the last expression given is equivalent to
\begin{lstlisting}
i isMemberOf: Integer
\end{lstlisting}

Suppose we want to tell if an object is a Number, but we don't care if
it is any particular kind of number (Integer or Float). We could use the
boolean OR bar (I), which is recognized by the boolean values true and
false:
\begin{lstlisting}
(i isMemberOf: Integer) | (i isMemberOf: Float)
\end{lstlisting}
A simplier method is to use the message isKindOf;. This message asks
whether the class of the object, or any of superclasses, is the same as the
argument. Try typing
\begin{lstlisting}
i isKindOf: Number
\end{lstlisting}
