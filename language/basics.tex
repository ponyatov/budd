\secrel{Основы}\secdown

Эта глава знакомит с основными понятиями языка Smalltalk; а именно объект, метод, класс, наследование и переопределение.

\bigskip

Традиционной моделью, описывающей поведение компьютера, выполняющего программу, 
является модель состояния процесса, или модель «сортировщика». В этом 
представлении компьютер является диспетчером данных, который следует некоторому 
блоку инструкций, блуждает по памяти, извлекает значения из различных слотов 
(адресов памяти), преобразует их каким-либо образом, и передает результаты 
обратно в другие слоты. Изучая значения в слотах, можно определить состояние 
машины или результаты, полученные вычислением. Хотя это может быть более или 
менее точная картина того, что происходит в компьютере, это мало помогает нам 
понять, как решать проблемы с помощью компьютера, и это, конечно, не те способы, 
которыми думают большинство людей решающих проблемы (за исключением сортировщиков и почтальонов).

Давайте рассмотрим реалистичную ситуацию, а затем посмотрим, как можно заставить 
компьютер более точно моделировать методы, которые люди используют для решения 
проблем в повседневной жизни. Предположим, я хочу послать цветы моей бабушке 
на день рождения. Она живет далеко в городе за много миль от меня. Задача 
достаточно проста для выполнения; Я просто иду к местному флористу, описываю 
виды и количество цветов, которые я хочу отправить, и я могу быть уверен, 
что они будут доставлены автоматически. Если я проведу расследование, я, 
вероятно, обнаружу, что мой флорист отправляет сообщение с описанием моего 
заказа другому флористу в городе моей бабушки. Тот флорист тогда составляет 
букет и доставляет цветы. Я мог бы спросить, чтобы узнать, как цветочный 
магазин в городе моей бабушки покупает цветы и, возможно, узнал, что они 
получены от оптового торговца цветами. Если я продолжу настаивать, я даже 
смогу проследить всю цепочку до фермера, который выращивает цветы, и узнать, 
какие запросы были сделаны членами цепочки, чтобы получить желаемый результат от каждого.

Важным моментом, однако, является то, что мне не нужно, да и вообще, 
я не хочу знать, как будет выполняться моя простая директива 
«отправить цветы моей бабушке». В реальной жизни мы называем этот 
процесс \term{делегирование полномочий}. В информатике это называется 
\term{абстракция} или \term{сокрытие информации}. В основе этих терминов лежит 
одно и то же. Есть ресурс (флорист, файловый сервер), который я хочу 
использовать. Чтобы общаться, я должен знать команды, на которые будет 
реагировать ресурс (отправить цветы моей бабушке, вернуть копию файла 
с именем "chapеer!"). Скорее всего, шаги, которые должен предпринять 
ресурс, чтобы ответить на мой запрос, гораздо более сложны, чем я 
понимаю, но мне нет смысла знать подробности того, как реализуется 
моя директива, до тех пор, пока ответ (доставка цветов, получение 
копии моего файла) четко определено и предсказуемо.

\term{Объектно-ориентированная} модель решения проблем рассматривает компьютер
способом, очень близким к такому подходу. Действительно, многие люди, 
которые не имеют никакого образования в области информатики, и не знают, 
как работает компьютер, находят объектно-ориенти\-рован\-ную модель решения 
проблем вполне естественной. Удивительно, однако, что многие люди, 
имеющие традиционный опыт программирования, изначально думают, что в 
концепции объекта есть что-то странное. Представление о том, что число «7» 
является объектом, а «+» --- это запрос на сложение, может поначалу 
показаться странным. Но вскоре единообразие, мощь и гибкость, которые 
метафора объект/сообщение привносит в решение проблемы, делает эту 
интерпретацию так же естественной.

Вселенная \st\ населена \term{объектами}. В моем примере с цветами я являюсь объектом, 
а цветочный магазин (или флорист в нем)\ --- другим объектом. Действия 
инициируются путем отправки \term{сообщений}\ (запросов) между объектами. Я передал 
просьбу «отправить цветы моей бабушке» флористу-объекту. Реакция \term{получателя}
моего сообщения состоит в том, чтобы выполнить некоторую последовательность 
действий или \term{метод}, чтобы удовлетворить мой запрос. Может быть, получатель 
сможет немедленно удовлетворить мой запрос. С другой стороны, чтобы удовлетворить 
мои потребности, получателю, возможно, придется передавать другие сообщения 
еще большему количеству объектов (например, сообщение, которое мой флорист 
отправляет флористу в городе моей бабушки, или команду на дисковод). Кроме 
того, существует явный ответ (например, квитанция или код результата), 
возвращенный непосредственно мне. Дэн Ингаллс описывает философию \st\ (Байт 81):

\begin{quote}
    Вместо того, чтобы обрабатывать структуры данных, которые насилуют и грабят 
    процессор, мы имеем вселенную объектов с хорошим поведением, которые вежливо 
    просят друг друга выполнить их различные желания.
\end{quote}
    
Такие антропоморфные точки зрения распространены среди \st-программистов. 
В последующих главах мы увидим, как язык Smalltalk воплощает этот 
объектно-ориентированный взгляд на программирование. Описывая решение 
нескольких проблем в \st, мы надеемся показать, как объектно-ориентированная 
модель помогает в создании больших программных систем, и помогает в решении 
многих проблем с использованием компьютера.

\secrel{Объекты, классы и наследование}

В Smalltalk все является объектом. В языке нет способа создать сущность, 
которая не является объектом. Среди основных компьютерных языков это 
единообразие в Smalltalk конкурирует, возможно, только с LISP, и, 
как и в LISP, единообразие создает и простоту, и мощь языка.

Объект обладает несколькими характеристиками (рисунок 1.1). Каждый объект 
содержит небольшой объем памяти, доступный только этому объекту. То есть 
ни один объект не может читать или изменять значения памяти в другом 
объекте. Конечно, поскольку всё в системе должно быть объектом, память 
объекта может содержать только ссылки на другие объекты. Мы обсудим 
это более подробно позже.

\fig{language/fig_1_1.png}{height=\textheight}

Все действия в системе \st\ производятся путем передачи сообщений. 
Сообщение\ --- это запрос у объекта выполнения какой-либо операции. Оно 
может содержать некоторые значения в качестве аргументов, которые будут 
использоваться как параметры при при выполнении запрошенной операции. 
Существует два способа рассматривать эту операцию передачи сообщения. 
Во-первых, передача сообщения соответствует вызову подпрограммы на обычном 
процедурном языке, таком как Паскаль. Это верно в том смысле, что работа 
отправителя останавливается до тех пор, пока получатель не выдаст результат.
\note{\st\ использует синхронные сообщения, поэтому отличия видны слабо, и часто возникают споры, чем сообщение отличается от вызова метода;
в этом смысле \st\ является вырожденным случаем, а наиболее явно это отличие видно в акторной модели \ref{actor}:
\emph{посылка сообщения не передает управление получателю}}
Затем результат возвращается отправителю, который продолжает выполнение 
с точки вызова. Однако сообщения могут создаваться динамически во время 
выполнения, и отношения между отправителем и получателем сообщения, как 
правило, гораздо более свободные, чем статические отношения между 
вызывающим и вызываемым в обычном языке программирования.

В реальном мире каждый объект индивидуален; однако каждый из них обладает 
общими характеристиками с другими подобными объектами. Например, в мешке 
яблок каждое яблоко отличается от всех других. Все же определенные заявления 
могут быть сделаны относительно всех яблок; например, все они будут 
пахнуть одинаково и иметь определенный вкус, все они могут быть использованы 
для выпечки пирогов одинаковым образом, и так далее. Этот процесс называется 
\term{классификацией}. То есть мы можем рассматривать яблоко как отдельный элемент 
или как \term{экземпляр} определенного \term{класса} (или категории) объектов. давайте 
обозначим класс всех яблок через \class{Apple}, заглавная буква и шрифт обозначает 
тот факт, что мы говорим о классе, а не об отдельном объекте.

Экземпляры класса \class{Orange} во многом отличаются от яблок и поэтому заслуживают 
отдельной категории. Но они также имеют много общих характеристик с яблоками. 
Таким образом, мы можем создать новый класс \class{Fruit}, который будет использоваться, 
когда мы хотим описать характеристики, общие для яблок и апельсинов. Класс 
\class{Fruit} включает в себя классы \class{Apple} и \class{Orange}. Таким образом, мы говорим, 
что \class{Fruit}\ --- это \term{суперкласс} \class{Apple} и \class{Orange}, а \class{Apple} и \class{Orange}, в свою 
очередь, являются \term{подклассами} \class{Fruit}.

Наконец, мы можем сделать еще один шаг этого анализа, сделав \class{Fruit} подклассом 
более универсальной категории, которую мы можем назвать \class{Object}. Таким образом, 
у нас есть иерархия категорий для объектов, расширяющаяся от базового класса 
\class{Object}, членом которого является все, вплоть до все более и более конкретных 
классов, пока мы не достигнем самого отдельного объекта.

Такая же ситуация имеет место в отношении всех сущностей в \st. То есть 
каждый объект является членом некоторого класса. За исключением класса \class{Object}, 
этот класс, в свою очередь, будет подклассом некоторого более крупного класса, 
который, в свою очередь, может быть частью другого класса, вплоть до одного 
класса \class{Object}, членом которого является каждый объект. Существует естественная 
древовидная структура (рисунок 1.2), которая иллюстрирует эту иерархию классов. 
Как мы уже делали, мы будем обозначать имена классов, используя первую заглавную 
букву, и обозначая имена объектов без использования заглавных букв. Так, 
например, число 7 является экземпляром класса \class{Integer}, как и число 8. Хотя 
7 и 8 являются различными объектами, они имеют некоторые общие характеристики 
в силу того, что они являются экземплярами одного и того же класса. Например, 
7 и 8 ответят на сообщение «+» целочисленным аргументом, выполнив сложение 
целых чисел. \class{Integer}\ --- это подкласс большего класса \class{Number}. Существуют и 
другие подклассы \class{Number}, например, \class{Float}, значения которых, например, 
3.1415926, являются его экземплярами. \class{Number} --- это подкласс \class{Magnitude} (класс, 
который будет обсуждаться позже), который, наконец, является подклассом \class{Object}.

\fig{language/fig_1_2.png}{height=.6\textheight}

\term{Поведение} объекта в ответ на конкретное сообщение определяется классом этого объекта. 
Например, 7 и 8 будут отвечать на сообщение «+» одинаково, потому что они оба 
являются экземплярами класса \class{Integer}. Список операторов, которые определяют, 
как экземпляр некоторого класса будет реагировать на сообщение, называется \term{методом}
для этого сообщения. Например, в классе \class{Integer} есть метод, связанный с 
сообщением «+». Весь набор сообщений, связанных с классом, называется \term{протоколом}
для этого класса. Класс \class{Integer} содержит в своем протоколе, например, сообщения 
для +, -, * и так далее. В \st\ протокол предоставляется как часть определения 
класса. Синтаксис определений классов будет описан в следующем разделе. 
Невозможно предоставить метод для отдельного объекта; каждый объект должен 
быть связан с некоторым классом, и поведение объекта в ответ на сообщения будет 
продиктовано методами, связанными с этим классом.

Если объект является экземпляром определенного класса, ясно, как будут использоваться 
методы, связанные с этим классом, но как насчет методов, связанных с суперклассами? 
Ответ в том, что любой метод, связанный с суперклассом, \term{наследуется} классом. 
Пример поможет прояснить эту концепцию. При отправке на номер сообщение \var{exp}
означает «вернуть значение $e$ (приблизительно 2,71828..), в степени указанного 
значения». Таким образом, \verb|2 exp| дает $e^2$, или приблизительно 7,38906. Теперь 
описание класса для \class{Integer} не предоставляет метод для сообщения \var{exp}, поэтому, 
когда система Little Smalltalk пытается найти связанный метод для сообщения 
\var{exp} в протоколе класса \class{Integer}, она не находит его. Таким образом, система 
\st\ затем анализирует протокол, связанный с непосредственным 
суперклассом \class{Integer}, а именно \class{Number}. Там, в протоколе для \class{Number}, она находит 
метод и выполняет его. Таким образом, мы говорим, что метод для \var{exp} \term{наследуется}
классом \class{Integer} из класса \class{Number}.
В \class{Number} метод, связанный с сообщением \var{exp}, выглядит следующим образом:
\begin{lstlisting}
    ^ self asFloat exp
\end{lstlisting}

Мы объясним синтаксис более подробно позже; на данный момент мы можем перевести 
этот код как «создать экземпляр \class{Float} с вашим значением (\verb|self asFloat|) и 
отправить этому объекту сообщение \var{exp}, запрашивающее $e$, возведенное в степень 
его значения. Возвратить\note{стрелка вверх $\wedge$ или $\uparrow$ указывает возвращаемое значение}
ответ на это сообщение". Таким образом, сообщение \var{asFloat} передается исходному 
целому числу, скажем, 2. Выполняется метод, связанный с этим сообщением, в 
результате чего получается значение с плавающей запятой 2.0. Сообщение \var{exp} 
затем передается этому значению. Это то же сообщение, которое первоначально 
было передано в целое число 2, только теперь класс получателя\ --- \class{Float}, 
а не \class{Integer}.

На рисунке 1.3 показана иерархия, представляющая несколько классов, включая числа. 
Как мы уже видели, метод для сообщения exp определен в классах \class{Number} и \class{Float}. 
Поиск метода начинается с класса объекта, а затем, при необходимости, проходит 
через различные суперклассы (по \term{цепочке наследования}). Если в сообщении \var{exp} 
дано значение с плавающей запятой, будет выполняться метод в классе \class{Float}, 
а не метод в классе \class{Number}. Таким образом, говорят, что метод для \var{exp} в \class{Float} 
\term{переопределяет} метод в классе \class{Number}.

\fig{language/fig_1_3.png}{height=.6\textheight}

Такие классы, как \class{Number} и \class{Magnitude}, которые обычно не имеют явных экземпляров, 
называются \term{абстрактными классами}. Абстрактные суперклассы важны для 
обеспечения того, чтобы экземпляры различных классов, такие как целые числа 
и числа с плавающей точкой, отвечали аналогичным образом в обычных ситуациях. 
Кроме того, устраняя необходимость дублировать методы для сообщений в 
суперклассе, они уменьшают размер описаний, необходимых для получения 
желаемого поведения.


\secrel{История, Фоновое чтение}

Концепции, относящиеся к объектно-ориентированному программированию, 
заложенные в \st, являются результатом длительного процесса 
разработки и эволюции языка. Основные понятия об объектах, сообщениях 
и классах пришли из языка Simula (Birtwistle 73). Хотя Simula 
позволяла пользователям создавать объектно-ориентированные системы, 
и классы, ответ на сообщение (эквивалент метода в Simula) все еще 
выражался в стандартном для ALGOL методе, ориентированном 
на данные и процедуры.

В семействе языков ALGOL концепция классов привела к разработке понятия 
модулей и абстрактных типов данных (Shaw 80), поддержка которых была 
фундаментальной целью в нескольких языках, таких как Euclid, CLU, Modula и Ada.

В то время как объектно-ориентированная философия постепенно получала признание 
в мире языков программирования, подобные идеи получили признание в сообществе 
архитекторов (Pinnow 82). Точно так же в дизайне операционных систем понятие 
независимых вычислений, которые взаимодействуют друг с другом исключительно 
путем обмена сообщениями, находило сторонников (Wulf 74), (Almes 85). Такое 
представление является естественным и удобным, когда вычисления могут 
физически выполняться на распределенных процессорах.

Прямые предки Smalltalk включают систему Flex (Kay 69), Smalltalk-72 (Goldberg 76) 
и Smalltalk-76 (Ingalls 78). Все языки Smalltalk были созданы в рамках 
проекта Dynabook, инициированного Аланом Кейем в Группе исследований 
обучения в Исследовательском центре Xerox в Пало-Альто. Эволюция языка, 
как показано в этих документах, показывает, что объектно-ориентированная 
модель постепенно расширяется и включает в себя все больше и больше языковых 
концепций. Например, в Smalltalk-72 числа и управляющие структуры обрабатываются 
как объекты, в отличие от Simula, но классы по-прежнему представляют собой 
особую форму. В Smalltalk-76 описания классов представлены в виде объектов, 
а объектно-ориентированная точка зрения расширена до интерфейса программирования. 
Этот интерфейс почти полностью описывается в объектно-ориентиро\-ван\-ной форме 
в среде программирования Smalltalk-80 (Goldberg, 83).

Объектно-ориентированный взгляд на программирование также повлиял на 
другие компьютерные языки, в частности, на понятия актеров (Hewit 73) и 
разновидностей (Weinreb 80) в Лиспе, а также на разработку языков для 
анимации и графики (Рейнольдс 82). Развитие акторов в Лиспе шло параллельно 
с развитием Smalltalk, и два языка влияли друг на друга.

Ковед и ЛаЛонд представляют обзоры, описывающие объектно-ориентированную 
точку зрения в различных ипостасях (Ковед 84) (ЛаЛонд 84). Ряд статей, 
описывающих различные аспекты системы Smalltalk-80, были включены в 
специальный выпуск журнала Byte (Байт 81).


\secly{Упражнения}

\begin{enumerate}

\item Определите следующие термины:

\noindent\begin{tabular}{l l l}
объект&
сообщение&
получатель\\
метод&
протокол&
класс\\
подкласс&
суперкласс&
наследование\\
переопределение&
абстрактный суперкласс\\
\end{tabular}

\item Приведите пример иерархии из повседневной жизни. Перечислите свойства, 
которые можно найти на каждом уровне, и выделите те, которые находятся на 
более низких уровнях, но не на более высоких уровнях.

\item Прочитайте о механизме классов в Simula (DaW 72) (BirtwistIe 73). 
Сравните и сопоставьте это с механизмом классов \st

\item В реальном мире объекты часто классифицируются ортогональными способами, 
а не в древовидной иерархии Smalltalk. Например, белоголовый орлан и кондор 
являются хищными птицами, но одна\ --- это североамериканская птица, а 
другая\ --- южноамериканская птица. Робин также североамериканская птица, 
но не хищная. Эти две отличительные характеристики являются ортогональными 
в том смысле, что ни одна из них не может быть логически названа надмножеством 
другой. Таким образом. навязывание классификации в древовидную структуру 
является неестественным, неэффективным или и тем, и другим. \\
Как можно классифицировать объекты Smalltalk ортогональными способами? Какие 
проблемы это создает для механизма наследования? Как можно преодолеть эти проблемы?

\end{enumerate}

\secup
