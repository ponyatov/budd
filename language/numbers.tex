\secrel{Numbers}

The class hierarchy for numbers could be described as follows:

\fig{language/numbers.png}{height=.5\textheight}

\noindent
Each of the classes Integer, Float, and Number implements methods for the
arithmetic operations. The classes Integer and Float perform operations only for
arguments of their own type. If an argument is not of the correct type, the
message is then passed to the superclass, Number. For example.,the method for +
in the class Integer looks something like the followin~:
\begin{lstlisting}
+ aNumber
    ^ <SameTypeOfObject self aNumber>
        ifTrue: [ <lntegerAddition self aNumber> ]
        ifFalse: t super + aNumber]
\end{lstlisting}

The primitive SameTyPeOfObject tests whetherthe two objects given as arguments
are instances of the same class. Since the pseudo variable self is an instance
of the class Integer, in this context the use of this primitive can be thought
of as equivalent to:
\begin{lstlisting}
aNumber isKindOf: Integer
\end{lstlisting}
although the primitive expression, since it does not require any further message
passing, is somewhat more efficient\note{The use of the primitive here is
justified on the grounds that arithmetic operations are used much more
frequently than other messages. Even so, some regard this as a weak argument and
would advocate the more direct and obvious Smalltalk expression over the less
clear primitive call.}. If the argument is of the correct class, primitive
IntegerAddition produces the integer sum of the two arguments. If the argument
is not an instance of class Integer, the message If+11 is passed to.the
superclass, namely Number. In the class Number the following methods appear:
\begin{lstlisting}
    maxtype: aNurnber.
        l' <GenerqlityTest self aNumber>
            ifTrue: [self]
            ifFalse: [aNumber coerce: self]
    + ?Number
        i (self maxtype: aNumber) + (aNumber maxtype: self)
\end{lstlisting}

To uiIderstand these methods, consider that a hierarchy of number classes can be
defined, consisting of Integer at the lowest level, followed by Float, then
followed by any others (including user-defined classes). When presented with two
objects of different levels in the hierarchy, the class Number chooses the
object with the more general class and passes to it the message coerce: with the
other object as an argument. For example typing i + 3.5 results in the message
coerce: 2 being passed to the object 3.5. The class Float, and any user-defined
dasses, must implement a method for this message.

To find the most general form for the operation you use the message maxtype:.
The method for this message uses the primitive GeneralityTest, which returns
true if the first argument is of a more general class than the second, and false
otherwise\note{There is a problem here in determining the relative generality of
two user-defined classes, or even the generality of user-defined classes and
known classes, such as Float. One of the projects described at the end of the
book invites the student to examine this problem and produce more general
solutions.}. The method for maxtype: therefore either returns its first argument
or coerces the argument into being of the class represented by the second
argument:
\begin{lstlisting}
    coerce: aNumber
        i aNumber asFloat
\end{lstlisting}

When Number has coerced both arguments into being the same type, the original
message is then passed back to the modified objects. Assuming the response to
coerce: was as expected, the objects should now be able to respond correctly to
this message, and the expected result is finally produced.
