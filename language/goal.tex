\secrel{Goal-Directed Evaluation}

A useful programming technique when used in conjunction with generators is
goal-directed evaluation. By this technique, a generator is repeatedly queried
for values until some condition is satisfied. In a certain sense the notion of
filters we have just described represents a simple form of goaldirected
evaluation. The goal of instances of FactorFilter, for example, is to find a
value from the underlying generator for which the given number is not a factor.
In the more general case of goal directed evaluation, the condition frequently
involves the outcome of several generators acting together. An example will
illustrate this Il1ethod.

Consider the problem of placing eight queens on a chess board in such a way that
no queen can attack any other queen (Figure 8.3). In this section we will
describe how such a problem can be formulated and solved using generators,
filters, and goal directed evaluation.

We first observe that in any solution, no two queens can occupy the same column,
and that no column can be empty. We can therefore assign a specific column to
each queen at the start, and reduce the problem to finding a correct row
assignment for each of the eight queens.

In general terms, our approach will be to place queens from left to right (the
order in which we assign numbers to columns). An acceptable solution for column
n is one in which no queen in columns 1 through n can attack any other queen in
those columns. Once we have found an acceptable solution in column 8 we are
finished. Before that, however, we can formulate the problem of finding an
acceptable solution in column n recursively, as follows:

\begin{enumerate}
    \item Find an acceptable solution for column n - 1. If there is none, return
    nil, there is no acceptable solution. Otherwise, place the queen for column
    n in row 1. Go to step 2.
    \item Test to see if any queen in columns 1 through n - 1 can attack the
    queen in column n. If not, then an acceptable solution has been found. if
    the queen can be attacked, then go to step 3.
    \item If the queen for column n is in row 8, go to step 4, otherwise advance
    the queen by one row and go back to step 2.
    \item Find the next acceptable solution for column n - 1. If there is none,
    return nil, otherwise, place the queen for column n in row 1 and go to step
    2.
\end{enumerate}

\fig{language/fig_8_3.png}{height=.7\textheight}
