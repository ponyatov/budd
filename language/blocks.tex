\secrel{Blocks}

An interesting feature of Smalltalk is the ability to encapsulate a sequence
of actions and then to perform those actions at a later time, perhaps even
in a different context. This feature is called a block (an instance of class
Block) and is formed by surrounding a sequence of Smalltalk statements
with square braces, as in:
\begin{lstlisting}
[ i <= i+1. i print ]
\end{lstlisting}

Within a block (and, as we will see in the next chapter, in general
within a method) a period is used as a statement separator. Since a block
is an object, it can be assigned to an identifier or passed as an argument
with a message or used in any other manner in which objects may be used.
In response to the unary message value, a block will execute in the context
in which it was defined, regardless of whether this is the current context
or not. That is, when the block given above is evaluated, the identifier i
will refer to the binding of the identifier i that was known at the time the
block was defined. Even if the block is passed as an argument into a class
in which there is a different instance variable i and then evaluated, the i
in the block will refer to the i in the context in which the block was defined.
Thus a block when used as a parameter is similar to the Algol-60 call-byname notion of a thunk.

The value returned by a block is the value of the last expression inside
that block. Frequently a block will contain a single expression, and the
value resulting from that block will be the value of the expression.

One way to think about blocks is as a type of in-line procedure declaration. Like procedures, a block can also take a number of arguments.
Parameters are denoted by colon-variables at the beginning of the block,
followed by a vertical bar and then the statements composing the block.
For example,
\begin{lstlisting}
[:x :y | (x + y) print]
\end{lstlisting}
is known as a two-parameter block (sometimes two-argument block). The
message value: is used to evaluate a block with parameters the number of
value: keywords given matches the number of arguments in the block. So,
for the example given above, the evaluating message would be value:value:.
