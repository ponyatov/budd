\secrel{Identifiers}

Identifiers in Little Smalltalk can be divided into three categories: instance
variables, class names, and pseudo-variables. An identifier beginning with
a capital letter is always a class name, whereas an identifier beginning
with a lowercase letter must represent either a pseudo variable or an
instance variable.

At the command level, new instance variables can be defined merely
by assigning a value to a name. The assignment arrow is formed as a twocharacter sequence consisting of a less than sign and a minus sign
\note{From now on the text will use the symbol $\leftarrow$ to represent this two-character sequence}:
\begin{lstlisting}
newname <- 17
\end{lstlisting}

Instance variables defined at the command level are known only at the
command level and cannot be used within a method for any class. As we
will see in a later chapter, instance variables within a class must all be
declared.

Class identifiers respond to a variety of messages that can be used to
discover information concerning the class the object represents. For example, the message respondTo, when passed to an object representing a
class, will cause the class to print a list of the messages to which instances
of the class will respond.

Pseudo variables look like normal identifiers (that is, they are named
by a sequence of letters beginning with a lower case letter), but unlike
identifiers they need not be declared. There are several pseudo variables:
self, super, selfProcess
\note{The pseudo-variables selfprocess and smalltalk are unique to Little Smalltalk and are
not part of the \st-80 system, where different techniques are used to obtain the currently executing process or to obtain information about the current environment. See Appendix 5 for an overView of the differences between Little Smalltalk and the Smalltalk-80
programming environment.}
, true, false, nil, and smalltalk. Arguments for
a method (to be discussed shortly) are also considered to be. pseudo-variables. Of the seven, self, super, and selfProcess are farthest from being
literal objects because their meaning depends entirely upon context. We
will discuss these in more detail when we describe class methods and
processes. The next three, true, false, and nil, are defined to be instances
(usually the only instances) of the classes True, False, and UndefinedObject, respectively. We will discuss these three in more detail when
we outline the behavior of different classes. The final pseudo variable,
smalltalk, is an instance of class Smalltalk and is used to centralize several
pieces of information concerning the currently executing environment.

Other types of objects in the Little Smalltalk system, such as blocks
and instances of user defined classes, will be discussed in later sections.
