\secrel{Primitives, Cascades, and Coercions}\secdown

This chapter introduces the syntax for cascaded expressions and describes the
notion of primitive expressions. It illustrates the use of primitives by showing
how primitives are used to produce the correct results for mixed mode arithmetic
operations.

\secrel{Cascades}

Suppose you wanted to send a number of messages to the same object. For example,
suppose you wanted to create a new instance of class Bag and to initialize it
with the values 0 and 1. One way to do this would be to use a temporary variable
as in:
\begin{lstlisting}
i~Bag new
i add: 0
i add: 1
\end{lstlisting}

Another way is to use a cascade. A cascade is a type of Smalltalk
expression\note{Note that although similar concepts appear in both the Little
Smalltalk system and the Xerox Smalltalk-80 programming environment, the syntax
and meaning of cascades and primitives are different in the two systems.
Appendix 5 describes how cascades and primitives are handled in the Xerox
system.}. It is written as an expression followed by a semicolon followed by
another expression without a receiver. When a cascade is evaluated, the first
expression is computed, and then used as the receiver for the second expression.
The result of a cascade is not the result ofthe second expression but the result
ofthe first expression. In other words, the result ofthe second expression is
ignored and discarded. Since a cascade is an expression, it can be used to form
another cascade, and so on, to any desired level. The result of a cascade of any
depth is always the result of the first expression. Thus, for example, you can
create a new Bag and initialize it all with one expression, as in:
\begin{lstlisting}
Bag new; add: 0 ; add: 1
\end{lstlisting}
The receiver for both the messages add: a and add: 1 is the result of the
expression Bag new, as is the result of the entire expression.

There are several advantages to using cascaded expressions. Since
there is no necessity to name the intermediate object, a cascade obviates
the need for a great many temporary variables. Also, since the entire sequence is an expression, it can be used anywhere expressions are legal,
for example, as an argument. Finally a cascade can often be used in place
of parentheses to separate two keyword messages or to apply one message
to another of lower precedence. For example, to print the result yielded
by a keyword message you could write:
\begin{lstlisting}
Integer respondsTo: #+ ; print
\end{lstlisting}
in place of
\begin{lstlisting}
(Integer respondsTo: # +)print
\end{lstlisting}

\secrel{Primitives}

At this point you may be wondering how anything ever gets accomplished in
Smalltalk. As described so far, actions are produced by one object sending a
message to another object. In response to this message, the second object may in
turn send still other messages to more objects. If this were all there were to
the language, there would seem to be a sort of infinite regress preventing any
substantial action from taking place.

The solution to this problem is yet another form of expression, called a
primitive. A primitive can be thought of as a system call, permitting access to
the underlying virtual machine. Syntactically, a primitive is denoted by an
angle bracket followed by a name followed by a list of objects followed by a
closing angle bracket, for example:
\begin{lstlisting}
<lntegerAddition i j>
\end{lstlisting}
Alternatively, every primitive is also assigned a number. The primitive name can
be replaced by the keyword "primitive" and this number. Thus, since the integer
addition primitive is number 10, the above expression could also be written as:
\begin{lstlisting}
<primitive 10 i j >
\end{lstlisting}
However, the second form is less representative of the operation being
performed. Primitives are chiefly used in class descriptions and should almost
never by typed directly at the command level. In any case, only the second form
(expressing the desired primitive by number) is recognized at the command level.

Note that a primitive call is not the same as a message passing, although the
differences are largely hidden from the user. There is no notion of a "receiver"
in a primitive expression, and a primitive cannot send further messages in
producing its response. For this reason most primitive operations, as the name
suggests, are very simple. A table of primitives is given in Appendix 4. In
general the value nil is returned, and an error message produced, if arguments
to a primitive are not correct or the primitive cannot produce the desired
result.

A portion ofthe class description for the class Radian is given in Figure 6.1.
This class uses the primitives Sin and Cos to compute the value of the
trigonometric functions on radian inputs. Notice how the class Radian responds
to the message printString by concatenating the string representation of its
value with the literal string "radians." Thus the message IfX print," if x is a
radian with value 0.85, will produce the message "0.85 radians." This technique
is used in many classes to provide more informative printed representations of
objects.

There is a temptation on the part of many novice Little Smalltalk users to use
primitives in place of Smalltalk expressions in the name of llefficiency." This
should be avoided. Such gains in efficiency are usually negligible, if they
exist at all. More importantly, the notion of primitive is really just a
pragmatic device permitting some operations to be specified that could not
otherwise be given in Smalltalk. The syntax is a bit obscure, and the notion
does not fit quite smoothly into the object-oriented framework of the rest of
Smalltalk. Thus primitives should be used only as a last resort, only after a
great deal of thought as to the alternatives, and only in extremely simple
methods that put a more lfSmalltalk-like" cloak over the primitive call.

\fig{language/fig_6_1.png}{height=.9\textheight}

\secrel{Numbers}

The class hierarchy for numbers could be described as follows:

\fig{language/numbers.png}{height=.5\textheight}

\noindent
Each of the classes Integer, Float, and Number implements methods for the
arithmetic operations. The classes Integer and Float perform operations only for
arguments of their own type. If an argument is not of the correct type, the
message is then passed to the superclass, Number. For example.,the method for +
in the class Integer looks something like the followin~:
\begin{lstlisting}
+ aNumber
    ^ <SameTypeOfObject self aNumber>
        ifTrue: [ <lntegerAddition self aNumber> ]
        ifFalse: t super + aNumber]
\end{lstlisting}

The primitive SameTyPeOfObject tests whetherthe two objects given as arguments
are instances of the same class. Since the pseudo variable self is an instance
of the class Integer, in this context the use of this primitive can be thought
of as equivalent to:
\begin{lstlisting}
aNumber isKindOf: Integer
\end{lstlisting}
although the primitive expression, since it does not require any further message
passing, is somewhat more efficient\note{The use of the primitive here is
justified on the grounds that arithmetic operations are used much more
frequently than other messages. Even so, some regard this as a weak argument and
would advocate the more direct and obvious Smalltalk expression over the less
clear primitive call.}. If the argument is of the correct class, primitive
IntegerAddition produces the integer sum of the two arguments. If the argument
is not an instance of class Integer, the message If+11 is passed to.the
superclass, namely Number. In the class Number the following methods appear:
\begin{lstlisting}
    maxtype: aNurnber.
        l' <GenerqlityTest self aNumber>
            ifTrue: [self]
            ifFalse: [aNumber coerce: self]
    + ?Number
        i (self maxtype: aNumber) + (aNumber maxtype: self)
\end{lstlisting}

To uiIderstand these methods, consider that a hierarchy of number classes can be
defined, consisting of Integer at the lowest level, followed by Float, then
followed by any others (including user-defined classes). When presented with two
objects of different levels in the hierarchy, the class Number chooses the
object with the more general class and passes to it the message coerce: with the
other object as an argument. For example typing i + 3.5 results in the message
coerce: 2 being passed to the object 3.5. The class Float, and any user-defined
dasses, must implement a method for this message.

To find the most general form for the operation you use the message maxtype:.
The method for this message uses the primitive GeneralityTest, which returns
true if the first argument is of a more general class than the second, and false
otherwise\note{There is a problem here in determining the relative generality of
two user-defined classes, or even the generality of user-defined classes and
known classes, such as Float. One of the projects described at the end of the
book invites the student to examine this problem and produce more general
solutions.}. The method for maxtype: therefore either returns its first argument
or coerces the argument into being of the class represented by the second
argument:
\begin{lstlisting}
    coerce: aNumber
        i aNumber asFloat
\end{lstlisting}

When Number has coerced both arguments into being the same type, the original
message is then passed back to the modified objects. Assuming the response to
coerce: was as expected, the objects should now be able to respond correctly to
this message, and the expected result is finally produced.


\secly{Exercises}

\begin{enumerate}

    \item One advantage cited for cascaded expressions was that several
Smalltalk statements could be combined together into one expression. For example
the initialized Bag discussed in the text could be used as an argument to
another object as follows:
\begin{lstlisting}
anObject foo: (Bag new; add: 0; add 1)
\end{lstlisting}
How else might this be done in a single expression without using cascading? You
can use temporary variables, if you wish.

\item What will the result of typing the following expression be? Explain why.
\begin{lstlisting}
2 + (3 print) ; + 4
\end{lstlisting}

\item Examine the class descriptions for the classes Object, Magnitude, Number,
Integer, and Float. Explain in detail how radians respond to the messages < =
and > = .

\item Rewrite the methods for = and < in class Radian so that radians can be
compared only to other radians.

\item Does the class Integer need to provide a method for coerce:? Why or why
not? How about the class Number?

\item An alternative to having instances of class Radian maintain their value in
an instance variable would be to make the class a subclass of class Float.
Discuss the advantages and disadvantages of this approach.

\item Recall that in the hierarchy of number classes all other classes have a
higher ranking then any of Integer or Float. Assume that the class Number
contains the following method
\begin{lstlisting}
    i 
        Complex new; imagpart: self
\end{lstlisting}
That is, in response to the message i, a number will create a new instance of
the type Complex and initialize it with the current object.

Using this method, define the class Complex used to manipulate complex numbers.
Your class should implement methods for the following messages:

\noindent
\begin{tabular}{l p{8cm}}
    new & Set both the imaginary and real portions of the complex number to
    zero. \\
    realpart: & Define the real part of the current number to be the argument.
    \\
    imagpart: & Define the imaginary part of the current number to be the
    argument. \\
    coerce: & Coerce a non-complex, returning an equivalent complex number with
    zero imaginary part. \\
    + & Complex addition. \\
    * & Complex multiplication. \\
    printString & produce a printable representation of the number. \\
\end{tabular}

\noindent
Test your class description by typing several example expressions involving
complex numbers.

\item A useful control structure in many programming languages is a multiway
switch statement, which permits the selection of one out ofseveral possibilities
based on the value of some selection key. Using cascades and blocks we can
implement a multiway switch in Smalltalk. The class Switch will use the message
new: both as a creation message and to assign the switch selection key. The
message case:do: will compare the first argument to the selection key, and, if
equal, the second argument (a block) will be evaluated. The message test:do:
uses blocks for both arguments. The first argument, a one parameter block, is
evaluated using the selection key; if it returns true, the second argument is
evaluated. A flag is maintained by each instance of Switch indicating whether
any condition has been satisfied. The message default: executes the argument, a
block, if no previous condition has been met.

For example, the following statement uses a variable suit representing the suit
of a card from a deck of cards. It places into color a string representing the
color ofthe card.
\begin{lstlisting}
    Switch new: suit;
    case: #spade do: [ color~lblack'] ;
    case: #c1ub do: [ color~lblack' ] ;
    test: [:x I(x = #diamond) or: [x = #heart] ] do: [ color~'red'] ;
    default: [ self error: 'unknown suit I , suit]
\end{lstlisting}
Provide a class description for Switch.

\end{enumerate}

\secup
