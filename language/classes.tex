\secrel{Basic Classes}\secdown

Основные классы, включенные в стандартную библиотеку Little Smalltalk, описаны в этой главе.

\bigskip

The classes included in the Little Smalltalk system can be roughly divided
into four overlapping groups. These groups are Basic Objects (Object,
UndefinedObject, Symbol, Boolean, True, False, Magnitude, Number,
Char, Integer, Float, Radian, Point), Collections (Collection, Bag, Set,
SequenceableCollection, KeyedCollection, Dictionary, Interval, List,
ArrayedCollection, Array, String, File, Random, ByteArray), Control
Structures (Boolean, True, False, Interval, Block), and System Management (Object, Class, Smalltalk, Process). The following sections will
briefly describe each of these categories. Appendix 3 provides detailed de;.
scriptions for each of the standard classes.

\secrel{Basic Objects}

The class Object is a superclass of all classes in the system and is used to
provide a consistent basic functionality and default behavior. For example,
the message = = is defined in class Object and is thus accepted by all
objects in the Little Smalltalk system. This message tests to see if the
expressions for the receiver and the argument represent the same object.
Another message defined in class Object is the message class, which returns the object representing the class of the receiver.

The last chapter introduced the classes associated with literal objects.
other types of objects are also basic to many applications. For example,
instances of the class Radian are used to represent radians. A radian is a
unit of measurement, independent of other numbers. Only radians will
respond to trigonometric functions such as sin and cos. Numbers can be
converted into radians by passing them the message radians. Similarly,
radians can be converted into numbers by sending them the message
asFloat. Only a limited range of arithmetic operations on Radians, such
as scaling by a numeric quantity or taking the difference of two radians,
are permitted. Radians are normalized by adding or subtracting multiples
of 2'11" from their value.

The class Point is used to represent ordered pairs of quantities. Ordered pairs are useful in the solution of many problems, such as storing
coordinate pairs in graphics applications. In fact, the class Number provides a convenient method for constructing points. All instances of class
Number will respond to the message @ by producing a point consisting
of the receiver and the argument. Thus 10 @ 12 generates a point representing the ordered pair (l0,12). The first value is known as the x-value
and will be returned in response to the message x. The second value is the
y-value and is returned in response to the message y

The class String provides messages useful in manipulating arrays of
characters. One important property of this class is that its instances are
the only objects in the Little Smalltalk system that can be displayed on an
output device such as a terminal or printer. Any object to be displayed
must first be converted into an instance of class String. The behavior
defined in class Object for the message print is to convert the object into
a string (using the message printString) and then to print that string (by
passing the message print to it).

The message printString is uniformly interpreted throughout the Little
Smalltalk system as "produce a string representation ofyour value." Classes
for which this makes sense (such as Integer) must define a method for
this message that will produce the appropriate string. By default (that is,
by a method in class Object that will be invoked unless overridden), a
string containing the name of the class of the object is produced. In subsequent chapters we will see several examples of how different classes
respond to the message printString.

\secrel{Collections}

The different subclasses and varieties of Collection provide the means for
managing and manipulating groups of objects in Smalltalk. The different
forms of collections are distinguished by several characteristics, whether
the size of the collection is \&ed or unbounded, the presence or absence
of an ordering, and their insertion or access method. For example, an array
is a collection with a fixed size and ordering, indexed by integer keys. A
dictionary, on the other hand, has no fixed size or ordering and can be
indexed by arbitrary elements. Nevertheless, arrays and dictionaries share
many features, such as their access method (at: and at:put:) and their ability
to respond to collect:, select:, and many other messages.

The table below lists some of the characteristics of several forms of
collections.

\fig{language/tbl1.png}{width=\textwidth}

Collections of one type can frequently be converted into collections of
different type by sending an appropriate message, for example, asBag (to
convert a collection into a Bag), asSet or asArray.

We can group the operations into several categories, independent of
the type of collection involved. The first basic action is adding an element
to a collection. Here collections divide into two groups. Those collections
that are indexed (Dictionary, Array) must be given an explicit key and
value, and, thus, the insertion method is the two-argument message at.·put:.
Those collections that are not indexed store only the value and thus use
the one argument message add:. A special case of this is the class List,
which maintains elements in a linear ordering. Here, values can be added
to the beginning or end of the collection by using the messages addFirst:
and addLast:.

Protocols for adding an element to a collection are similar to those for
removing an element from a collection. In collections that do not require
a key, an element can be removed with the message remove:, the argument
being the object to be removed. In keyed collections, the removal message
uses the key (removeKey:), and not the value. In collections with fixed sizes
(Array and String), elements cannot be removed. In a List, an element
can be removed from either the beginning of the list (removeFirst) or the
end of the list (removeLast).

Once an element has been placed into a collection, the next step is to
access the element. Those collections using keys require a key for access
and use the message at:. For those that do not require a key, the only
question (since one already has the value)-is whether the value is in the
collection. Thus the appropriate message is includes: (which also works
for keyed collections). A special case is List where one can access either
the beginning or the end of the list by using the messages first and last.

The access methods at: and includes: access a value by position. Frequently, however, one needs to access an element by value without knowing
a position. For example, one may want to find the first positive element
in an array of integers. To facilitate this search there is a message named
detect:. The message detect: takes as an argument a one-parameter block.

It evaluates the block on each element in the collection and returns
the value of the first element for which the block evaluates true. For example, if x is an array containing numeric values, the message detect: could
be used to discover the first positive value.
\begin{lstlisting}
    x # ( - 2 - 3 4 5)
    x detect: [ :y Iy > 0 ]
4
\end{lstlisting}

An error message is produced and nil returned if no value satisfies the
condition. This can be changed using the message detect:ifAbsent:
\begin{lstlisting}
        x detect: [ :y I y > 10 ]
    error: no element satisfies condition
    nil
        x detect: [ :y I y > 10] ifAbsent: [ 23 ]
    23
\end{lstlisting}

In ordered collections, the search is performed in order, whereas in
unordered collections, the search is implementation dependent, and no
specific order is guaranteed.

If, instead of finding the first element that satisfies some condition,
you want to find all elements of a collection that satisfy some condition,
then the appropriate message is select:. Like detect:, select: takes as an
argument a one-parameter block. What it returns is another collection, of
the same type as the receiver, containing those values for which the argument block evaluated true. A similar message, reject:, returns the complementary set.
\begin{lstlisting}
        x select: [ :y Iy > 0 ]
    #( 45)
        x reject: [ :y Iy > 0 ]
    #( -2 -3)
\end{lstlisting}

The message do: can be used to perform some computation on every
element in a collection. Like select: and reject:, this message takes a oneargument block. The action performed is to evaluate the block on each
element of the collection.
\begin{lstlisting}
        x do: [ :y I(y + 1 ) print]
    -1
    -2
    5
    6
\end{lstlisting}

The message do: returns nil as its result. If, instead of performing a
computation on each element, you want to produce a new collection containing the results, the message collect: can be used. Again like select: and
reject:, this message takes as argument a one-parameter block and returns
a collection of the same variety as the receiver. The elements of the new
collection, however, are the results of the argument block on each element
of the receiver collection.
\begin{lstlisting}
    x collect: [ :y Iy sign]
#( -1 -1 1 1 )
\end{lstlisting}

Frequently the solution to a problem will involve processing all the
values of a collection and returning a single result. An example would be
taking the sum of the elements in a numerical array. In Little Smalltalk,
the message used to accomplish this is inject:into: The message inject:into:
takes two arguments: a value and a two-parameter block. The action performed in response to this message is to loop over each element in the
collection, passing the element and either the initial value or the result of
the last iteration as arguments to the block. For example, the sum of the
array x could be produced using inject:
\begin{lstlisting}
    x inject: 0 into: [ :a :b I a + b ]
4
\end{lstlisting}
The following command returns the number of times the value 4 occurs
in x:
\begin{lstlisting}
    x inject: 0 into: [ :a :b I(a = = 4) ifTrue: [ b + 1 ] ifFalse: [ b ]]
1
\end{lstlisting}

We have described the broad categories of messages used by collections. There are many other messages specific to certain classes; they are
described in detail in Appendix 3. We next will provide a brief overview
of the most common types of collections.

The classes Bag and Set represent unordered groups of elements. An
element may appear any number of times in a bag but only once in a set.
Elements are added and removed by value.

A Dictionary is also an unordered collection of elements; however,
unlike a bag, insertions and removal of elements from a dictionary requires
an explicit key. Both the key and value portions of a dictionary entry can
be any object, although commonly the keys are instances of String, Symbol or Number

The class Interval represents a sequence of numbers in an arithmetic
progression, either ascending or descending. Instances of Interval are
created by numbers in response to the message to: or to:by:. In conjunction
with the message do:, an Interval creates a control structure similar to do
or for loops in Algol-like languages.
\begin{lstlisting}
        (1 to: 10 by: 2) do: [ :x I x print]
    1
    3
    5
    7
    9
\end{lstlisting}
Although instances of class Interval can be considered to be a collection,
they cannot have additional elements added to them. They can, however,
be accessed randomly using the message at:.
\begin{lstlisting}
    (2 to: 7 by: 3) at: 2
5
\end{lstlisting}

A List is a group of objects having a specific linear ordering. Insertion
and removal is from either the beginning or the end of the collection. Thus
a list can be used to implement both a stack and queue.

A File is a type of collection in which the elements of the collection
are stored on an external medium, typically a disk. A file can be opened
in one of three modes. In character mode every access or read returns a
single character from the file. In Integer mode every read returns a single
word as an integer value. In string mode every read returns a single line
as an instance of class String. Elements cannot be removed from a file,
although they may be overwritten. Because access to external devices is
typically slower than access to memory, many of the operations on files
may be quite slow.

An Array is perhaps the most commonly used data structure in Little
Smalltalk programs. Arrays have fixed sizes, and, while elements cannot
be inserted or removed from an array, the elements can be overwritten.
Literal arrays can be represented by a pound sign preceding a list of array
elements, for example:
\begin{lstlisting}
#( 2 $a 'joel 3.1415 )
\end{lstlisting}

A String can be considered to be a special form of array, where the
elements must be characters. In addition, as we have been illustrating in
many examples, a literal string can be written by surrounding the text with
quote marks.

The class ByteArray represents a special form of array where each
element must be a number in the range 0 through 255. Byte arrays are
used extensively in the internal representations of objects in the Little
Smalltalk system. Byte arrays can be written as a pound sign preceding a
list of elements enclosed in square braces, for example:
\begin{lstlisting}
    #[ 0 127 32 115 ]
\end{lstlisting}

There are two other classesthat are commonly used to representgroups
of data, although they are not subclasses of Collection. The class Point,
already discussed, can be considered to be a small collection of two items.
The class Random can be thought of as providing protocol for an infinite
collection of pseudo-random numbers. This "list," of course, is never actually created in its entirety; rather each number is generated as required
in response to the message next. The values produced by instances of class
Random are floating values in the range 0.0 to 1.0. Other messages can be
used to convert this into either an integer or a floating value in any range.

\secrel{Control Structures}

One of the more surprising aspects of Smalltalk is the fact that control
structures are not provided as part ofthe basic syntax but rather are defined
using the message passing paradigm. The basic control structure in Smalltalk, as in most computer languages, is the conditional test: IF some condition is satisfied THEN perform some actions ELSE perform some other
actions. In Smalltalk this is accomplished by passing messages to instances
of class Boolean. The class True (a subclass of Boolean) defines methods
for the messages ifTrue: and ifFalse: (similar methods are defined for class
False). The arguments used with these messages are blocks. If the condition is satisfied (Le., the receiver is true and the message is ifTrue:, or
the receiver is false and the message is ifFalse:), the argument block is
evaluated, and the result it produces is returned. If the condition is not
satisfied, the value nil is returned.
\begin{lstlisting}
        (3 < 5) ifTrue: { 17 ]
    17
        (3 < 5) ifFalse: { 17 ]
    nil
\end{lstlisting}

The combined forms ifTrue:ifFalse: and ifFalse:ifTrue: are also recognized:
\begin{lstlisting}
        (3 < 5) ifTrue: { 17 ] ifFalse: { 23 ]
    17
        (3 > 5) ifTrue: { 17 ] ifFalse: { 23 ]
    23
\end{lstlisting}

The message and: and or: are similar to ifTrue: and ifFalse:. They are
also used with booleans and passed as arguments objects.of class Block.
And: i and or: provide "short circuit" evaluation of booleans; that is, the
argument block is evaluated only if necessary to determine the result of
the boolean expression.
\begin{lstlisting}
((i < 10) and: { (b at: i) = 4] ifTrue: { i print]
\end{lstlisting}
In this example, the expression "(b at:i) =4" will be evaluated only if the
expression <10<10)" is true. If the first expression returns false, the argument block used with the and: message is not 
evaluated\note{In actual fact the Little Smalltalk parser will, for efficiency, often optimize conditions
to remove the message passing overhead. Nevertheless, the underlying paradigm holds true
and will, in fact, be used under some conditions (for example, when the arguments are not
a block).}.' Notice that the
relational < returns either true (an instance of class True) or false (an
instance of class False) and that the message and: is implemented in class
Boolean, a superclass of both True and False. Various other boolean
operations, such as not are also defined in this class.

Next to conditional tests, the most common control structure is a loop.
A loop is produced by passing the timesRepeat: message with a parameterless bock as an argument to an integer. The value of the integer is the
number of times to execute the loop. For example;
\begin{lstlisting}
        5 timesRepeat:{ 8 print]
    8
    8
    8
    8
    8
\end{lstlisting}
will print the number 8 five times.

A more general loop is used to produce numbers in arithmetic progression. The messages to: and to:by:, when passed to a number, produce
an instance of class Interval. As we noted in the last section, an interval
is a collection of values in arithmetic progression. We can then use the
message do: to enumerate the elements in the progression. For example:
\begin{lstlisting}
        (2 to: 9 by: 3) do: [ :x I x print]
    2
    5
    8
\end{lstlisting}

A more general form of loop is the while loop. A while loop is formed
using blocks as both receiver and argument. The result of the receiver
block must be a boolean. The actions performed by the block are to evaluate
the receiver block and, as long as it returns true, to evaluate the argument
block. For example, by using an additional variable the previous loop could
have been written:
\begin{lstlisting}
        i~2.
        [ i < = 9 ] whileTrue: [ i print. i i + 3 ]
    2
    5
    8
\end{lstlisting}
Since both the receiver and the argument block can contain any number
of expressions (the value of a block is always the value ofthe last expression,
regardless of how many expressions the block contains), sometimes no
argument block is necessary. The unary message whileTrue (or whileFalse) can then be used. For example:
\begin{lstlisting}
        i~2.
        [i print. i i + 3 . i < = 9] whileTrue.
    2
    5
    8
\end{lstlisting}

\secrel{Class Management}

The class Class is used to provide protocol for manipulating classes. Thus,
for example, the methods new and new: are implemented in class Class
to allow instance creation. Classes themselves cannot be created by new
but must be generated by compilation (which is the topic of the next
chapter).

The messages new and new: are treated differently in one respect by
the Little Smalltalk system: if a class defines a method for these messages,
then, each time a new instance of the class is created (by sending the
message new or new: to the class object), the newly created object is immediately initialized by sending it the same message, and the resulting
object is returned to the user. This happens at all levels of the class hierarchy, even if the message is defined multiple times. (That is, later
definitions of new are in addition to, and do not override, definitions higher
in the class hierarchy). One should be careful to distinguish the message
new passed to the class object used to create the object from the same
message passed to the newly created object used for initialization. Since
the second message is produced internally, and not by the user, it is easy
to overlook.
\begin{lstlisting}
        Array new: 3
    #( nil nil nil )
        UnoefinedObject new
    nil
\end{lstlisting}

The argument used with new: is not used by the object creation protocol
but only by the object initialization method. In later chapters we will see
how this feature can be used to automatically initialize objects.

The class Smalltalk provides protocol for the pseudo variable smalltalk. By passing messages to smalltalk, the user ~an discover and set
many attributes of the currently executing environment.
\begin{lstlisting}
    smaIitaIk date
Fri May 24 14:03: 16 1985
\end{lstlisting}

Another message, time:, requires a block as argument. The integer value
it returns represents the number ofseconds elapsed in evaluating the block.
\begin{lstlisting}
    smalltalk time:[ ( 1 to: 10000 ) do: [:x I] ]
104
\end{lstlisting}

Smalltalk is a subclass of Dictionary and thus responds to the messages at: and at:put:. Since smalltalk is accessible anywhere in any object,
it can be used to pass information from one object to another or to provide
global information used by a number of objects. Of course, it is the usees .
responsibility to insure that two objects do not try to store different information using the same key. With the exception of message passing, this
pseudo variable is the only means of communication between separate
objects. Although permitted, the use of the pseudo-variable in this manner
is at odds with the pure object-oriented philosophy of Little Smalltalk and
should be discouraged. The necessity for global variables is often the mark
of a poorly developed solution to a problem.

The pseudo variable smalltalk also provides a means to construct and
evaluate messages at run time by using the message perfonn:with
Arguments:. The first argument to this message must be a symbol indicating
the message to be processed. The second argument must be an array
representing the receiver and arguments to be processed. The second argument must be an array representing the receiver and arguments to be
used in evaluating the message. The response is the value returned by the
first argument of this array in response to the message, with the remainder
of the arguments in the second array as the argument values for the message. For example:
\begin{lstlisting}
    smalltalk perform: #between:and: withArguments: #(3 1.03.14)
True
\end{lstlisting}

An instance of class Process is used to represent a sequence of Smalltalk statements in execution. Processes cannot be created directly by
the user but are created by the system or by passing the message newProcess or fork to a block. Processes will be discussed in more detail in
Chapter 10.

\secrel{Abstract Superclasses}

We did not discuss the classes Collection, KeyedCollection,
SequenceableCollection, or ArrayedCollection in the last section, even
though they were listed as forms of "collection" in the beginning of this
chapter. These classes, along with such classes as Boolean, Magnitude,
or Number, are what are known as abstract superclasses. Instances of
abstract superclasses are seldom useful by themselves, but the classes are
important in providing methods that can be inherited by subclasses. For
example, while it is legal in Smalltalk to say:
\begin{lstlisting}
x Collection new
\end{lstlisting}
and the resulting variable x will indeed be an instance of class Collection,
the object is not particularly useful. It has no insertion or deletion protocol,
for example. An instance of class Set, however, is very useful, and the
messages defined in class Collection are important in providing functionality to objects of this class and to other subclasses of class Collection.

The selection and design of abstract superclasses is one of the more
important arts in Smalltalk programming. For example, if one were designing a system to manipulate banking accounts, a class Account might be
a useful abstract superclass for classes CheckingAccount and
SavingsAccount. The actions specific to the individual types of accounts
would be in the subclasses, whereas any common behavior, such as the
actions necessary for opening and closing an account or querying the
balance, might be implemented in the superclass.


\secly{EXERCISES}

\begin{enumerate}

\item Suppose you created a new instance of the class UndefinedObject, as
follows:
\begin{lstlisting}
    I~UndefinedObject new
\end{lstlisting}
How does i respond to print? To isNil or notNil? To the object comparison message = = with nil ? Is i nil? List facts to support your
answer.

\item Note that the messages arcSin and arcCos produce an object of type
Radian and not oftype Float. Furthermore, only objects oftype Radian
respond to sin and cos. An alternative would have been to eliminate
the class Radian and to permit all objects of class Float to respond
to the messages sin and cos. Discuss the advantages and disadvantages
of these two different arrangements.

\item Suppose you have a Bag containing numbers. How would you go about
producing an instance of the class List containing the numbers listed
in sorted order?

\item What is the class of Class? what is the superclasss of Class?

\item Many times, two ormore different sequences ofmessages to collections
will have the same effect. In each of the following, describe a sequence
of messages that will have the same effect ,\S the indicated message.
\begin{enumerate}
    \item implement reject: in terms of select:
    \item implement size in terms of inject:into:
    \item implement includes: in terms of inject:into:.
\end{enumerate}

\end{enumerate}

\secup
