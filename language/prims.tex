\secrel{Primitives}

At this point you may be wondering how anything ever gets accomplished in
Smalltalk. As described so far, actions are produced by one object sending a
message to another object. In response to this message, the second object may in
turn send still other messages to more objects. If this were all there were to
the language, there would seem to be a sort of infinite regress preventing any
substantial action from taking place.

The solution to this problem is yet another form of expression, called a
primitive. A primitive can be thought of as a system call, permitting access to
the underlying virtual machine. Syntactically, a primitive is denoted by an
angle bracket followed by a name followed by a list of objects followed by a
closing angle bracket, for example:
\begin{lstlisting}
<lntegerAddition i j>
\end{lstlisting}
Alternatively, every primitive is also assigned a number. The primitive name can
be replaced by the keyword "primitive" and this number. Thus, since the integer
addition primitive is number 10, the above expression could also be written as:
\begin{lstlisting}
<primitive 10 i j >
\end{lstlisting}
However, the second form is less representative of the operation being
performed. Primitives are chiefly used in class descriptions and should almost
never by typed directly at the command level. In any case, only the second form
(expressing the desired primitive by number) is recognized at the command level.

Note that a primitive call is not the same as a message passing, although the
differences are largely hidden from the user. There is no notion of a "receiver"
in a primitive expression, and a primitive cannot send further messages in
producing its response. For this reason most primitive operations, as the name
suggests, are very simple. A table of primitives is given in Appendix 4. In
general the value nil is returned, and an error message produced, if arguments
to a primitive are not correct or the primitive cannot produce the desired
result.

A portion ofthe class description for the class Radian is given in Figure 6.1.
This class uses the primitives Sin and Cos to compute the value of the
trigonometric functions on radian inputs. Notice how the class Radian responds
to the message printString by concatenating the string representation of its
value with the literal string "radians." Thus the message IfX print," if x is a
radian with value 0.85, will produce the message "0.85 radians." This technique
is used in many classes to provide more informative printed representations of
objects.

There is a temptation on the part of many novice Little Smalltalk users to use
primitives in place of Smalltalk expressions in the name of llefficiency." This
should be avoided. Such gains in efficiency are usually negligible, if they
exist at all. More importantly, the notion of primitive is really just a
pragmatic device permitting some operations to be specified that could not
otherwise be given in Smalltalk. The syntax is a bit obscure, and the notion
does not fit quite smoothly into the object-oriented framework of the rest of
Smalltalk. Thus primitives should be used only as a last resort, only after a
great deal of thought as to the alternatives, and only in extremely simple
methods that put a more lfSmalltalk-like" cloak over the primitive call.

\fig{language/fig_6_1.png}{height=.9\textheight}
