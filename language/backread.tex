\secrel{История, Фоновое чтение}

Концепции, относящиеся к объектно-ориентированному программированию, 
заложенные в \st, являются результатом длительного процесса 
разработки и эволюции языка. Основные понятия об объектах, сообщениях 
и классах пришли из языка Simula (Birtwistle 73). Хотя Simula 
позволяла пользователям создавать объектно-ориентированные системы, 
и классы, ответ на сообщение (эквивалент метода в Simula) все еще 
выражался в стандартном для ALGOL методе, ориентированном 
на данные и процедуры.

В семействе языков ALGOL концепция классов привела к разработке понятия 
модулей и абстрактных типов данных (Shaw 80), поддержка которых была 
фундаментальной целью в нескольких языках, таких как Euclid, CLU, Modula и Ada.

В то время как объектно-ориентированная философия постепенно получала признание 
в мире языков программирования, подобные идеи получили признание в сообществе 
архитекторов (Pinnow 82). Точно так же в дизайне операционных систем понятие 
независимых вычислений, которые взаимодействуют друг с другом исключительно 
путем обмена сообщениями, находило сторонников (Wulf 74), (Almes 85). Такое 
представление является естественным и удобным, когда вычисления могут 
физически выполняться на распределенных процессорах.

Прямые предки Smalltalk включают систему Flex (Kay 69), Smalltalk-72 (Goldberg 76) 
и Smalltalk-76 (Ingalls 78). Все языки Smalltalk были созданы в рамках 
проекта Dynabook, инициированного Аланом Кейем в Группе исследований 
обучения в Исследовательском центре Xerox в Пало-Альто. Эволюция языка, 
как показано в этих документах, показывает, что объектно-ориентированная 
модель постепенно расширяется и включает в себя все больше и больше языковых 
концепций. Например, в Smalltalk-72 числа и управляющие структуры обрабатываются 
как объекты, в отличие от Simula, но классы по-прежнему представляют собой 
особую форму. В Smalltalk-76 описания классов представлены в виде объектов, 
а объектно-ориентированная точка зрения расширена до интерфейса программирования. 
Этот интерфейс почти полностью описывается в объектно-ориентиро\-ван\-ной форме 
в среде программирования Smalltalk-80 (Goldberg, 83).

Объектно-ориентированный взгляд на программирование также повлиял на 
другие компьютерные языки, в частности, на понятия актеров (Hewit 73) и 
разновидностей (Weinreb 80) в Лиспе, а также на разработку языков для 
анимации и графики (Рейнольдс 82). Развитие акторов в Лиспе шло параллельно 
с развитием Smalltalk, и два языка влияли друг на друга.

Ковед и ЛаЛонд представляют обзоры, описывающие объектно-ориентированную 
точку зрения в различных ипостасях (Ковед 84) (ЛаЛонд 84). Ряд статей, 
описывающих различные аспекты системы Smalltalk-80, были включены в 
специальный выпуск журнала Byte (Байт 81).
