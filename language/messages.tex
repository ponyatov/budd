\secrel{Messages}

As noted in Chapter I, all actions in Smalltalk are produced by sending
messages to objects. This section begins by describing the syntax used to
produce messages.

Any message can be divided into three parts; a receiver, a message
selector, and zero or more arguments. The receiver and argument portions
of a message can be specified by other message expressions, or they may
be specified by a single token, such as an identifier or a literal.

The first type of message selector requires no arguments and is called
a unary message. A unary message selector consists of an identifier, the
first letter of which must be lowercase. For example:
\begin{lstlisting}
7 sign
\end{lstlisting}
illustrates the message sign being passed to the number 7. Unary messages,
like all messages, elicit a response, which is simply another object. The
response to sign is an integer, either -1,0, or 1, depending upon the sign
of the object the message was sent to (the receiver). Unary messages parse
left to right, so, for example:
\begin{lstlisting}
7 factorial sqrt
\end{lstlisting}
returns $\sqrt{7!}$, or approximately 70.993.    

The second form of message, called a binary message, takes one argument. 
A binary message is formed from one or two adjacent nonalphabetic characters\note{Some 
characters, such as braces, parenthesis or periods, cannot be used to form
binary messages. See the description in Appendix 2 for a more complete description of the
restrictions.}.
Binary messages tend to be used for arithmetic
operations, although this is not enforced by the system and there are
notable exceptions. An example of a binary message is arithmetic addition:
\begin{lstlisting}
7 + 4
\end{lstlisting}
At first the fact that this is interpreted as "send the message + with
argument 4 to-the object 7"may seem strange; however, soon the uniform
treatment of objects and message passing in Smalltalk makes this seem
natural.

Binary messages, like unary messages, parse left to right. Thus
\begin{lstlisting}
7 + 4 * 3
\end{lstlisting}
results in 33, not 19. \emph{Unary messages have a higher precedence than binary
messages}, thus
\begin{lstlisting}
7 + 17 sqrt
\end{lstlisting}
evaluates as $7 + (\sqrt{17})$, not $\sqrt{(7 + 17)}$.

The most general type of message is a keyword message. The selector
for a keyword message consists of one or more keywords. Each keyword
is followed by an argument. A keyword is simply an identifier (again, the
first character must be lower case) followed by a colon. The argument can
be any expression, although if the expression is formed using a keyword
message, it must be placed in parentheses to avoid ambiguity. Example
keyword expressions are:
\begin{lstlisting}
7 max: 14.
7 between: 2 and: 24
\end{lstlisting}

When we wish to express the name of the message being requested by
a keyword message, we catenate the keyword tokens. Thus we say the
message selector being expressed in the second example above is between:and:. There can be any number of keywords in a keyword message,
although in practice few messages have more than three.

Keyword messages have lower precedence than either binary or unary
messages. Thus
\begin{lstlisting}
7 between: 2 sqrt and: 4 + 2
\end{lstlisting}
\begin{lstlisting}
7 between: (2 sqrt) and: (4 + 2)
\end{lstlisting}
            