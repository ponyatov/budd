\secrel{Class Management}

The class Class is used to provide protocol for manipulating classes. Thus,
for example, the methods new and new: are implemented in class Class
to allow instance creation. Classes themselves cannot be created by new
but must be generated by compilation (which is the topic of the next
chapter).

The messages new and new: are treated differently in one respect by
the Little Smalltalk system: if a class defines a method for these messages,
then, each time a new instance of the class is created (by sending the
message new or new: to the class object), the newly created object is immediately initialized by sending it the same message, and the resulting
object is returned to the user. This happens at all levels of the class hierarchy, even if the message is defined multiple times. (That is, later
definitions of new are in addition to, and do not override, definitions higher
in the class hierarchy). One should be careful to distinguish the message
new passed to the class object used to create the object from the same
message passed to the newly created object used for initialization. Since
the second message is produced internally, and not by the user, it is easy
to overlook.
\begin{lstlisting}
        Array new: 3
    #( nil nil nil )
        UnoefinedObject new
    nil
\end{lstlisting}

The argument used with new: is not used by the object creation protocol
but only by the object initialization method. In later chapters we will see
how this feature can be used to automatically initialize objects.

The class Smalltalk provides protocol for the pseudo variable smalltalk. By passing messages to smalltalk, the user ~an discover and set
many attributes of the currently executing environment.
\begin{lstlisting}
    smaIitaIk date
Fri May 24 14:03: 16 1985
\end{lstlisting}

Another message, time:, requires a block as argument. The integer value
it returns represents the number ofseconds elapsed in evaluating the block.
\begin{lstlisting}
    smalltalk time:[ ( 1 to: 10000 ) do: [:x I] ]
104
\end{lstlisting}

Smalltalk is a subclass of Dictionary and thus responds to the messages at: and at:put:. Since smalltalk is accessible anywhere in any object,
it can be used to pass information from one object to another or to provide
global information used by a number of objects. Of course, it is the usees .
responsibility to insure that two objects do not try to store different information using the same key. With the exception of message passing, this
pseudo variable is the only means of communication between separate
objects. Although permitted, the use of the pseudo-variable in this manner
is at odds with the pure object-oriented philosophy of Little Smalltalk and
should be discouraged. The necessity for global variables is often the mark
of a poorly developed solution to a problem.

The pseudo variable smalltalk also provides a means to construct and
evaluate messages at run time by using the message perfonn:with
Arguments:. The first argument to this message must be a symbol indicating
the message to be processed. The second argument must be an array
representing the receiver and arguments to be processed. The second argument must be an array representing the receiver and arguments to be
used in evaluating the message. The response is the value returned by the
first argument of this array in response to the message, with the remainder
of the arguments in the second array as the argument values for the message. For example:
\begin{lstlisting}
    smalltalk perform: #between:and: withArguments: #(3 1.03.14)
True
\end{lstlisting}

An instance of class Process is used to represent a sequence of Smalltalk statements in execution. Processes cannot be created directly by
the user but are created by the system or by passing the message newProcess or fork to a block. Processes will be discussed in more detail in
Chapter 10.
