\secrel{Литеральные константы}

Некоторые объекты, \term{литералы}, отличаются тем, что их имя однозначно идентифицирует 
класс и значение объекта, независимо от контекста, и тем фактом, что их не нужно 
объявлять перед использованием. Например, символ 7, независимо от того, где он 
появляется, всегда обозначает один и тот же объект. В Algol-подобных языках 
такой символ, как 7, обычно обозначает «значение», а не идентификатор. В
\st\ это различие гораздо менее отчетливо. Все объекты, включая числа, являются 
объектами, а объекты характеризуются сообщениями, которые они принимают, и их 
ответами на них. Таким образом, 7 обозначает объект так же, как идентификатор, 
такой как \var{x} (в надлежащем контексте), может обозначать объект.

Числа, пожалуй, самые распространенные литеральные объекты. Существует два класса 
чисел, которые могут быть записаны как литеральные объекты, а именно целые числа и 
значения с плавающей точкой. Числа отвечают на различные арифметические сообщения 
(унаследованные от класса \class{Number}) и сообщения отношений (унаследованные от класса 
\class{Magnitude}). Экземпляр класса \class{Integer} состоит из необязательного знака, за которым 
следует любое количество цифр. Число с плавающей запятой состоит из целого числа, 
за которым следует точка (десятичная точка) и другого целого числа без знака 
(дробная часть) и/или буквы \verb|е| и целого числа со знаком (экспоненциальная часть). 
Любому числу может предшествовать основание системы счисления, которое представляет собой 
положительное целое число, за которым следует буква \verb|r|. Для оснований больше 10 
буквы от A до Z интерпретируются как цифры. Примеры чисел:

\begin{lstlisting}
7
16rFF
-3.1415926
2e32
2.4e-32
15rC.ABC
\end{lstlisting}
    
\noindent
Основание системы счисления в основном используется просто ради удобства и внешнего вида. 
Число 16rFF совпадает с числом 10r255 или просто 255.
    
Класс \class{Char} предоставляет возможности для работы со значениями букв. Буквы отличаются 
от цифр. Поскольку символы имеют порядок, заданный последовательностью сортировки, 
их можно сравнивать и, следовательно, они являются подклассом класса \class{Magnitude}. 
Символ написан в виде знака доллара, за которым следует буква (или цифра). 
Ниже приведены примеры экземпляров этого класса:
    
\begin{lstlisting}
$A
$7
$
$$
\end{lstlisting}
    
Экземпляр класса \class{String} представлен последовательностью букв между одинарными кавычками. 
Встраивание кавычки в строку требует двух соседних кавычек. Строка похожа на массив; 
фактически класс \class{String} является подклассом \class{ArrayedCollection}, как и класс \class{Array}. 
И строки, и массивы могут быть объединены вместе для формирования больших строк 
с помощью оператора \term{конкатенации} строк\ --- запятой (,). Примеры строк:
    
\begin{lstlisting}
'a String'
'a String with an '' embedded quote mark'
\end{lstlisting}
    
Массив \class{Array} записывается в виде знака фунта (\verb|#|), за которым следует список элементов 
массива в скобках. Элементами массива являются литеральные объекты (числа или символы), 
строки или другие массивы. В списке массивов ведущий знак фунта на символах 
и массивах может быть исключен. Примеры массивов:

\begin{lstlisting}
#(this is an array of symbols)
#(12 'abc' (another array))
\end{lstlisting}

\noindent
Массивы и строки используют сообщения \verb|at:| и \verb|at:put:| 
для выбора и изменения определенных элементов в их коллекциях.

Класс \class{Symbol}\ --- это еще один литеральный класс. Символ записывается в виде 
знака фунта (\verb|#|), за которым следует любая последовательность букв. 
Пробелы между символами не допускаются. В отличие от строки (которая также 
является последовательностью букв) символ не может быть разбит на более 
мелкие части. Кроме того, одна и та же последовательность букв, используемая 
в разных местах, всегда будет обозначать один и тот же объект. В отличие от 
чисел, символов или букв, символы не имеют порядка и не могут сравниваться 
(за исключением, конечно, равенства объектов). Примеры символов:

\begin{lstlisting}
#aSymbol
#AndAnother
#+++
#very.long.symbol.with.periods
\end{lstlisting}
