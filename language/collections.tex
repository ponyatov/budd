\secrel{Collections}

The different subclasses and varieties of Collection provide the means for
managing and manipulating groups of objects in Smalltalk. The different
forms of collections are distinguished by several characteristics, whether
the size of the collection is \&ed or unbounded, the presence or absence
of an ordering, and their insertion or access method. For example, an array
is a collection with a fixed size and ordering, indexed by integer keys. A
dictionary, on the other hand, has no fixed size or ordering and can be
indexed by arbitrary elements. Nevertheless, arrays and dictionaries share
many features, such as their access method (at: and at:put:) and their ability
to respond to collect:, select:, and many other messages.

The table below lists some of the characteristics of several forms of
collections.

\fig{language/tbl1.png}{width=\textwidth}

Collections of one type can frequently be converted into collections of
different type by sending an appropriate message, for example, asBag (to
convert a collection into a Bag), asSet or asArray.

We can group the operations into several categories, independent of
the type of collection involved. The first basic action is adding an element
to a collection. Here collections divide into two groups. Those collections
that are indexed (Dictionary, Array) must be given an explicit key and
value, and, thus, the insertion method is the two-argument message at.·put:.
Those collections that are not indexed store only the value and thus use
the one argument message add:. A special case of this is the class List,
which maintains elements in a linear ordering. Here, values can be added
to the beginning or end of the collection by using the messages addFirst:
and addLast:.

Protocols for adding an element to a collection are similar to those for
removing an element from a collection. In collections that do not require
a key, an element can be removed with the message remove:, the argument
being the object to be removed. In keyed collections, the removal message
uses the key (removeKey:), and not the value. In collections with fixed sizes
(Array and String), elements cannot be removed. In a List, an element
can be removed from either the beginning of the list (removeFirst) or the
end of the list (removeLast).

Once an element has been placed into a collection, the next step is to
access the element. Those collections using keys require a key for access
and use the message at:. For those that do not require a key, the only
question (since one already has the value)-is whether the value is in the
collection. Thus the appropriate message is includes: (which also works
for keyed collections). A special case is List where one can access either
the beginning or the end of the list by using the messages first and last.

The access methods at: and includes: access a value by position. Frequently, however, one needs to access an element by value without knowing
a position. For example, one may want to find the first positive element
in an array of integers. To facilitate this search there is a message named
detect:. The message detect: takes as an argument a one-parameter block.

It evaluates the block on each element in the collection and returns
the value of the first element for which the block evaluates true. For example, if x is an array containing numeric values, the message detect: could
be used to discover the first positive value.
\begin{lstlisting}
    x # ( - 2 - 3 4 5)
    x detect: [ :y Iy > 0 ]
4
\end{lstlisting}

An error message is produced and nil returned if no value satisfies the
condition. This can be changed using the message detect:ifAbsent:
\begin{lstlisting}
        x detect: [ :y I y > 10 ]
    error: no element satisfies condition
    nil
        x detect: [ :y I y > 10] ifAbsent: [ 23 ]
    23
\end{lstlisting}

In ordered collections, the search is performed in order, whereas in
unordered collections, the search is implementation dependent, and no
specific order is guaranteed.

If, instead of finding the first element that satisfies some condition,
you want to find all elements of a collection that satisfy some condition,
then the appropriate message is select:. Like detect:, select: takes as an
argument a one-parameter block. What it returns is another collection, of
the same type as the receiver, containing those values for which the argument block evaluated true. A similar message, reject:, returns the complementary set.
\begin{lstlisting}
        x select: [ :y Iy > 0 ]
    #( 45)
        x reject: [ :y Iy > 0 ]
    #( -2 -3)
\end{lstlisting}

The message do: can be used to perform some computation on every
element in a collection. Like select: and reject:, this message takes a oneargument block. The action performed is to evaluate the block on each
element of the collection.
\begin{lstlisting}
        x do: [ :y I(y + 1 ) print]
    -1
    -2
    5
    6
\end{lstlisting}

The message do: returns nil as its result. If, instead of performing a
computation on each element, you want to produce a new collection containing the results, the message collect: can be used. Again like select: and
reject:, this message takes as argument a one-parameter block and returns
a collection of the same variety as the receiver. The elements of the new
collection, however, are the results of the argument block on each element
of the receiver collection.
\begin{lstlisting}
    x collect: [ :y Iy sign]
#( -1 -1 1 1 )
\end{lstlisting}

Frequently the solution to a problem will involve processing all the
values of a collection and returning a single result. An example would be
taking the sum of the elements in a numerical array. In Little Smalltalk,
the message used to accomplish this is inject:into: The message inject:into:
takes two arguments: a value and a two-parameter block. The action performed in response to this message is to loop over each element in the
collection, passing the element and either the initial value or the result of
the last iteration as arguments to the block. For example, the sum of the
array x could be produced using inject:
\begin{lstlisting}
    x inject: 0 into: [ :a :b I a + b ]
4
\end{lstlisting}
The following command returns the number of times the value 4 occurs
in x:
\begin{lstlisting}
    x inject: 0 into: [ :a :b I(a = = 4) ifTrue: [ b + 1 ] ifFalse: [ b ]]
1
\end{lstlisting}

We have described the broad categories of messages used by collections. There are many other messages specific to certain classes; they are
described in detail in Appendix 3. We next will provide a brief overview
of the most common types of collections.

The classes Bag and Set represent unordered groups of elements. An
element may appear any number of times in a bag but only once in a set.
Elements are added and removed by value.

A Dictionary is also an unordered collection of elements; however,
unlike a bag, insertions and removal of elements from a dictionary requires
an explicit key. Both the key and value portions of a dictionary entry can
be any object, although commonly the keys are instances of String, Symbol or Number

The class Interval represents a sequence of numbers in an arithmetic
progression, either ascending or descending. Instances of Interval are
created by numbers in response to the message to: or to:by:. In conjunction
with the message do:, an Interval creates a control structure similar to do
or for loops in Algol-like languages.
\begin{lstlisting}
        (1 to: 10 by: 2) do: [ :x I x print]
    1
    3
    5
    7
    9
\end{lstlisting}
Although instances of class Interval can be considered to be a collection,
they cannot have additional elements added to them. They can, however,
be accessed randomly using the message at:.
\begin{lstlisting}
    (2 to: 7 by: 3) at: 2
5
\end{lstlisting}

A List is a group of objects having a specific linear ordering. Insertion
and removal is from either the beginning or the end of the collection. Thus
a list can be used to implement both a stack and queue.

A File is a type of collection in which the elements of the collection
are stored on an external medium, typically a disk. A file can be opened
in one of three modes. In character mode every access or read returns a
single character from the file. In Integer mode every read returns a single
word as an integer value. In string mode every read returns a single line
as an instance of class String. Elements cannot be removed from a file,
although they may be overwritten. Because access to external devices is
typically slower than access to memory, many of the operations on files
may be quite slow.

An Array is perhaps the most commonly used data structure in Little
Smalltalk programs. Arrays have fixed sizes, and, while elements cannot
be inserted or removed from an array, the elements can be overwritten.
Literal arrays can be represented by a pound sign preceding a list of array
elements, for example:
\begin{lstlisting}
#( 2 $a 'joel 3.1415 )
\end{lstlisting}

A String can be considered to be a special form of array, where the
elements must be characters. In addition, as we have been illustrating in
many examples, a literal string can be written by surrounding the text with
quote marks.

The class ByteArray represents a special form of array where each
element must be a number in the range 0 through 255. Byte arrays are
used extensively in the internal representations of objects in the Little
Smalltalk system. Byte arrays can be written as a pound sign preceding a
list of elements enclosed in square braces, for example:
\begin{lstlisting}
    #[ 0 127 32 115 ]
\end{lstlisting}

There are two other classesthat are commonly used to representgroups
of data, although they are not subclasses of Collection. The class Point,
already discussed, can be considered to be a small collection of two items.
The class Random can be thought of as providing protocol for an infinite
collection of pseudo-random numbers. This "list," of course, is never actually created in its entirety; rather each number is generated as required
in response to the message next. The values produced by instances of class
Random are floating values in the range 0.0 to 1.0. Other messages can be
used to convert this into either an integer or a floating value in any range.
