
\clearpage
\secly{Предисловие}

\subsecly{Система Little Smalltalk: немного истории}

Весной 1984 года я преподавал курс по языкам программирования в университете Аризоны. 
При подготовке лекций для этого курса я заинтересовался концепцией объектно-ориентированного 
программирования и, в частности, тем, как объектно-ориентированная парадигма изменила 
подход программистов к решению проблем. В течение этого семестра и следующего лета я 
собрал как можно больше материалов об объектно-ориентированном программировании, 
особенно о системе программирования \st-80, разработанной в исследовательском 
центре Xerox Palo Alto (Xerox PARC). Тем не менее, я продолжал расстраиваться из-за 
своей неспособности получить практический опыт написания и использования программ на \st.

В то время единственной системой \st, о которой я знал, была оригинальная система, 
работающая на Dorado, дорогой машине, недоступной (в то время) за пределами Xerox PARC. 
Доступной мне возможностью был VAX-780 с Unix2 использовавший обычные ASCII терминалы. 
Таким образом, оказалось, что мои шансы на запуск системы Xerox \st-80 в 
ближайшей перспективе были весьма невелики; поэтому, несколько студентов и я 
решили летом 1984 года создать нашу собственную систему \st.

Осенью 1984 года мы с дюжиной студентов создали систему Little Smalltalk в рамках 
семинара для выпускников по реализации языка программирования. С самого начала наши 
цели были гораздо менее амбициозными, чем у первоначальных разработчиков системы 
\st-80. Несмотря на то, что мы оценили важность инновационных концепций в 
средах программирования и графике, впервые разработанных группой Xerox, мы до боли 
осознавали наши собственные ограничения, как в рабочей силе, так и в оборудовании. 
Нашими целями в порядке важности были:
\begin{itemize}[nosep]
    \item Новая система должна поддерживать язык, максимально приближенный к 
    опубликованному описанию \st-80 \cite{blue}.
    \item Система должна работать под Unix, используя только обычные текстовые терминалы.
    \item Система должна быть написана на \ci\ и быть максимально переносимой.
    \item Система должна быть маленькой. В частности, она должна работать на 
    16-битных машинах с раздельной памятью команд и данных, 
    но предпочтительно даже на машинах без этой функции.
\end{itemize}

Оглядываясь назад, мы, кажется, достигли наших целей довольно хорошо. Язык, понимаемый 
системой Little Smalltalk, достаточно близок к языку системы программирования 
\st-80, так что пользователи, похоже, испытывают только небольшие трудности 
(по крайней мере, с языком) при переходе от одной системы к другой. Система 
оказалась чрезвычайно переносимой: она была перенесена на дюжину разновидностей 
Unix, работающих на разных машинах. Более 200 сайтов теперь используют 
систему Little Smalltalk.

\subsecly{О системе Little Smalltalk}

Эта книга состоит из двух частей. Первый раздел описывает язык системы Little 
Smalltalk. Хотя большинство читателей, возможно, до знакомства с Smalltalk 
имели некоторый опыт использования хотя бы одного другого языка программирования, 
в тексте не делается никаких предположений относительно подготовки читателя. 
Большинство студентов старших курсов или аспирантов должны быть в состоянии 
понять материал в первом разделе. Эта часть текста может использоваться отдельно.

Вторая часть книги описывает фактическую реализацию системы Little Smalltalk. 
Этот раздел требует от читателя более глубоких знаний в области информатики. 
Поскольку Little Smalltalk написан на C, требуется хотя бы элементарное знание 
этого языка. Также желательная хорошая подготовка по структурам данных. 
Читателю будет желательно, хотя и не обязательно, иметь некоторое 
представление о построении компиляторов для обычного языка, такого как Pascal.


\subsecly{Благодарности}

I am, of course, most grateful to the students in the graduate seminar at the
University of Arizona where the Little Smalltalk system was developed. The many
heated discussions and insightful ideas generated were most enjoyable and
stimulating. Participants in that seminar were Mike Benhase, Nick Buchholz, Dave
Burns, John Cabral, Clayton Curtis, Roger Hayes, Tom Hicks, Rob McConeghy,
Kelvin Nilsen, May Lee Noah, Sean O'Malley, and Dennis Vadner. This text grew
out of notes developed for that course, and includes many ideas contributed by
the participants. In particular I wish to thank Dave Burns for the original
versions of the simulation described in Chapter 7 and Mike Benhase and Dennis
Vadner for their work on processes and the dining philosophers solution
presented in Chapter 10.

Discussions with many people have yielded insights or examples that eventually
found their way into this book. I wish to thank, in particular, Jane Cameron,
Chris Fraser, Ralph Griswold, Paul Klint, Gary Levin, and Dave Robson.

Irv Elshoff provided valuable assistance by trying to learn Smalltalk from an
early manuscript and by making many useful and detailed comments on the text.

J. A. Davis from Iowa State University, Paul Klint from the CWI, David Robson
from Xerox Palo Alto Research Center, and. Frances Van Scoy from West Virginia
University provided careful and detailed comments on earlier drafts of the book.

Charlie Allen at Purdue, Jan Gray at Waterloo and Charles Hayden at AT\&T were
early non-Arizona users of Little Smalltalk and were extremely helpful in
finding bugs in the earlier distributions.

I wish to thank Ralph Griswold, Dave Hanson, and Chris Fraser, all chairmen of
the computer science department at the University of Arizona at various times in
the last five years, for helping to make the department such a pleasant place to
work. Finally I wish to thank Paul Vitanyi and Lambert Meertens for providing me
with the chance to work at the Centrum voor Wiskunde en Informatica in Amsterdam
for the year between my time in Arizona and my move to Oregon, and for
permitting me to finish work on the book while there.

\subsecly{Получение оригинальной системы Little Smalltalk}

\url{https://github.com/crcx/littlesmalltalk} Архив Little Smalltalk 
(с обновлениями для работы на современных платформах). 
Он также собирает форки и документацию по этой исторической системе.

\bigskip

The Little Smalltalk system can be obtained directly from the author. The
system is distributed on 9-track tapes in tar format (the standard unix
distribution format). The distribution tape includes all sources and on-line
documentation for the system. For further information on the distribution,
including cost, write to the following address:

\bigskip\noindent
Smalltalk Distribution\\
Department of Computer Science\\
Oregon State University\\
Corvallis, Oregon\\
97331\\
USA



\secrel{The Language}\secdown

\secrel{Основы}\secdown

Эта глава знакомит с основными понятиями языка Smalltalk; а именно объект, метод, класс, наследование и переопределение.

\bigskip

Традиционной моделью, описывающей поведение компьютера, выполняющего программу, 
является модель состояния процесса, или модель «сортировщика». В этом 
представлении компьютер является диспетчером данных, который следует некоторому 
блоку инструкций, блуждает по памяти, извлекает значения из различных слотов 
(адресов памяти), преобразует их каким-либо образом, и передает результаты 
обратно в другие слоты. Изучая значения в слотах, можно определить состояние 
машины или результаты, полученные вычислением. Хотя это может быть более или 
менее точная картина того, что происходит в компьютере, это мало помогает нам 
понять, как решать проблемы с помощью компьютера, и это, конечно, не те способы, 
которыми думают большинство людей решающих проблемы (за исключением сортировщиков и почтальонов).

Давайте рассмотрим реалистичную ситуацию, а затем посмотрим, как можно заставить 
компьютер более точно моделировать методы, которые люди используют для решения 
проблем в повседневной жизни. Предположим, я хочу послать цветы моей бабушке 
на день рождения. Она живет далеко в городе за много миль от меня. Задача 
достаточно проста для выполнения; Я просто иду к местному флористу, описываю 
виды и количество цветов, которые я хочу отправить, и я могу быть уверен, 
что они будут доставлены автоматически. Если я проведу расследование, я, 
вероятно, обнаружу, что мой флорист отправляет сообщение с описанием моего 
заказа другому флористу в городе моей бабушки. Тот флорист тогда составляет 
букет и доставляет цветы. Я мог бы спросить, чтобы узнать, как цветочный 
магазин в городе моей бабушки покупает цветы и, возможно, узнал, что они 
получены от оптового торговца цветами. Если я продолжу настаивать, я даже 
смогу проследить всю цепочку до фермера, который выращивает цветы, и узнать, 
какие запросы были сделаны членами цепочки, чтобы получить желаемый результат от каждого.

Важным моментом, однако, является то, что мне не нужно, да и вообще, 
я не хочу знать, как будет выполняться моя простая директива 
«отправить цветы моей бабушке». В реальной жизни мы называем этот 
процесс \term{делегирование полномочий}. В информатике это называется 
\term{абстракция} или \term{сокрытие информации}. В основе этих терминов лежит 
одно и то же. Есть ресурс (флорист, файловый сервер), который я хочу 
использовать. Чтобы общаться, я должен знать команды, на которые будет 
реагировать ресурс (отправить цветы моей бабушке, вернуть копию файла 
с именем "chapеer!"). Скорее всего, шаги, которые должен предпринять 
ресурс, чтобы ответить на мой запрос, гораздо более сложны, чем я 
понимаю, но мне нет смысла знать подробности того, как реализуется 
моя директива, до тех пор, пока ответ (доставка цветов, получение 
копии моего файла) четко определено и предсказуемо.

\term{Объектно-ориентированная} модель решения проблем рассматривает компьютер
способом, очень близким к такому подходу. Действительно, многие люди, 
которые не имеют никакого образования в области информатики, и не знают, 
как работает компьютер, находят объектно-ориенти\-рован\-ную модель решения 
проблем вполне естественной. Удивительно, однако, что многие люди, 
имеющие традиционный опыт программирования, изначально думают, что в 
концепции объекта есть что-то странное. Представление о том, что число «7» 
является объектом, а «+» --- это запрос на сложение, может поначалу 
показаться странным. Но вскоре единообразие, мощь и гибкость, которые 
метафора объект/сообщение привносит в решение проблемы, делает эту 
интерпретацию так же естественной.

Вселенная \st\ населена \term{объектами}. В моем примере с цветами я являюсь объектом, 
а цветочный магазин (или флорист в нем)\ --- другим объектом. Действия 
инициируются путем отправки \term{сообщений}\ (запросов) между объектами. Я передал 
просьбу «отправить цветы моей бабушке» флористу-объекту. Реакция \term{получателя}
моего сообщения состоит в том, чтобы выполнить некоторую последовательность 
действий или \term{метод}, чтобы удовлетворить мой запрос. Может быть, получатель 
сможет немедленно удовлетворить мой запрос. С другой стороны, чтобы удовлетворить 
мои потребности, получателю, возможно, придется передавать другие сообщения 
еще большему количеству объектов (например, сообщение, которое мой флорист 
отправляет флористу в городе моей бабушки, или команду на дисковод). Кроме 
того, существует явный ответ (например, квитанция или код результата), 
возвращенный непосредственно мне. Дэн Ингаллс описывает философию \st\ (Байт 81):

\begin{quote}
    Вместо того, чтобы обрабатывать структуры данных, которые насилуют и грабят 
    процессор, мы имеем вселенную объектов с хорошим поведением, которые вежливо 
    просят друг друга выполнить их различные желания.
\end{quote}
    
Такие антропоморфные точки зрения распространены среди \st-программистов. 
В последующих главах мы увидим, как язык Smalltalk воплощает этот 
объектно-ориентированный взгляд на программирование. Описывая решение 
нескольких проблем в \st, мы надеемся показать, как объектно-ориентированная 
модель помогает в создании больших программных систем, и помогает в решении 
многих проблем с использованием компьютера.

\secrel{Объекты, классы и наследование}

В Smalltalk все является объектом. В языке нет способа создать сущность, 
которая не является объектом. Среди основных компьютерных языков это 
единообразие в Smalltalk конкурирует, возможно, только с LISP, и, 
как и в LISP, единообразие создает и простоту, и мощь языка.

Объект обладает несколькими характеристиками (рисунок 1.1). Каждый объект 
содержит небольшой объем памяти, доступный только этому объекту. То есть 
ни один объект не может читать или изменять значения памяти в другом 
объекте. Конечно, поскольку всё в системе должно быть объектом, память 
объекта может содержать только ссылки на другие объекты. Мы обсудим 
это более подробно позже.

\fig{language/fig_1_1.png}{height=\textheight}

Все действия в системе \st\ производятся путем передачи сообщений. 
Сообщение\ --- это запрос у объекта выполнения какой-либо операции. Оно 
может содержать некоторые значения в качестве аргументов, которые будут 
использоваться как параметры при при выполнении запрошенной операции. 
Существует два способа рассматривать эту операцию передачи сообщения. 
Во-первых, передача сообщения соответствует вызову подпрограммы на обычном 
процедурном языке, таком как Паскаль. Это верно в том смысле, что работа 
отправителя останавливается до тех пор, пока получатель не выдаст результат.
\note{\st\ использует синхронные сообщения, поэтому отличия видны слабо, и часто возникают споры, чем сообщение отличается от вызова метода;
в этом смысле \st\ является вырожденным случаем, а наиболее явно это отличие видно в акторной модели \ref{actor}:
\emph{посылка сообщения не передает управление получателю}}
Затем результат возвращается отправителю, который продолжает выполнение 
с точки вызова. Однако сообщения могут создаваться динамически во время 
выполнения, и отношения между отправителем и получателем сообщения, как 
правило, гораздо более свободные, чем статические отношения между 
вызывающим и вызываемым в обычном языке программирования.

В реальном мире каждый объект индивидуален; однако каждый из них обладает 
общими характеристиками с другими подобными объектами. Например, в мешке 
яблок каждое яблоко отличается от всех других. Все же определенные заявления 
могут быть сделаны относительно всех яблок; например, все они будут 
пахнуть одинаково и иметь определенный вкус, все они могут быть использованы 
для выпечки пирогов одинаковым образом, и так далее. Этот процесс называется 
\term{классификацией}. То есть мы можем рассматривать яблоко как отдельный элемент 
или как \term{экземпляр} определенного \term{класса} (или категории) объектов. давайте 
обозначим класс всех яблок через \class{Apple}, заглавная буква и шрифт обозначает 
тот факт, что мы говорим о классе, а не об отдельном объекте.

Экземпляры класса \class{Orange} во многом отличаются от яблок и поэтому заслуживают 
отдельной категории. Но они также имеют много общих характеристик с яблоками. 
Таким образом, мы можем создать новый класс \class{Fruit}, который будет использоваться, 
когда мы хотим описать характеристики, общие для яблок и апельсинов. Класс 
\class{Fruit} включает в себя классы \class{Apple} и \class{Orange}. Таким образом, мы говорим, 
что \class{Fruit}\ --- это \term{суперкласс} \class{Apple} и \class{Orange}, а \class{Apple} и \class{Orange}, в свою 
очередь, являются \term{подклассами} \class{Fruit}.

Наконец, мы можем сделать еще один шаг этого анализа, сделав \class{Fruit} подклассом 
более универсальной категории, которую мы можем назвать \class{Object}. Таким образом, 
у нас есть иерархия категорий для объектов, расширяющаяся от базового класса 
\class{Object}, членом которого является все, вплоть до все более и более конкретных 
классов, пока мы не достигнем самого отдельного объекта.

Такая же ситуация имеет место в отношении всех сущностей в \st. То есть 
каждый объект является членом некоторого класса. За исключением класса \class{Object}, 
этот класс, в свою очередь, будет подклассом некоторого более крупного класса, 
который, в свою очередь, может быть частью другого класса, вплоть до одного 
класса \class{Object}, членом которого является каждый объект. Существует естественная 
древовидная структура (рисунок 1.2), которая иллюстрирует эту иерархию классов. 
Как мы уже делали, мы будем обозначать имена классов, используя первую заглавную 
букву, и обозначая имена объектов без использования заглавных букв. Так, 
например, число 7 является экземпляром класса \class{Integer}, как и число 8. Хотя 
7 и 8 являются различными объектами, они имеют некоторые общие характеристики 
в силу того, что они являются экземплярами одного и того же класса. Например, 
7 и 8 ответят на сообщение «+» целочисленным аргументом, выполнив сложение 
целых чисел. \class{Integer}\ --- это подкласс большего класса \class{Number}. Существуют и 
другие подклассы \class{Number}, например, \class{Float}, значения которых, например, 
3.1415926, являются его экземплярами. \class{Number} --- это подкласс \class{Magnitude} (класс, 
который будет обсуждаться позже), который, наконец, является подклассом \class{Object}.

\fig{language/fig_1_2.png}{height=.6\textheight}

\term{Поведение} объекта в ответ на конкретное сообщение определяется классом этого объекта. 
Например, 7 и 8 будут отвечать на сообщение «+» одинаково, потому что они оба 
являются экземплярами класса \class{Integer}. Список операторов, которые определяют, 
как экземпляр некоторого класса будет реагировать на сообщение, называется \term{методом}
для этого сообщения. Например, в классе \class{Integer} есть метод, связанный с 
сообщением «+». Весь набор сообщений, связанных с классом, называется \term{протоколом}
для этого класса. Класс \class{Integer} содержит в своем протоколе, например, сообщения 
для +, -, * и так далее. В \st\ протокол предоставляется как часть определения 
класса. Синтаксис определений классов будет описан в следующем разделе. 
Невозможно предоставить метод для отдельного объекта; каждый объект должен 
быть связан с некоторым классом, и поведение объекта в ответ на сообщения будет 
продиктовано методами, связанными с этим классом.

Если объект является экземпляром определенного класса, ясно, как будут использоваться 
методы, связанные с этим классом, но как насчет методов, связанных с суперклассами? 
Ответ в том, что любой метод, связанный с суперклассом, \term{наследуется} классом. 
Пример поможет прояснить эту концепцию. При отправке на номер сообщение \var{exp}
означает «вернуть значение $e$ (приблизительно 2,71828..), в степени указанного 
значения». Таким образом, \verb|2 exp| дает $e^2$, или приблизительно 7,38906. Теперь 
описание класса для \class{Integer} не предоставляет метод для сообщения \var{exp}, поэтому, 
когда система Little Smalltalk пытается найти связанный метод для сообщения 
\var{exp} в протоколе класса \class{Integer}, она не находит его. Таким образом, система 
\st\ затем анализирует протокол, связанный с непосредственным 
суперклассом \class{Integer}, а именно \class{Number}. Там, в протоколе для \class{Number}, она находит 
метод и выполняет его. Таким образом, мы говорим, что метод для \var{exp} \term{наследуется}
классом \class{Integer} из класса \class{Number}.
В \class{Number} метод, связанный с сообщением \var{exp}, выглядит следующим образом:
\begin{lstlisting}
    ^ self asFloat exp
\end{lstlisting}

Мы объясним синтаксис более подробно позже; на данный момент мы можем перевести 
этот код как «создать экземпляр \class{Float} с вашим значением (\verb|self asFloat|) и 
отправить этому объекту сообщение \var{exp}, запрашивающее $e$, возведенное в степень 
его значения. Возвратить\note{стрелка вверх $\wedge$ или $\uparrow$ указывает возвращаемое значение}
ответ на это сообщение". Таким образом, сообщение \var{asFloat} передается исходному 
целому числу, скажем, 2. Выполняется метод, связанный с этим сообщением, в 
результате чего получается значение с плавающей запятой 2.0. Сообщение \var{exp} 
затем передается этому значению. Это то же сообщение, которое первоначально 
было передано в целое число 2, только теперь класс получателя\ --- \class{Float}, 
а не \class{Integer}.

На рисунке 1.3 показана иерархия, представляющая несколько классов, включая числа. 
Как мы уже видели, метод для сообщения exp определен в классах \class{Number} и \class{Float}. 
Поиск метода начинается с класса объекта, а затем, при необходимости, проходит 
через различные суперклассы (по \term{цепочке наследования}). Если в сообщении \var{exp} 
дано значение с плавающей запятой, будет выполняться метод в классе \class{Float}, 
а не метод в классе \class{Number}. Таким образом, говорят, что метод для \var{exp} в \class{Float} 
\term{переопределяет} метод в классе \class{Number}.

\fig{language/fig_1_3.png}{height=.6\textheight}

Такие классы, как \class{Number} и \class{Magnitude}, которые обычно не имеют явных экземпляров, 
называются \term{абстрактными классами}. Абстрактные суперклассы важны для 
обеспечения того, чтобы экземпляры различных классов, такие как целые числа 
и числа с плавающей точкой, отвечали аналогичным образом в обычных ситуациях. 
Кроме того, устраняя необходимость дублировать методы для сообщений в 
суперклассе, они уменьшают размер описаний, необходимых для получения 
желаемого поведения.


\secrel{История, Фоновое чтение}

Концепции, относящиеся к объектно-ориентированному программированию, 
заложенные в \st, являются результатом длительного процесса 
разработки и эволюции языка. Основные понятия об объектах, сообщениях 
и классах пришли из языка Simula (Birtwistle 73). Хотя Simula 
позволяла пользователям создавать объектно-ориентированные системы, 
и классы, ответ на сообщение (эквивалент метода в Simula) все еще 
выражался в стандартном для ALGOL методе, ориентированном 
на данные и процедуры.

В семействе языков ALGOL концепция классов привела к разработке понятия 
модулей и абстрактных типов данных (Shaw 80), поддержка которых была 
фундаментальной целью в нескольких языках, таких как Euclid, CLU, Modula и Ada.

В то время как объектно-ориентированная философия постепенно получала признание 
в мире языков программирования, подобные идеи получили признание в сообществе 
архитекторов (Pinnow 82). Точно так же в дизайне операционных систем понятие 
независимых вычислений, которые взаимодействуют друг с другом исключительно 
путем обмена сообщениями, находило сторонников (Wulf 74), (Almes 85). Такое 
представление является естественным и удобным, когда вычисления могут 
физически выполняться на распределенных процессорах.

Прямые предки Smalltalk включают систему Flex (Kay 69), Smalltalk-72 (Goldberg 76) 
и Smalltalk-76 (Ingalls 78). Все языки Smalltalk были созданы в рамках 
проекта Dynabook, инициированного Аланом Кейем в Группе исследований 
обучения в Исследовательском центре Xerox в Пало-Альто. Эволюция языка, 
как показано в этих документах, показывает, что объектно-ориентированная 
модель постепенно расширяется и включает в себя все больше и больше языковых 
концепций. Например, в Smalltalk-72 числа и управляющие структуры обрабатываются 
как объекты, в отличие от Simula, но классы по-прежнему представляют собой 
особую форму. В Smalltalk-76 описания классов представлены в виде объектов, 
а объектно-ориентированная точка зрения расширена до интерфейса программирования. 
Этот интерфейс почти полностью описывается в объектно-ориентиро\-ван\-ной форме 
в среде программирования Smalltalk-80 (Goldberg, 83).

Объектно-ориентированный взгляд на программирование также повлиял на 
другие компьютерные языки, в частности, на понятия актеров (Hewit 73) и 
разновидностей (Weinreb 80) в Лиспе, а также на разработку языков для 
анимации и графики (Рейнольдс 82). Развитие акторов в Лиспе шло параллельно 
с развитием Smalltalk, и два языка влияли друг на друга.

Ковед и ЛаЛонд представляют обзоры, описывающие объектно-ориентированную 
точку зрения в различных ипостасях (Ковед 84) (ЛаЛонд 84). Ряд статей, 
описывающих различные аспекты системы Smalltalk-80, были включены в 
специальный выпуск журнала Byte (Байт 81).


\secly{Упражнения}

\begin{enumerate}

\item Определите следующие термины:

\noindent\begin{tabular}{l l l}
объект&
сообщение&
получатель\\
метод&
протокол&
класс\\
подкласс&
суперкласс&
наследование\\
переопределение&
абстрактный суперкласс\\
\end{tabular}

\item Приведите пример иерархии из повседневной жизни. Перечислите свойства, 
которые можно найти на каждом уровне, и выделите те, которые находятся на 
более низких уровнях, но не на более высоких уровнях.

\item Прочитайте о механизме классов в Simula (DaW 72) (BirtwistIe 73). 
Сравните и сопоставьте это с механизмом классов \st

\item В реальном мире объекты часто классифицируются ортогональными способами, 
а не в древовидной иерархии Smalltalk. Например, белоголовый орлан и кондор 
являются хищными птицами, но одна\ --- это североамериканская птица, а 
другая\ --- южноамериканская птица. Робин также североамериканская птица, 
но не хищная. Эти две отличительные характеристики являются ортогональными 
в том смысле, что ни одна из них не может быть логически названа надмножеством 
другой. Таким образом. навязывание классификации в древовидную структуру 
является неестественным, неэффективным или и тем, и другим. \\
Как можно классифицировать объекты Smalltalk ортогональными способами? Какие 
проблемы это создает для механизма наследования? Как можно преодолеть эти проблемы?

\end{enumerate}

\secup


\secrel{Syntax}\secdown

В этой главе представлен синтаксис для литеральных объектов (таких как числа) 
и синтаксис для сообщений. В нем объясняется, как использовать систему Little Smalltalk 
для вычисления выражений, набираемых непосредственно с клавиатуры, и как использовать 
несколько простых сообщений для получения информации о различных типах объектов.

\bigskip

В этой главе будет описано, как объекты представляются и 
управляются в Little Smalltalk. Как мы отмечали в главе 1, 
все в \st\ --- это объект. Обсуждение синтаксиса начинается 
с описания того, как объекты представлены.

\secrel{Литеральные константы}

Некоторые объекты, \term{литералы}, отличаются тем, что их имя однозначно идентифицирует 
класс и значение объекта, независимо от контекста, и тем фактом, что их не нужно 
объявлять перед использованием. Например, символ 7, независимо от того, где он 
появляется, всегда обозначает один и тот же объект. В Algol-подобных языках 
такой символ, как 7, обычно обозначает «значение», а не идентификатор. В
\st\ это различие гораздо менее отчетливо. Все объекты, включая числа, являются 
объектами, а объекты характеризуются сообщениями, которые они принимают, и их 
ответами на них. Таким образом, 7 обозначает объект так же, как идентификатор, 
такой как \var{x} (в надлежащем контексте), может обозначать объект.

Числа, пожалуй, самые распространенные литеральные объекты. Существует два класса 
чисел, которые могут быть записаны как литеральные объекты, а именно целые числа и 
значения с плавающей точкой. Числа отвечают на различные арифметические сообщения 
(унаследованные от класса \class{Number}) и сообщения отношений (унаследованные от класса 
\class{Magnitude}). Экземпляр класса \class{Integer} состоит из необязательного знака, за которым 
следует любое количество цифр. Число с плавающей запятой состоит из целого числа, 
за которым следует точка (десятичная точка) и другого целого числа без знака 
(дробная часть) и/или буквы \verb|е| и целого числа со знаком (экспоненциальная часть). 
Любому числу может предшествовать основание системы счисления, которое представляет собой 
положительное целое число, за которым следует буква \verb|r|. Для оснований больше 10 
буквы от A до Z интерпретируются как цифры. Примеры чисел:

\begin{lstlisting}
7
16rFF
-3.1415926
2e32
2.4e-32
15rC.ABC
\end{lstlisting}
    
\noindent
Основание системы счисления в основном используется просто ради удобства и внешнего вида. 
Число 16rFF совпадает с числом 10r255 или просто 255.
    
Класс \class{Char} предоставляет возможности для работы со значениями букв. Буквы отличаются 
от цифр. Поскольку символы имеют порядок, заданный последовательностью сортировки, 
их можно сравнивать и, следовательно, они являются подклассом класса \class{Magnitude}. 
Символ написан в виде знака доллара, за которым следует буква (или цифра). 
Ниже приведены примеры экземпляров этого класса:
    
\begin{lstlisting}
$A
$7
$
$$
\end{lstlisting}
    
Экземпляр класса \class{String} представлен последовательностью букв между одинарными кавычками. 
Встраивание кавычки в строку требует двух соседних кавычек. Строка похожа на массив; 
фактически класс \class{String} является подклассом \class{ArrayedCollection}, как и класс \class{Array}. 
И строки, и массивы могут быть объединены вместе для формирования больших строк 
с помощью оператора \term{конкатенации} строк\ --- запятой (,). Примеры строк:
    
\begin{lstlisting}
'a String'
'a String with an '' embedded quote mark'
\end{lstlisting}
    
Массив \class{Array} записывается в виде знака фунта (\verb|#|), за которым следует список элементов 
массива в скобках. Элементами массива являются литеральные объекты (числа или символы), 
строки или другие массивы. В списке массивов ведущий знак фунта на символах 
и массивах может быть исключен. Примеры массивов:

\begin{lstlisting}
#(this is an array of symbols)
#(12 'abc' (another array))
\end{lstlisting}

\noindent
Массивы и строки используют сообщения \verb|at:| и \verb|at:put:| 
для выбора и изменения определенных элементов в их коллекциях.

Класс \class{Symbol}\ --- это еще один литеральный класс. Символ записывается в виде 
знака фунта (\verb|#|), за которым следует любая последовательность букв. 
Пробелы между символами не допускаются. В отличие от строки (которая также 
является последовательностью букв) символ не может быть разбит на более 
мелкие части. Кроме того, одна и та же последовательность букв, используемая 
в разных местах, всегда будет обозначать один и тот же объект. В отличие от 
чисел, символов или букв, символы не имеют порядка и не могут сравниваться 
(за исключением, конечно, равенства объектов). Примеры символов:

\begin{lstlisting}
#aSymbol
#AndAnother
#+++
#very.long.symbol.with.periods
\end{lstlisting}

\secrel{Identifiers}

Identifiers in Little Smalltalk can be divided into three categories: instance
variables, class names, and pseudo-variables. An identifier beginning with
a capital letter is always a class name, whereas an identifier beginning
with a lowercase letter must represent either a pseudo variable or an
instance variable.

At the command level, new instance variables can be defined merely
by assigning a value to a name. The assignment arrow is formed as a twocharacter sequence consisting of a less than sign and a minus sign
\note{From now on the text will use the symbol $\leftarrow$ to represent this two-character sequence}:
\begin{lstlisting}
newname <- 17
\end{lstlisting}

Instance variables defined at the command level are known only at the
command level and cannot be used within a method for any class. As we
will see in a later chapter, instance variables within a class must all be
declared.

Class identifiers respond to a variety of messages that can be used to
discover information concerning the class the object represents. For example, the message respondTo, when passed to an object representing a
class, will cause the class to print a list of the messages to which instances
of the class will respond.

Pseudo variables look like normal identifiers (that is, they are named
by a sequence of letters beginning with a lower case letter), but unlike
identifiers they need not be declared. There are several pseudo variables:
self, super, selfProcess
\note{The pseudo-variables selfprocess and smalltalk are unique to Little Smalltalk and are
not part of the \st-80 system, where different techniques are used to obtain the currently executing process or to obtain information about the current environment. See Appendix 5 for an overView of the differences between Little Smalltalk and the Smalltalk-80
programming environment.}
, true, false, nil, and smalltalk. Arguments for
a method (to be discussed shortly) are also considered to be. pseudo-variables. Of the seven, self, super, and selfProcess are farthest from being
literal objects because their meaning depends entirely upon context. We
will discuss these in more detail when we describe class methods and
processes. The next three, true, false, and nil, are defined to be instances
(usually the only instances) of the classes True, False, and UndefinedObject, respectively. We will discuss these three in more detail when
we outline the behavior of different classes. The final pseudo variable,
smalltalk, is an instance of class Smalltalk and is used to centralize several
pieces of information concerning the currently executing environment.

Other types of objects in the Little Smalltalk system, such as blocks
and instances of user defined classes, will be discussed in later sections.

\secrel{Messages}

As noted in Chapter I, all actions in Smalltalk are produced by sending
messages to objects. This section begins by describing the syntax used to
produce messages.

Any message can be divided into three parts; a receiver, a message
selector, and zero or more arguments. The receiver and argument portions
of a message can be specified by other message expressions, or they may
be specified by a single token, such as an identifier or a literal.

The first type of message selector requires no arguments and is called
a unary message. A unary message selector consists of an identifier, the
first letter of which must be lowercase. For example:
\begin{lstlisting}
7 sign
\end{lstlisting}
illustrates the message sign being passed to the number 7. Unary messages,
like all messages, elicit a response, which is simply another object. The
response to sign is an integer, either -1,0, or 1, depending upon the sign
of the object the message was sent to (the receiver). Unary messages parse
left to right, so, for example:
\begin{lstlisting}
7 factorial sqrt
\end{lstlisting}
returns $\sqrt{7!}$, or approximately 70.993.    

The second form of message, called a binary message, takes one argument. 
A binary message is formed from one or two adjacent nonalphabetic characters\note{Some 
characters, such as braces, parenthesis or periods, cannot be used to form
binary messages. See the description in Appendix 2 for a more complete description of the
restrictions.}.
Binary messages tend to be used for arithmetic
operations, although this is not enforced by the system and there are
notable exceptions. An example of a binary message is arithmetic addition:
\begin{lstlisting}
7 + 4
\end{lstlisting}
At first the fact that this is interpreted as "send the message + with
argument 4 to-the object 7"may seem strange; however, soon the uniform
treatment of objects and message passing in Smalltalk makes this seem
natural.

Binary messages, like unary messages, parse left to right. Thus
\begin{lstlisting}
7 + 4 * 3
\end{lstlisting}
results in 33, not 19. \emph{Unary messages have a higher precedence than binary
messages}, thus
\begin{lstlisting}
7 + 17 sqrt
\end{lstlisting}
evaluates as $7 + (\sqrt{17})$, not $\sqrt{(7 + 17)}$.

The most general type of message is a keyword message. The selector
for a keyword message consists of one or more keywords. Each keyword
is followed by an argument. A keyword is simply an identifier (again, the
first character must be lower case) followed by a colon. The argument can
be any expression, although if the expression is formed using a keyword
message, it must be placed in parentheses to avoid ambiguity. Example
keyword expressions are:
\begin{lstlisting}
7 max: 14.
7 between: 2 and: 24
\end{lstlisting}

When we wish to express the name of the message being requested by
a keyword message, we catenate the keyword tokens. Thus we say the
message selector being expressed in the second example above is between:and:. There can be any number of keywords in a keyword message,
although in practice few messages have more than three.

Keyword messages have lower precedence than either binary or unary
messages. Thus
\begin{lstlisting}
7 between: 2 sqrt and: 4 + 2
\end{lstlisting}
\begin{lstlisting}
7 between: (2 sqrt) and: (4 + 2)
\end{lstlisting}
            
\secrel{Getting Started}

You now have enough information to try getting some hands-on experience
using the Little Smalltalk system. After logging on, type the command st.
After a moment, the message "Little Smalltalk" should appear, and the
cursor should be indented by a small amount on the next line. If, at this
point, you type in a Smalltalk expression and hit the return key, the expression will be evaluated and the result printed. Try typing "3 + 4" and see
what happens. The result should be a 7, produced at the left margin. The
cursor then should advance to the next line and once more tab over several
spaces. Try typing "5 + 4 sqrt." Can you explain the outcome? Try "(5 +
4) sqrt."

Try typing
\begin{lstlisting}
i <- 3
\end{lstlisting}
Notice that, since assignment expressions do not have a value, no value
was printed. However, if you now type
\begin{lstlisting}
i
\end{lstlisting}
the most recent object assigned to the name will be produced.

The name last always contains the value of the last expression computed. Try typing
\begin{lstlisting}
27 + 3 sqrt
\end{lstlisting}
followed by
\begin{lstlisting}
last
\end{lstlisting}

\secrel{Finding Out About Objects}

There are various messages that can be used to discover facts about an
object. The message class, for example, will tell you the class of an object.
Try typing
\begin{lstlisting}
7 class
\end{lstlisting}
The message superClass, when passed to an instance of Class, will return
the immediate superclass of that class. Try typing
\begin{lstlisting}
Integer superClass
7 class superClass
\end{lstlisting}
What is the superclass of Object?

The keyword message respondsTo: can be used to discover if an object
will respond to a particular message. The argument must be a symbol,
representing the message. Try typing
\begin{lstlisting}
 7 respondsTo: #+
$A respondsTo: #between:and:
$A respondsTo: #sqrt
\end{lstlisting}
When passed to a ciass, the message respondTo: inquires whetherinstances
of the class respond to the given message. For example,
\begin{lstlisting}
Integer respondsTo: #+
\end{lstlisting}

You can discover if two objects are the same using the binary message
==. The message \verb|~~| is the logical inverse of ==. Try typing
\begin{lstlisting}
i <- 17
i == 17
17~~17
\end{lstlisting}

One way to tell if an object is an instance of a particular class is to
connect the unary message class and the binary message ==. Try typing
\begin{lstlisting}
i class == Integer
\end{lstlisting}

A simple abbreviation for this is the message isMemberOf:. For example, the last expression given is equivalent to
\begin{lstlisting}
i isMemberOf: Integer
\end{lstlisting}

Suppose we want to tell if an object is a Number, but we don't care if
it is any particular kind of number (Integer or Float). We could use the
boolean OR bar (I), which is recognized by the boolean values true and
false:
\begin{lstlisting}
(i isMemberOf: Integer) | (i isMemberOf: Float)
\end{lstlisting}
A simplier method is to use the message isKindOf;. This message asks
whether the class of the object, or any of superclasses, is the same as the
argument. Try typing
\begin{lstlisting}
i isKindOf: Number
\end{lstlisting}

\secrel{Blocks}

An interesting feature of Smalltalk is the ability to encapsulate a sequence
of actions and then to perform those actions at a later time, perhaps even
in a different context. This feature is called a block (an instance of class
Block) and is formed by surrounding a sequence of Smalltalk statements
with square braces, as in:
\begin{lstlisting}
[ i <= i+1. i print ]
\end{lstlisting}

Within a block (and, as we will see in the next chapter, in general
within a method) a period is used as a statement separator. Since a block
is an object, it can be assigned to an identifier or passed as an argument
with a message or used in any other manner in which objects may be used.
In response to the unary message value, a block will execute in the context
in which it was defined, regardless of whether this is the current context
or not. That is, when the block given above is evaluated, the identifier i
will refer to the binding of the identifier i that was known at the time the
block was defined. Even if the block is passed as an argument into a class
in which there is a different instance variable i and then evaluated, the i
in the block will refer to the i in the context in which the block was defined.
Thus a block when used as a parameter is similar to the Algol-60 call-byname notion of a thunk.

The value returned by a block is the value of the last expression inside
that block. Frequently a block will contain a single expression, and the
value resulting from that block will be the value of the expression.

One way to think about blocks is as a type of in-line procedure declaration. Like procedures, a block can also take a number of arguments.
Parameters are denoted by colon-variables at the beginning of the block,
followed by a vertical bar and then the statements composing the block.
For example,
\begin{lstlisting}
[:x :y | (x + y) print]
\end{lstlisting}
is known as a two-parameter block (sometimes two-argument block). The
message value: is used to evaluate a block with parameters the number of
value: keywords given matches the number of arguments in the block. So,
for the example given above, the evaluating message would be value:value:.

\secrel{Comments and Continuations}

A pair of double quote marks (II) are used to enclose a comment. One
must be careful not to confuse the double quote mark with two adjacent
single quote marks CI), which look very similar. The text of the comment
can be arbitrary and is ignored by the Little SmaIItalk system.

The Little Smalltalk system assumes that each line typed at the terminal
is a complete SmaIItalk expression. Should it be necessary to continue a
long expression on two or more lines, a special ineJication must be given
to the Little SmaIItaik system to prevent it from misinterpreting the partial
expression on the first line and generating an unintentional error message.
This special indication is a backwards slash C""-) as the last character on
all intermediate lines, for example:
\begin{lstlisting}
    2 +     \
    3 * 7   \
    + 5     \
40
\end{lstlisting}


\secly{EXERCISES}

\begin{enumerate}

    \item Show the order of evaluation for the subexpressions in the following
    expression:
    \begin{lstlisting}
        7/2 between: 7 + 17 sqrt and: 3 * 5
    \end{lstlisting}

    \item Type the following expressions:
    \begin{lstlisting}
        7 = = 7
        label = = label
        #abe = = #abc
    \end{lstlisting}
    How do you explain this behavior?
    
    \item What values will be printed in place of the question marks in the
    following sequence:
    \begin{lstlisting}
            i <- 17
            j<-[i< +1]
            j print
        ??
            j print
        ??
            i < - 23
            i print
        ??
            j value print
        ??
            i print
        ??
            j value print
        ??
            i print
        ??
    \end{lstlisting}

\end{enumerate}

\secup


\secrel{Basic Classes}\secdown

The basic classes included in the Little Smalltalk standard library are
explained in this chapter.

\secrel{Basic Objects}
\secrel{Collections}
\secrel{Control Structures}
\secrel{Class Management}
\secrel{Abstract Superclasses}

\secup

\secrel{Class Definition}\secdown

This chapter introduces the syntax used for defining classes. An example class definition is presented.

\secrel{An Illustrative Example}
\secrel{Processing a Class Definition}

\secup

\secrel{A Simple Application}\secdown

This chapter illustrates the development of a simple application in
Smalltalk and describes how environments can be saved and restored.

\secrel{Saving Environments}

\secup

\secrel{Primitives, Cascades, and Coercions}\secdown

This chapter introduces the syntax for cascaded expressions and describes the notion of primitive expressions. It illustrates the use of
primitives by showing how primitives are used to produce the correct
results for mixed mode arithmetic operations.

\secrel{Cascades}
\secrel{Primitives}
\secrel{Numbers}

\secup

\secrel{A Simulation}\secdown

This chapter presents a simple simulation of an ice cream store, illustrating the ease with which simulations can be described in Smalltalk.

\secrel{The Ice Cream Store Simulation}
\secrel{Further Reading}

\secup

\secrel{Generators}\secdown

This chapter introduces the concept of generators and shows how
generators can be used in the solution of problems requiring goaldirected evaluation.

\secrel{Filters}
\secrel{Goal-Directed Evaluation}
\secrel{Operatkms on Generators}
\secrel{Further Reading}

\secup

\secrel{Graphics}\secdown

Although graphics are not fundamental to Little Smalltalk in the same
way that they are an intrinsic part of the Smalltalk-80 system, it is still
possible to describe some graphics functions using the language. This
chapter details three types of approaches to graphics.

\secrel{Character Graphics}
\secrel{Line Graphics}
\secrel{Bit-Mapped Graphics}

\secup

\secrel{Processes}\secdown

This chapter introduces the concepts of processes and semaphores. It
illustrates these concepts using the dining philosophers problem.

\secrel{Semaphores}
\secrel{Monitors}
\secrel{The Dining Philosophers Problem }
\secrel{Further Reading}

\secup

\secup
