
\clearpage
\secly{Предисловие}

\subsecly{Система Little Smalltalk: немного истории}

Весной 1984 года я преподавал курс по языкам программирования в университете Аризоны. 
При подготовке лекций для этого курса я заинтересовался концепцией объектно-ориентированного 
программирования и, в частности, тем, как объектно-ориентированная парадигма изменила 
подход программистов к решению проблем. В течение этого семестра и следующего лета я 
собрал как можно больше материалов об объектно-ориентированном программировании, 
особенно о системе программирования \st-80, разработанной в исследовательском 
центре Xerox Palo Alto (Xerox PARC). Тем не менее, я продолжал расстраиваться из-за 
своей неспособности получить практический опыт написания и использования программ на \st.

В то время единственной системой \st, о которой я знал, была оригинальная система, 
работающая на Dorado, дорогой машине, недоступной (в то время) за пределами Xerox PARC. 
Доступной мне возможностью был VAX-780 с Unix2 использовавший обычные ASCII терминалы. 
Таким образом, оказалось, что мои шансы на запуск системы Xerox \st-80 в 
ближайшей перспективе были весьма невелики; поэтому, несколько студентов и я 
решили летом 1984 года создать нашу собственную систему \st.

Осенью 1984 года мы с дюжиной студентов создали систему Little Smalltalk в рамках 
семинара для выпускников по реализации языка программирования. С самого начала наши 
цели были гораздо менее амбициозными, чем у первоначальных разработчиков системы 
\st-80. Несмотря на то, что мы оценили важность инновационных концепций в 
средах программирования и графике, впервые разработанных группой Xerox, мы до боли 
осознавали наши собственные ограничения, как в рабочей силе, так и в оборудовании. 
Нашими целями в порядке важности были:
\begin{itemize}[nosep]
    \item Новая система должна поддерживать язык, максимально приближенный к 
    опубликованному описанию \st-80 \cite{blue}.
    \item Система должна работать под Unix, используя только обычные текстовые терминалы.
    \item Система должна быть написана на \ci\ и быть максимально переносимой.
    \item Система должна быть маленькой. В частности, она должна работать на 
    16-битных машинах с раздельной памятью команд и данных, 
    но предпочтительно даже на машинах без этой функции.
\end{itemize}

Оглядываясь назад, мы, кажется, достигли наших целей довольно хорошо. Язык, понимаемый 
системой Little Smalltalk, достаточно близок к языку системы программирования 
\st-80, так что пользователи, похоже, испытывают только небольшие трудности 
(по крайней мере, с языком) при переходе от одной системы к другой. Система 
оказалась чрезвычайно переносимой: она была перенесена на дюжину разновидностей 
Unix, работающих на разных машинах. Более 200 сайтов теперь используют 
систему Little Smalltalk.

\subsecly{О системе Little Smalltalk}

Эта книга состоит из двух частей. Первый раздел описывает язык системы Little 
Smalltalk. Хотя большинство читателей, возможно, до знакомства с Smalltalk 
имели некоторый опыт использования хотя бы одного другого языка программирования, 
в тексте не делается никаких предположений относительно подготовки читателя. 
Большинство студентов старших курсов или аспирантов должны быть в состоянии 
понять материал в первом разделе. Эта часть текста может использоваться отдельно.

Вторая часть книги описывает фактическую реализацию системы Little Smalltalk. 
Этот раздел требует от читателя более глубоких знаний в области информатики. 
Поскольку Little Smalltalk написан на C, требуется хотя бы элементарное знание 
этого языка. Также желательная хорошая подготовка по структурам данных. 
Читателю будет желательно, хотя и не обязательно, иметь некоторое 
представление о построении компиляторов для обычного языка, такого как Pascal.


\subsecly{Благодарности}

I am, of course, most grateful to the students in the graduate seminar at the
University of Arizona where the Little Smalltalk system was developed. The many
heated discussions and insightful ideas generated were most enjoyable and
stimulating. Participants in that seminar were Mike Benhase, Nick Buchholz, Dave
Burns, John Cabral, Clayton Curtis, Roger Hayes, Tom Hicks, Rob McConeghy,
Kelvin Nilsen, May Lee Noah, Sean O'Malley, and Dennis Vadner. This text grew
out of notes developed for that course, and includes many ideas contributed by
the participants. In particular I wish to thank Dave Burns for the original
versions of the simulation described in Chapter 7 and Mike Benhase and Dennis
Vadner for their work on processes and the dining philosophers solution
presented in Chapter 10.

Discussions with many people have yielded insights or examples that eventually
found their way into this book. I wish to thank, in particular, Jane Cameron,
Chris Fraser, Ralph Griswold, Paul Klint, Gary Levin, and Dave Robson.

Irv Elshoff provided valuable assistance by trying to learn Smalltalk from an
early manuscript and by making many useful and detailed comments on the text.

J. A. Davis from Iowa State University, Paul Klint from the CWI, David Robson
from Xerox Palo Alto Research Center, and. Frances Van Scoy from West Virginia
University provided careful and detailed comments on earlier drafts of the book.

Charlie Allen at Purdue, Jan Gray at Waterloo and Charles Hayden at AT\&T were
early non-Arizona users of Little Smalltalk and were extremely helpful in
finding bugs in the earlier distributions.

I wish to thank Ralph Griswold, Dave Hanson, and Chris Fraser, all chairmen of
the computer science department at the University of Arizona at various times in
the last five years, for helping to make the department such a pleasant place to
work. Finally I wish to thank Paul Vitanyi and Lambert Meertens for providing me
with the chance to work at the Centrum voor Wiskunde en Informatica in Amsterdam
for the year between my time in Arizona and my move to Oregon, and for
permitting me to finish work on the book while there.

\subsecly{Получение оригинальной системы Little Smalltalk}

\url{https://github.com/crcx/littlesmalltalk} Архив Little Smalltalk 
(с обновлениями для работы на современных платформах). 
Он также собирает форки и документацию по этой исторической системе.

\bigskip

The Little Smalltalk system can be obtained directly from the author. The
system is distributed on 9-track tapes in tar format (the standard unix
distribution format). The distribution tape includes all sources and on-line
documentation for the system. For further information on the distribution,
including cost, write to the following address:

\bigskip\noindent
Smalltalk Distribution\\
Department of Computer Science\\
Oregon State University\\
Corvallis, Oregon\\
97331\\
USA



\secrel{The Language}\secdown

\secrel{Основы}\secdown

Эта глава знакомит с основными понятиями языка Smalltalk; а именно объект, метод, класс, наследование и переопределение.

\bigskip

Традиционной моделью, описывающей поведение компьютера, выполняющего программу, 
является модель состояния процесса, или модель «сортировщика». В этом 
представлении компьютер является диспетчером данных, который следует некоторому 
блоку инструкций, блуждает по памяти, извлекает значения из различных слотов 
(адресов памяти), преобразует их каким-либо образом, и передает результаты 
обратно в другие слоты. Изучая значения в слотах, можно определить состояние 
машины или результаты, полученные вычислением. Хотя это может быть более или 
менее точная картина того, что происходит в компьютере, это мало помогает нам 
понять, как решать проблемы с помощью компьютера, и это, конечно, не те способы, 
которыми думают большинство людей решающих проблемы (за исключением сортировщиков и почтальонов).

Давайте рассмотрим реалистичную ситуацию, а затем посмотрим, как можно заставить 
компьютер более точно моделировать методы, которые люди используют для решения 
проблем в повседневной жизни. Предположим, я хочу послать цветы моей бабушке 
на день рождения. Она живет далеко в городе за много миль от меня. Задача 
достаточно проста для выполнения; Я просто иду к местному флористу, описываю 
виды и количество цветов, которые я хочу отправить, и я могу быть уверен, 
что они будут доставлены автоматически. Если я проведу расследование, я, 
вероятно, обнаружу, что мой флорист отправляет сообщение с описанием моего 
заказа другому флористу в городе моей бабушки. Тот флорист тогда составляет 
букет и доставляет цветы. Я мог бы спросить, чтобы узнать, как цветочный 
магазин в городе моей бабушки покупает цветы и, возможно, узнал, что они 
получены от оптового торговца цветами. Если я продолжу настаивать, я даже 
смогу проследить всю цепочку до фермера, который выращивает цветы, и узнать, 
какие запросы были сделаны членами цепочки, чтобы получить желаемый результат от каждого.

Важным моментом, однако, является то, что мне не нужно, да и вообще, 
я не хочу знать, как будет выполняться моя простая директива 
«отправить цветы моей бабушке». В реальной жизни мы называем этот 
процесс \term{делегирование полномочий}. В информатике это называется 
\term{абстракция} или \term{сокрытие информации}. В основе этих терминов лежит 
одно и то же. Есть ресурс (флорист, файловый сервер), который я хочу 
использовать. Чтобы общаться, я должен знать команды, на которые будет 
реагировать ресурс (отправить цветы моей бабушке, вернуть копию файла 
с именем "chapеer!"). Скорее всего, шаги, которые должен предпринять 
ресурс, чтобы ответить на мой запрос, гораздо более сложны, чем я 
понимаю, но мне нет смысла знать подробности того, как реализуется 
моя директива, до тех пор, пока ответ (доставка цветов, получение 
копии моего файла) четко определено и предсказуемо.

\term{Объектно-ориентированная} модель решения проблем рассматривает компьютер
способом, очень близким к такому подходу. Действительно, многие люди, 
которые не имеют никакого образования в области информатики, и не знают, 
как работает компьютер, находят объектно-ориенти\-рован\-ную модель решения 
проблем вполне естественной. Удивительно, однако, что многие люди, 
имеющие традиционный опыт программирования, изначально думают, что в 
концепции объекта есть что-то странное. Представление о том, что число «7» 
является объектом, а «+» --- это запрос на сложение, может поначалу 
показаться странным. Но вскоре единообразие, мощь и гибкость, которые 
метафора объект/сообщение привносит в решение проблемы, делает эту 
интерпретацию так же естественной.

Вселенная \st\ населена \term{объектами}. В моем примере с цветами я являюсь объектом, 
а цветочный магазин (или флорист в нем)\ --- другим объектом. Действия 
инициируются путем отправки \term{сообщений}\ (запросов) между объектами. Я передал 
просьбу «отправить цветы моей бабушке» флористу-объекту. Реакция \term{получателя}
моего сообщения состоит в том, чтобы выполнить некоторую последовательность 
действий или \term{метод}, чтобы удовлетворить мой запрос. Может быть, получатель 
сможет немедленно удовлетворить мой запрос. С другой стороны, чтобы удовлетворить 
мои потребности, получателю, возможно, придется передавать другие сообщения 
еще большему количеству объектов (например, сообщение, которое мой флорист 
отправляет флористу в городе моей бабушки, или команду на дисковод). Кроме 
того, существует явный ответ (например, квитанция или код результата), 
возвращенный непосредственно мне. Дэн Ингаллс описывает философию \st\ (Байт 81):

\begin{quote}
    Вместо того, чтобы обрабатывать структуры данных, которые насилуют и грабят 
    процессор, мы имеем вселенную объектов с хорошим поведением, которые вежливо 
    просят друг друга выполнить их различные желания.
\end{quote}
    
Такие антропоморфные точки зрения распространены среди \st-программистов. 
В последующих главах мы увидим, как язык Smalltalk воплощает этот 
объектно-ориентированный взгляд на программирование. Описывая решение 
нескольких проблем в \st, мы надеемся показать, как объектно-ориентированная 
модель помогает в создании больших программных систем, и помогает в решении 
многих проблем с использованием компьютера.

\secrel{Объекты, классы и наследование}

В Smalltalk все является объектом. В языке нет способа создать сущность, 
которая не является объектом. Среди основных компьютерных языков это 
единообразие в Smalltalk конкурирует, возможно, только с LISP, и, 
как и в LISP, единообразие создает и простоту, и мощь языка.

Объект обладает несколькими характеристиками (рисунок 1.1). Каждый объект 
содержит небольшой объем памяти, доступный только этому объекту. То есть 
ни один объект не может читать или изменять значения памяти в другом 
объекте. Конечно, поскольку всё в системе должно быть объектом, память 
объекта может содержать только ссылки на другие объекты. Мы обсудим 
это более подробно позже.

\fig{language/fig_1_1.png}{height=\textheight}

Все действия в системе \st\ производятся путем передачи сообщений. 
Сообщение\ --- это запрос у объекта выполнения какой-либо операции. Оно 
может содержать некоторые значения в качестве аргументов, которые будут 
использоваться как параметры при при выполнении запрошенной операции. 
Существует два способа рассматривать эту операцию передачи сообщения. 
Во-первых, передача сообщения соответствует вызову подпрограммы на обычном 
процедурном языке, таком как Паскаль. Это верно в том смысле, что работа 
отправителя останавливается до тех пор, пока получатель не выдаст результат.
\note{\st\ использует синхронные сообщения, поэтому отличия видны слабо, и часто возникают споры, чем сообщение отличается от вызова метода;
в этом смысле \st\ является вырожденным случаем, а наиболее явно это отличие видно в акторной модели \ref{actor}:
\emph{посылка сообщения не передает управление получателю}}
Затем результат возвращается отправителю, который продолжает выполнение 
с точки вызова. Однако сообщения могут создаваться динамически во время 
выполнения, и отношения между отправителем и получателем сообщения, как 
правило, гораздо более свободные, чем статические отношения между 
вызывающим и вызываемым в обычном языке программирования.

В реальном мире каждый объект индивидуален; однако каждый из них обладает 
общими характеристиками с другими подобными объектами. Например, в мешке 
яблок каждое яблоко отличается от всех других. Все же определенные заявления 
могут быть сделаны относительно всех яблок; например, все они будут 
пахнуть одинаково и иметь определенный вкус, все они могут быть использованы 
для выпечки пирогов одинаковым образом, и так далее. Этот процесс называется 
\term{классификацией}. То есть мы можем рассматривать яблоко как отдельный элемент 
или как \term{экземпляр} определенного \term{класса} (или категории) объектов. давайте 
обозначим класс всех яблок через \class{Apple}, заглавная буква и шрифт обозначает 
тот факт, что мы говорим о классе, а не об отдельном объекте.

Экземпляры класса \class{Orange} во многом отличаются от яблок и поэтому заслуживают 
отдельной категории. Но они также имеют много общих характеристик с яблоками. 
Таким образом, мы можем создать новый класс \class{Fruit}, который будет использоваться, 
когда мы хотим описать характеристики, общие для яблок и апельсинов. Класс 
\class{Fruit} включает в себя классы \class{Apple} и \class{Orange}. Таким образом, мы говорим, 
что \class{Fruit}\ --- это \term{суперкласс} \class{Apple} и \class{Orange}, а \class{Apple} и \class{Orange}, в свою 
очередь, являются \term{подклассами} \class{Fruit}.

Наконец, мы можем сделать еще один шаг этого анализа, сделав \class{Fruit} подклассом 
более универсальной категории, которую мы можем назвать \class{Object}. Таким образом, 
у нас есть иерархия категорий для объектов, расширяющаяся от базового класса 
\class{Object}, членом которого является все, вплоть до все более и более конкретных 
классов, пока мы не достигнем самого отдельного объекта.

Такая же ситуация имеет место в отношении всех сущностей в \st. То есть 
каждый объект является членом некоторого класса. За исключением класса \class{Object}, 
этот класс, в свою очередь, будет подклассом некоторого более крупного класса, 
который, в свою очередь, может быть частью другого класса, вплоть до одного 
класса \class{Object}, членом которого является каждый объект. Существует естественная 
древовидная структура (рисунок 1.2), которая иллюстрирует эту иерархию классов. 
Как мы уже делали, мы будем обозначать имена классов, используя первую заглавную 
букву, и обозначая имена объектов без использования заглавных букв. Так, 
например, число 7 является экземпляром класса \class{Integer}, как и число 8. Хотя 
7 и 8 являются различными объектами, они имеют некоторые общие характеристики 
в силу того, что они являются экземплярами одного и того же класса. Например, 
7 и 8 ответят на сообщение «+» целочисленным аргументом, выполнив сложение 
целых чисел. \class{Integer}\ --- это подкласс большего класса \class{Number}. Существуют и 
другие подклассы \class{Number}, например, \class{Float}, значения которых, например, 
3.1415926, являются его экземплярами. \class{Number} --- это подкласс \class{Magnitude} (класс, 
который будет обсуждаться позже), который, наконец, является подклассом \class{Object}.

\fig{language/fig_1_2.png}{height=.6\textheight}

\term{Поведение} объекта в ответ на конкретное сообщение определяется классом этого объекта. 
Например, 7 и 8 будут отвечать на сообщение «+» одинаково, потому что они оба 
являются экземплярами класса \class{Integer}. Список операторов, которые определяют, 
как экземпляр некоторого класса будет реагировать на сообщение, называется \term{методом}
для этого сообщения. Например, в классе \class{Integer} есть метод, связанный с 
сообщением «+». Весь набор сообщений, связанных с классом, называется \term{протоколом}
для этого класса. Класс \class{Integer} содержит в своем протоколе, например, сообщения 
для +, -, * и так далее. В \st\ протокол предоставляется как часть определения 
класса. Синтаксис определений классов будет описан в следующем разделе. 
Невозможно предоставить метод для отдельного объекта; каждый объект должен 
быть связан с некоторым классом, и поведение объекта в ответ на сообщения будет 
продиктовано методами, связанными с этим классом.

Если объект является экземпляром определенного класса, ясно, как будут использоваться 
методы, связанные с этим классом, но как насчет методов, связанных с суперклассами? 
Ответ в том, что любой метод, связанный с суперклассом, \term{наследуется} классом. 
Пример поможет прояснить эту концепцию. При отправке на номер сообщение \var{exp}
означает «вернуть значение $e$ (приблизительно 2,71828..), в степени указанного 
значения». Таким образом, \verb|2 exp| дает $e^2$, или приблизительно 7,38906. Теперь 
описание класса для \class{Integer} не предоставляет метод для сообщения \var{exp}, поэтому, 
когда система Little Smalltalk пытается найти связанный метод для сообщения 
\var{exp} в протоколе класса \class{Integer}, она не находит его. Таким образом, система 
\st\ затем анализирует протокол, связанный с непосредственным 
суперклассом \class{Integer}, а именно \class{Number}. Там, в протоколе для \class{Number}, она находит 
метод и выполняет его. Таким образом, мы говорим, что метод для \var{exp} \term{наследуется}
классом \class{Integer} из класса \class{Number}.
В \class{Number} метод, связанный с сообщением \var{exp}, выглядит следующим образом:
\begin{lstlisting}
    ^ self asFloat exp
\end{lstlisting}

Мы объясним синтаксис более подробно позже; на данный момент мы можем перевести 
этот код как «создать экземпляр \class{Float} с вашим значением (\verb|self asFloat|) и 
отправить этому объекту сообщение \var{exp}, запрашивающее $e$, возведенное в степень 
его значения. Возвратить\note{стрелка вверх $\wedge$ или $\uparrow$ указывает возвращаемое значение}
ответ на это сообщение". Таким образом, сообщение \var{asFloat} передается исходному 
целому числу, скажем, 2. Выполняется метод, связанный с этим сообщением, в 
результате чего получается значение с плавающей запятой 2.0. Сообщение \var{exp} 
затем передается этому значению. Это то же сообщение, которое первоначально 
было передано в целое число 2, только теперь класс получателя\ --- \class{Float}, 
а не \class{Integer}.

На рисунке 1.3 показана иерархия, представляющая несколько классов, включая числа. 
Как мы уже видели, метод для сообщения exp определен в классах \class{Number} и \class{Float}. 
Поиск метода начинается с класса объекта, а затем, при необходимости, проходит 
через различные суперклассы (по \term{цепочке наследования}). Если в сообщении \var{exp} 
дано значение с плавающей запятой, будет выполняться метод в классе \class{Float}, 
а не метод в классе \class{Number}. Таким образом, говорят, что метод для \var{exp} в \class{Float} 
\term{переопределяет} метод в классе \class{Number}.

\fig{language/fig_1_3.png}{height=.6\textheight}

Такие классы, как \class{Number} и \class{Magnitude}, которые обычно не имеют явных экземпляров, 
называются \term{абстрактными классами}. Абстрактные суперклассы важны для 
обеспечения того, чтобы экземпляры различных классов, такие как целые числа 
и числа с плавающей точкой, отвечали аналогичным образом в обычных ситуациях. 
Кроме того, устраняя необходимость дублировать методы для сообщений в 
суперклассе, они уменьшают размер описаний, необходимых для получения 
желаемого поведения.


\secrel{История, Фоновое чтение}

Концепции, относящиеся к объектно-ориентированному программированию, 
заложенные в \st, являются результатом длительного процесса 
разработки и эволюции языка. Основные понятия об объектах, сообщениях 
и классах пришли из языка Simula (Birtwistle 73). Хотя Simula 
позволяла пользователям создавать объектно-ориентированные системы, 
и классы, ответ на сообщение (эквивалент метода в Simula) все еще 
выражался в стандартном для ALGOL методе, ориентированном 
на данные и процедуры.

В семействе языков ALGOL концепция классов привела к разработке понятия 
модулей и абстрактных типов данных (Shaw 80), поддержка которых была 
фундаментальной целью в нескольких языках, таких как Euclid, CLU, Modula и Ada.

В то время как объектно-ориентированная философия постепенно получала признание 
в мире языков программирования, подобные идеи получили признание в сообществе 
архитекторов (Pinnow 82). Точно так же в дизайне операционных систем понятие 
независимых вычислений, которые взаимодействуют друг с другом исключительно 
путем обмена сообщениями, находило сторонников (Wulf 74), (Almes 85). Такое 
представление является естественным и удобным, когда вычисления могут 
физически выполняться на распределенных процессорах.

Прямые предки Smalltalk включают систему Flex (Kay 69), Smalltalk-72 (Goldberg 76) 
и Smalltalk-76 (Ingalls 78). Все языки Smalltalk были созданы в рамках 
проекта Dynabook, инициированного Аланом Кейем в Группе исследований 
обучения в Исследовательском центре Xerox в Пало-Альто. Эволюция языка, 
как показано в этих документах, показывает, что объектно-ориентированная 
модель постепенно расширяется и включает в себя все больше и больше языковых 
концепций. Например, в Smalltalk-72 числа и управляющие структуры обрабатываются 
как объекты, в отличие от Simula, но классы по-прежнему представляют собой 
особую форму. В Smalltalk-76 описания классов представлены в виде объектов, 
а объектно-ориентированная точка зрения расширена до интерфейса программирования. 
Этот интерфейс почти полностью описывается в объектно-ориентиро\-ван\-ной форме 
в среде программирования Smalltalk-80 (Goldberg, 83).

Объектно-ориентированный взгляд на программирование также повлиял на 
другие компьютерные языки, в частности, на понятия актеров (Hewit 73) и 
разновидностей (Weinreb 80) в Лиспе, а также на разработку языков для 
анимации и графики (Рейнольдс 82). Развитие акторов в Лиспе шло параллельно 
с развитием Smalltalk, и два языка влияли друг на друга.

Ковед и ЛаЛонд представляют обзоры, описывающие объектно-ориентированную 
точку зрения в различных ипостасях (Ковед 84) (ЛаЛонд 84). Ряд статей, 
описывающих различные аспекты системы Smalltalk-80, были включены в 
специальный выпуск журнала Byte (Байт 81).


\secly{Упражнения}

\begin{enumerate}

\item Определите следующие термины:

\noindent\begin{tabular}{l l l}
объект&
сообщение&
получатель\\
метод&
протокол&
класс\\
подкласс&
суперкласс&
наследование\\
переопределение&
абстрактный суперкласс\\
\end{tabular}

\item Приведите пример иерархии из повседневной жизни. Перечислите свойства, 
которые можно найти на каждом уровне, и выделите те, которые находятся на 
более низких уровнях, но не на более высоких уровнях.

\item Прочитайте о механизме классов в Simula (DaW 72) (BirtwistIe 73). 
Сравните и сопоставьте это с механизмом классов \st

\item В реальном мире объекты часто классифицируются ортогональными способами, 
а не в древовидной иерархии Smalltalk. Например, белоголовый орлан и кондор 
являются хищными птицами, но одна\ --- это североамериканская птица, а 
другая\ --- южноамериканская птица. Робин также североамериканская птица, 
но не хищная. Эти две отличительные характеристики являются ортогональными 
в том смысле, что ни одна из них не может быть логически названа надмножеством 
другой. Таким образом. навязывание классификации в древовидную структуру 
является неестественным, неэффективным или и тем, и другим. \\
Как можно классифицировать объекты Smalltalk ортогональными способами? Какие 
проблемы это создает для механизма наследования? Как можно преодолеть эти проблемы?

\end{enumerate}

\secup


\secrel{Syntax}\secdown

В этой главе представлен синтаксис для литеральных объектов (таких как числа) 
и синтаксис для сообщений. В нем объясняется, как использовать систему Little Smalltalk 
для вычисления выражений, набираемых непосредственно с клавиатуры, и как использовать 
несколько простых сообщений для получения информации о различных типах объектов.

\bigskip

В этой главе будет описано, как объекты представляются и 
управляются в Little Smalltalk. Как мы отмечали в главе 1, 
все в \st\ --- это объект. Обсуждение синтаксиса начинается 
с описания того, как объекты представлены.

\secrel{Литеральные константы}

Некоторые объекты, \term{литералы}, отличаются тем, что их имя однозначно идентифицирует 
класс и значение объекта, независимо от контекста, и тем фактом, что их не нужно 
объявлять перед использованием. Например, символ 7, независимо от того, где он 
появляется, всегда обозначает один и тот же объект. В Algol-подобных языках 
такой символ, как 7, обычно обозначает «значение», а не идентификатор. В
\st\ это различие гораздо менее отчетливо. Все объекты, включая числа, являются 
объектами, а объекты характеризуются сообщениями, которые они принимают, и их 
ответами на них. Таким образом, 7 обозначает объект так же, как идентификатор, 
такой как \var{x} (в надлежащем контексте), может обозначать объект.

Числа, пожалуй, самые распространенные литеральные объекты. Существует два класса 
чисел, которые могут быть записаны как литеральные объекты, а именно целые числа и 
значения с плавающей точкой. Числа отвечают на различные арифметические сообщения 
(унаследованные от класса \class{Number}) и сообщения отношений (унаследованные от класса 
\class{Magnitude}). Экземпляр класса \class{Integer} состоит из необязательного знака, за которым 
следует любое количество цифр. Число с плавающей запятой состоит из целого числа, 
за которым следует точка (десятичная точка) и другого целого числа без знака 
(дробная часть) и/или буквы \verb|е| и целого числа со знаком (экспоненциальная часть). 
Любому числу может предшествовать основание системы счисления, которое представляет собой 
положительное целое число, за которым следует буква \verb|r|. Для оснований больше 10 
буквы от A до Z интерпретируются как цифры. Примеры чисел:

\begin{lstlisting}
7
16rFF
-3.1415926
2e32
2.4e-32
15rC.ABC
\end{lstlisting}
    
\noindent
Основание системы счисления в основном используется просто ради удобства и внешнего вида. 
Число 16rFF совпадает с числом 10r255 или просто 255.
    
Класс \class{Char} предоставляет возможности для работы со значениями букв. Буквы отличаются 
от цифр. Поскольку символы имеют порядок, заданный последовательностью сортировки, 
их можно сравнивать и, следовательно, они являются подклассом класса \class{Magnitude}. 
Символ написан в виде знака доллара, за которым следует буква (или цифра). 
Ниже приведены примеры экземпляров этого класса:
    
\begin{lstlisting}
$A
$7
$
$$
\end{lstlisting}
    
Экземпляр класса \class{String} представлен последовательностью букв между одинарными кавычками. 
Встраивание кавычки в строку требует двух соседних кавычек. Строка похожа на массив; 
фактически класс \class{String} является подклассом \class{ArrayedCollection}, как и класс \class{Array}. 
И строки, и массивы могут быть объединены вместе для формирования больших строк 
с помощью оператора \term{конкатенации} строк\ --- запятой (,). Примеры строк:
    
\begin{lstlisting}
'a String'
'a String with an '' embedded quote mark'
\end{lstlisting}
    
Массив \class{Array} записывается в виде знака фунта (\verb|#|), за которым следует список элементов 
массива в скобках. Элементами массива являются литеральные объекты (числа или символы), 
строки или другие массивы. В списке массивов ведущий знак фунта на символах 
и массивах может быть исключен. Примеры массивов:

\begin{lstlisting}
#(this is an array of symbols)
#(12 'abc' (another array))
\end{lstlisting}

\noindent
Массивы и строки используют сообщения \verb|at:| и \verb|at:put:| 
для выбора и изменения определенных элементов в их коллекциях.

Класс \class{Symbol}\ --- это еще один литеральный класс. Символ записывается в виде 
знака фунта (\verb|#|), за которым следует любая последовательность букв. 
Пробелы между символами не допускаются. В отличие от строки (которая также 
является последовательностью букв) символ не может быть разбит на более 
мелкие части. Кроме того, одна и та же последовательность букв, используемая 
в разных местах, всегда будет обозначать один и тот же объект. В отличие от 
чисел, символов или букв, символы не имеют порядка и не могут сравниваться 
(за исключением, конечно, равенства объектов). Примеры символов:

\begin{lstlisting}
#aSymbol
#AndAnother
#+++
#very.long.symbol.with.periods
\end{lstlisting}

\secrel{Identifiers}

Identifiers in Little Smalltalk can be divided into three categories: instance
variables, class names, and pseudo-variables. An identifier beginning with
a capital letter is always a class name, whereas an identifier beginning
with a lowercase letter must represent either a pseudo variable or an
instance variable.

At the command level, new instance variables can be defined merely
by assigning a value to a name. The assignment arrow is formed as a twocharacter sequence consisting of a less than sign and a minus sign
\note{From now on the text will use the symbol $\leftarrow$ to represent this two-character sequence}:
\begin{lstlisting}
newname <- 17
\end{lstlisting}

Instance variables defined at the command level are known only at the
command level and cannot be used within a method for any class. As we
will see in a later chapter, instance variables within a class must all be
declared.

Class identifiers respond to a variety of messages that can be used to
discover information concerning the class the object represents. For example, the message respondTo, when passed to an object representing a
class, will cause the class to print a list of the messages to which instances
of the class will respond.

Pseudo variables look like normal identifiers (that is, they are named
by a sequence of letters beginning with a lower case letter), but unlike
identifiers they need not be declared. There are several pseudo variables:
self, super, selfProcess
\note{The pseudo-variables selfprocess and smalltalk are unique to Little Smalltalk and are
not part of the \st-80 system, where different techniques are used to obtain the currently executing process or to obtain information about the current environment. See Appendix 5 for an overView of the differences between Little Smalltalk and the Smalltalk-80
programming environment.}
, true, false, nil, and smalltalk. Arguments for
a method (to be discussed shortly) are also considered to be. pseudo-variables. Of the seven, self, super, and selfProcess are farthest from being
literal objects because their meaning depends entirely upon context. We
will discuss these in more detail when we describe class methods and
processes. The next three, true, false, and nil, are defined to be instances
(usually the only instances) of the classes True, False, and UndefinedObject, respectively. We will discuss these three in more detail when
we outline the behavior of different classes. The final pseudo variable,
smalltalk, is an instance of class Smalltalk and is used to centralize several
pieces of information concerning the currently executing environment.

Other types of objects in the Little Smalltalk system, such as blocks
and instances of user defined classes, will be discussed in later sections.

\secrel{Messages}

As noted in Chapter I, all actions in Smalltalk are produced by sending
messages to objects. This section begins by describing the syntax used to
produce messages.

Any message can be divided into three parts; a receiver, a message
selector, and zero or more arguments. The receiver and argument portions
of a message can be specified by other message expressions, or they may
be specified by a single token, such as an identifier or a literal.

The first type of message selector requires no arguments and is called
a unary message. A unary message selector consists of an identifier, the
first letter of which must be lowercase. For example:
\begin{lstlisting}
7 sign
\end{lstlisting}
illustrates the message sign being passed to the number 7. Unary messages,
like all messages, elicit a response, which is simply another object. The
response to sign is an integer, either -1,0, or 1, depending upon the sign
of the object the message was sent to (the receiver). Unary messages parse
left to right, so, for example:
\begin{lstlisting}
7 factorial sqrt
\end{lstlisting}
returns $\sqrt{7!}$, or approximately 70.993.    

The second form of message, called a binary message, takes one argument. 
A binary message is formed from one or two adjacent nonalphabetic characters\note{Some 
characters, such as braces, parenthesis or periods, cannot be used to form
binary messages. See the description in Appendix 2 for a more complete description of the
restrictions.}.
Binary messages tend to be used for arithmetic
operations, although this is not enforced by the system and there are
notable exceptions. An example of a binary message is arithmetic addition:
\begin{lstlisting}
7 + 4
\end{lstlisting}
At first the fact that this is interpreted as "send the message + with
argument 4 to-the object 7"may seem strange; however, soon the uniform
treatment of objects and message passing in Smalltalk makes this seem
natural.

Binary messages, like unary messages, parse left to right. Thus
\begin{lstlisting}
7 + 4 * 3
\end{lstlisting}
results in 33, not 19. \emph{Unary messages have a higher precedence than binary
messages}, thus
\begin{lstlisting}
7 + 17 sqrt
\end{lstlisting}
evaluates as $7 + (\sqrt{17})$, not $\sqrt{(7 + 17)}$.

The most general type of message is a keyword message. The selector
for a keyword message consists of one or more keywords. Each keyword
is followed by an argument. A keyword is simply an identifier (again, the
first character must be lower case) followed by a colon. The argument can
be any expression, although if the expression is formed using a keyword
message, it must be placed in parentheses to avoid ambiguity. Example
keyword expressions are:
\begin{lstlisting}
7 max: 14.
7 between: 2 and: 24
\end{lstlisting}

When we wish to express the name of the message being requested by
a keyword message, we catenate the keyword tokens. Thus we say the
message selector being expressed in the second example above is between:and:. There can be any number of keywords in a keyword message,
although in practice few messages have more than three.

Keyword messages have lower precedence than either binary or unary
messages. Thus
\begin{lstlisting}
7 between: 2 sqrt and: 4 + 2
\end{lstlisting}
\begin{lstlisting}
7 between: (2 sqrt) and: (4 + 2)
\end{lstlisting}
            
\secrel{Getting Started}

You now have enough information to try getting some hands-on experience
using the Little Smalltalk system. After logging on, type the command st.
After a moment, the message "Little Smalltalk" should appear, and the
cursor should be indented by a small amount on the next line. If, at this
point, you type in a Smalltalk expression and hit the return key, the expression will be evaluated and the result printed. Try typing "3 + 4" and see
what happens. The result should be a 7, produced at the left margin. The
cursor then should advance to the next line and once more tab over several
spaces. Try typing "5 + 4 sqrt." Can you explain the outcome? Try "(5 +
4) sqrt."

Try typing
\begin{lstlisting}
i <- 3
\end{lstlisting}
Notice that, since assignment expressions do not have a value, no value
was printed. However, if you now type
\begin{lstlisting}
i
\end{lstlisting}
the most recent object assigned to the name will be produced.

The name last always contains the value of the last expression computed. Try typing
\begin{lstlisting}
27 + 3 sqrt
\end{lstlisting}
followed by
\begin{lstlisting}
last
\end{lstlisting}

\secrel{Finding Out About Objects}

There are various messages that can be used to discover facts about an
object. The message class, for example, will tell you the class of an object.
Try typing
\begin{lstlisting}
7 class
\end{lstlisting}
The message superClass, when passed to an instance of Class, will return
the immediate superclass of that class. Try typing
\begin{lstlisting}
Integer superClass
7 class superClass
\end{lstlisting}
What is the superclass of Object?

The keyword message respondsTo: can be used to discover if an object
will respond to a particular message. The argument must be a symbol,
representing the message. Try typing
\begin{lstlisting}
 7 respondsTo: #+
$A respondsTo: #between:and:
$A respondsTo: #sqrt
\end{lstlisting}
When passed to a ciass, the message respondTo: inquires whetherinstances
of the class respond to the given message. For example,
\begin{lstlisting}
Integer respondsTo: #+
\end{lstlisting}

You can discover if two objects are the same using the binary message
==. The message \verb|~~| is the logical inverse of ==. Try typing
\begin{lstlisting}
i <- 17
i == 17
17~~17
\end{lstlisting}

One way to tell if an object is an instance of a particular class is to
connect the unary message class and the binary message ==. Try typing
\begin{lstlisting}
i class == Integer
\end{lstlisting}

A simple abbreviation for this is the message isMemberOf:. For example, the last expression given is equivalent to
\begin{lstlisting}
i isMemberOf: Integer
\end{lstlisting}

Suppose we want to tell if an object is a Number, but we don't care if
it is any particular kind of number (Integer or Float). We could use the
boolean OR bar (I), which is recognized by the boolean values true and
false:
\begin{lstlisting}
(i isMemberOf: Integer) | (i isMemberOf: Float)
\end{lstlisting}
A simplier method is to use the message isKindOf;. This message asks
whether the class of the object, or any of superclasses, is the same as the
argument. Try typing
\begin{lstlisting}
i isKindOf: Number
\end{lstlisting}

\secrel{Blocks}

An interesting feature of Smalltalk is the ability to encapsulate a sequence
of actions and then to perform those actions at a later time, perhaps even
in a different context. This feature is called a block (an instance of class
Block) and is formed by surrounding a sequence of Smalltalk statements
with square braces, as in:
\begin{lstlisting}
[ i <= i+1. i print ]
\end{lstlisting}

Within a block (and, as we will see in the next chapter, in general
within a method) a period is used as a statement separator. Since a block
is an object, it can be assigned to an identifier or passed as an argument
with a message or used in any other manner in which objects may be used.
In response to the unary message value, a block will execute in the context
in which it was defined, regardless of whether this is the current context
or not. That is, when the block given above is evaluated, the identifier i
will refer to the binding of the identifier i that was known at the time the
block was defined. Even if the block is passed as an argument into a class
in which there is a different instance variable i and then evaluated, the i
in the block will refer to the i in the context in which the block was defined.
Thus a block when used as a parameter is similar to the Algol-60 call-byname notion of a thunk.

The value returned by a block is the value of the last expression inside
that block. Frequently a block will contain a single expression, and the
value resulting from that block will be the value of the expression.

One way to think about blocks is as a type of in-line procedure declaration. Like procedures, a block can also take a number of arguments.
Parameters are denoted by colon-variables at the beginning of the block,
followed by a vertical bar and then the statements composing the block.
For example,
\begin{lstlisting}
[:x :y | (x + y) print]
\end{lstlisting}
is known as a two-parameter block (sometimes two-argument block). The
message value: is used to evaluate a block with parameters the number of
value: keywords given matches the number of arguments in the block. So,
for the example given above, the evaluating message would be value:value:.

\secrel{Comments and Continuations}

A pair of double quote marks (II) are used to enclose a comment. One
must be careful not to confuse the double quote mark with two adjacent
single quote marks CI), which look very similar. The text of the comment
can be arbitrary and is ignored by the Little SmaIItalk system.

The Little Smalltalk system assumes that each line typed at the terminal
is a complete SmaIItalk expression. Should it be necessary to continue a
long expression on two or more lines, a special ineJication must be given
to the Little SmaIItaik system to prevent it from misinterpreting the partial
expression on the first line and generating an unintentional error message.
This special indication is a backwards slash C""-) as the last character on
all intermediate lines, for example:
\begin{lstlisting}
    2 +     \
    3 * 7   \
    + 5     \
40
\end{lstlisting}


\secly{EXERCISES}

\begin{enumerate}

    \item Show the order of evaluation for the subexpressions in the following
    expression:
    \begin{lstlisting}
        7/2 between: 7 + 17 sqrt and: 3 * 5
    \end{lstlisting}

    \item Type the following expressions:
    \begin{lstlisting}
        7 = = 7
        label = = label
        #abe = = #abc
    \end{lstlisting}
    How do you explain this behavior?
    
    \item What values will be printed in place of the question marks in the
    following sequence:
    \begin{lstlisting}
            i <- 17
            j<-[i< +1]
            j print
        ??
            j print
        ??
            i < - 23
            i print
        ??
            j value print
        ??
            i print
        ??
            j value print
        ??
            i print
        ??
    \end{lstlisting}

\end{enumerate}

\secup


\secrel{Basic Classes}\secdown

Основные классы, включенные в стандартную библиотеку Little Smalltalk, описаны в этой главе.

\bigskip

The classes included in the Little Smalltalk system can be roughly divided
into four overlapping groups. These groups are Basic Objects (Object,
UndefinedObject, Symbol, Boolean, True, False, Magnitude, Number,
Char, Integer, Float, Radian, Point), Collections (Collection, Bag, Set,
SequenceableCollection, KeyedCollection, Dictionary, Interval, List,
ArrayedCollection, Array, String, File, Random, ByteArray), Control
Structures (Boolean, True, False, Interval, Block), and System Management (Object, Class, Smalltalk, Process). The following sections will
briefly describe each of these categories. Appendix 3 provides detailed de;.
scriptions for each of the standard classes.

\secrel{Basic Objects}

The class Object is a superclass of all classes in the system and is used to
provide a consistent basic functionality and default behavior. For example,
the message = = is defined in class Object and is thus accepted by all
objects in the Little Smalltalk system. This message tests to see if the
expressions for the receiver and the argument represent the same object.
Another message defined in class Object is the message class, which returns the object representing the class of the receiver.

The last chapter introduced the classes associated with literal objects.
other types of objects are also basic to many applications. For example,
instances of the class Radian are used to represent radians. A radian is a
unit of measurement, independent of other numbers. Only radians will
respond to trigonometric functions such as sin and cos. Numbers can be
converted into radians by passing them the message radians. Similarly,
radians can be converted into numbers by sending them the message
asFloat. Only a limited range of arithmetic operations on Radians, such
as scaling by a numeric quantity or taking the difference of two radians,
are permitted. Radians are normalized by adding or subtracting multiples
of 2'11" from their value.

The class Point is used to represent ordered pairs of quantities. Ordered pairs are useful in the solution of many problems, such as storing
coordinate pairs in graphics applications. In fact, the class Number provides a convenient method for constructing points. All instances of class
Number will respond to the message @ by producing a point consisting
of the receiver and the argument. Thus 10 @ 12 generates a point representing the ordered pair (l0,12). The first value is known as the x-value
and will be returned in response to the message x. The second value is the
y-value and is returned in response to the message y

The class String provides messages useful in manipulating arrays of
characters. One important property of this class is that its instances are
the only objects in the Little Smalltalk system that can be displayed on an
output device such as a terminal or printer. Any object to be displayed
must first be converted into an instance of class String. The behavior
defined in class Object for the message print is to convert the object into
a string (using the message printString) and then to print that string (by
passing the message print to it).

The message printString is uniformly interpreted throughout the Little
Smalltalk system as "produce a string representation ofyour value." Classes
for which this makes sense (such as Integer) must define a method for
this message that will produce the appropriate string. By default (that is,
by a method in class Object that will be invoked unless overridden), a
string containing the name of the class of the object is produced. In subsequent chapters we will see several examples of how different classes
respond to the message printString.

\secrel{Collections}

The different subclasses and varieties of Collection provide the means for
managing and manipulating groups of objects in Smalltalk. The different
forms of collections are distinguished by several characteristics, whether
the size of the collection is \&ed or unbounded, the presence or absence
of an ordering, and their insertion or access method. For example, an array
is a collection with a fixed size and ordering, indexed by integer keys. A
dictionary, on the other hand, has no fixed size or ordering and can be
indexed by arbitrary elements. Nevertheless, arrays and dictionaries share
many features, such as their access method (at: and at:put:) and their ability
to respond to collect:, select:, and many other messages.

The table below lists some of the characteristics of several forms of
collections.

\fig{language/tbl1.png}{width=\textwidth}

Collections of one type can frequently be converted into collections of
different type by sending an appropriate message, for example, asBag (to
convert a collection into a Bag), asSet or asArray.

We can group the operations into several categories, independent of
the type of collection involved. The first basic action is adding an element
to a collection. Here collections divide into two groups. Those collections
that are indexed (Dictionary, Array) must be given an explicit key and
value, and, thus, the insertion method is the two-argument message at.·put:.
Those collections that are not indexed store only the value and thus use
the one argument message add:. A special case of this is the class List,
which maintains elements in a linear ordering. Here, values can be added
to the beginning or end of the collection by using the messages addFirst:
and addLast:.

Protocols for adding an element to a collection are similar to those for
removing an element from a collection. In collections that do not require
a key, an element can be removed with the message remove:, the argument
being the object to be removed. In keyed collections, the removal message
uses the key (removeKey:), and not the value. In collections with fixed sizes
(Array and String), elements cannot be removed. In a List, an element
can be removed from either the beginning of the list (removeFirst) or the
end of the list (removeLast).

Once an element has been placed into a collection, the next step is to
access the element. Those collections using keys require a key for access
and use the message at:. For those that do not require a key, the only
question (since one already has the value)-is whether the value is in the
collection. Thus the appropriate message is includes: (which also works
for keyed collections). A special case is List where one can access either
the beginning or the end of the list by using the messages first and last.

The access methods at: and includes: access a value by position. Frequently, however, one needs to access an element by value without knowing
a position. For example, one may want to find the first positive element
in an array of integers. To facilitate this search there is a message named
detect:. The message detect: takes as an argument a one-parameter block.

It evaluates the block on each element in the collection and returns
the value of the first element for which the block evaluates true. For example, if x is an array containing numeric values, the message detect: could
be used to discover the first positive value.
\begin{lstlisting}
    x # ( - 2 - 3 4 5)
    x detect: [ :y Iy > 0 ]
4
\end{lstlisting}

An error message is produced and nil returned if no value satisfies the
condition. This can be changed using the message detect:ifAbsent:
\begin{lstlisting}
        x detect: [ :y I y > 10 ]
    error: no element satisfies condition
    nil
        x detect: [ :y I y > 10] ifAbsent: [ 23 ]
    23
\end{lstlisting}

In ordered collections, the search is performed in order, whereas in
unordered collections, the search is implementation dependent, and no
specific order is guaranteed.

If, instead of finding the first element that satisfies some condition,
you want to find all elements of a collection that satisfy some condition,
then the appropriate message is select:. Like detect:, select: takes as an
argument a one-parameter block. What it returns is another collection, of
the same type as the receiver, containing those values for which the argument block evaluated true. A similar message, reject:, returns the complementary set.
\begin{lstlisting}
        x select: [ :y Iy > 0 ]
    #( 45)
        x reject: [ :y Iy > 0 ]
    #( -2 -3)
\end{lstlisting}

The message do: can be used to perform some computation on every
element in a collection. Like select: and reject:, this message takes a oneargument block. The action performed is to evaluate the block on each
element of the collection.
\begin{lstlisting}
        x do: [ :y I(y + 1 ) print]
    -1
    -2
    5
    6
\end{lstlisting}

The message do: returns nil as its result. If, instead of performing a
computation on each element, you want to produce a new collection containing the results, the message collect: can be used. Again like select: and
reject:, this message takes as argument a one-parameter block and returns
a collection of the same variety as the receiver. The elements of the new
collection, however, are the results of the argument block on each element
of the receiver collection.
\begin{lstlisting}
    x collect: [ :y Iy sign]
#( -1 -1 1 1 )
\end{lstlisting}

Frequently the solution to a problem will involve processing all the
values of a collection and returning a single result. An example would be
taking the sum of the elements in a numerical array. In Little Smalltalk,
the message used to accomplish this is inject:into: The message inject:into:
takes two arguments: a value and a two-parameter block. The action performed in response to this message is to loop over each element in the
collection, passing the element and either the initial value or the result of
the last iteration as arguments to the block. For example, the sum of the
array x could be produced using inject:
\begin{lstlisting}
    x inject: 0 into: [ :a :b I a + b ]
4
\end{lstlisting}
The following command returns the number of times the value 4 occurs
in x:
\begin{lstlisting}
    x inject: 0 into: [ :a :b I(a = = 4) ifTrue: [ b + 1 ] ifFalse: [ b ]]
1
\end{lstlisting}

We have described the broad categories of messages used by collections. There are many other messages specific to certain classes; they are
described in detail in Appendix 3. We next will provide a brief overview
of the most common types of collections.

The classes Bag and Set represent unordered groups of elements. An
element may appear any number of times in a bag but only once in a set.
Elements are added and removed by value.

A Dictionary is also an unordered collection of elements; however,
unlike a bag, insertions and removal of elements from a dictionary requires
an explicit key. Both the key and value portions of a dictionary entry can
be any object, although commonly the keys are instances of String, Symbol or Number

The class Interval represents a sequence of numbers in an arithmetic
progression, either ascending or descending. Instances of Interval are
created by numbers in response to the message to: or to:by:. In conjunction
with the message do:, an Interval creates a control structure similar to do
or for loops in Algol-like languages.
\begin{lstlisting}
        (1 to: 10 by: 2) do: [ :x I x print]
    1
    3
    5
    7
    9
\end{lstlisting}
Although instances of class Interval can be considered to be a collection,
they cannot have additional elements added to them. They can, however,
be accessed randomly using the message at:.
\begin{lstlisting}
    (2 to: 7 by: 3) at: 2
5
\end{lstlisting}

A List is a group of objects having a specific linear ordering. Insertion
and removal is from either the beginning or the end of the collection. Thus
a list can be used to implement both a stack and queue.

A File is a type of collection in which the elements of the collection
are stored on an external medium, typically a disk. A file can be opened
in one of three modes. In character mode every access or read returns a
single character from the file. In Integer mode every read returns a single
word as an integer value. In string mode every read returns a single line
as an instance of class String. Elements cannot be removed from a file,
although they may be overwritten. Because access to external devices is
typically slower than access to memory, many of the operations on files
may be quite slow.

An Array is perhaps the most commonly used data structure in Little
Smalltalk programs. Arrays have fixed sizes, and, while elements cannot
be inserted or removed from an array, the elements can be overwritten.
Literal arrays can be represented by a pound sign preceding a list of array
elements, for example:
\begin{lstlisting}
#( 2 $a 'joel 3.1415 )
\end{lstlisting}

A String can be considered to be a special form of array, where the
elements must be characters. In addition, as we have been illustrating in
many examples, a literal string can be written by surrounding the text with
quote marks.

The class ByteArray represents a special form of array where each
element must be a number in the range 0 through 255. Byte arrays are
used extensively in the internal representations of objects in the Little
Smalltalk system. Byte arrays can be written as a pound sign preceding a
list of elements enclosed in square braces, for example:
\begin{lstlisting}
    #[ 0 127 32 115 ]
\end{lstlisting}

There are two other classesthat are commonly used to representgroups
of data, although they are not subclasses of Collection. The class Point,
already discussed, can be considered to be a small collection of two items.
The class Random can be thought of as providing protocol for an infinite
collection of pseudo-random numbers. This "list," of course, is never actually created in its entirety; rather each number is generated as required
in response to the message next. The values produced by instances of class
Random are floating values in the range 0.0 to 1.0. Other messages can be
used to convert this into either an integer or a floating value in any range.

\secrel{Control Structures}

One of the more surprising aspects of Smalltalk is the fact that control
structures are not provided as part ofthe basic syntax but rather are defined
using the message passing paradigm. The basic control structure in Smalltalk, as in most computer languages, is the conditional test: IF some condition is satisfied THEN perform some actions ELSE perform some other
actions. In Smalltalk this is accomplished by passing messages to instances
of class Boolean. The class True (a subclass of Boolean) defines methods
for the messages ifTrue: and ifFalse: (similar methods are defined for class
False). The arguments used with these messages are blocks. If the condition is satisfied (Le., the receiver is true and the message is ifTrue:, or
the receiver is false and the message is ifFalse:), the argument block is
evaluated, and the result it produces is returned. If the condition is not
satisfied, the value nil is returned.
\begin{lstlisting}
        (3 < 5) ifTrue: { 17 ]
    17
        (3 < 5) ifFalse: { 17 ]
    nil
\end{lstlisting}

The combined forms ifTrue:ifFalse: and ifFalse:ifTrue: are also recognized:
\begin{lstlisting}
        (3 < 5) ifTrue: { 17 ] ifFalse: { 23 ]
    17
        (3 > 5) ifTrue: { 17 ] ifFalse: { 23 ]
    23
\end{lstlisting}

The message and: and or: are similar to ifTrue: and ifFalse:. They are
also used with booleans and passed as arguments objects.of class Block.
And: i and or: provide "short circuit" evaluation of booleans; that is, the
argument block is evaluated only if necessary to determine the result of
the boolean expression.
\begin{lstlisting}
((i < 10) and: { (b at: i) = 4] ifTrue: { i print]
\end{lstlisting}
In this example, the expression "(b at:i) =4" will be evaluated only if the
expression <10<10)" is true. If the first expression returns false, the argument block used with the and: message is not 
evaluated\note{In actual fact the Little Smalltalk parser will, for efficiency, often optimize conditions
to remove the message passing overhead. Nevertheless, the underlying paradigm holds true
and will, in fact, be used under some conditions (for example, when the arguments are not
a block).}.' Notice that the
relational < returns either true (an instance of class True) or false (an
instance of class False) and that the message and: is implemented in class
Boolean, a superclass of both True and False. Various other boolean
operations, such as not are also defined in this class.

Next to conditional tests, the most common control structure is a loop.
A loop is produced by passing the timesRepeat: message with a parameterless bock as an argument to an integer. The value of the integer is the
number of times to execute the loop. For example;
\begin{lstlisting}
        5 timesRepeat:{ 8 print]
    8
    8
    8
    8
    8
\end{lstlisting}
will print the number 8 five times.

A more general loop is used to produce numbers in arithmetic progression. The messages to: and to:by:, when passed to a number, produce
an instance of class Interval. As we noted in the last section, an interval
is a collection of values in arithmetic progression. We can then use the
message do: to enumerate the elements in the progression. For example:
\begin{lstlisting}
        (2 to: 9 by: 3) do: [ :x I x print]
    2
    5
    8
\end{lstlisting}

A more general form of loop is the while loop. A while loop is formed
using blocks as both receiver and argument. The result of the receiver
block must be a boolean. The actions performed by the block are to evaluate
the receiver block and, as long as it returns true, to evaluate the argument
block. For example, by using an additional variable the previous loop could
have been written:
\begin{lstlisting}
        i~2.
        [ i < = 9 ] whileTrue: [ i print. i i + 3 ]
    2
    5
    8
\end{lstlisting}
Since both the receiver and the argument block can contain any number
of expressions (the value of a block is always the value ofthe last expression,
regardless of how many expressions the block contains), sometimes no
argument block is necessary. The unary message whileTrue (or whileFalse) can then be used. For example:
\begin{lstlisting}
        i~2.
        [i print. i i + 3 . i < = 9] whileTrue.
    2
    5
    8
\end{lstlisting}

\secrel{Class Management}

The class Class is used to provide protocol for manipulating classes. Thus,
for example, the methods new and new: are implemented in class Class
to allow instance creation. Classes themselves cannot be created by new
but must be generated by compilation (which is the topic of the next
chapter).

The messages new and new: are treated differently in one respect by
the Little Smalltalk system: if a class defines a method for these messages,
then, each time a new instance of the class is created (by sending the
message new or new: to the class object), the newly created object is immediately initialized by sending it the same message, and the resulting
object is returned to the user. This happens at all levels of the class hierarchy, even if the message is defined multiple times. (That is, later
definitions of new are in addition to, and do not override, definitions higher
in the class hierarchy). One should be careful to distinguish the message
new passed to the class object used to create the object from the same
message passed to the newly created object used for initialization. Since
the second message is produced internally, and not by the user, it is easy
to overlook.
\begin{lstlisting}
        Array new: 3
    #( nil nil nil )
        UnoefinedObject new
    nil
\end{lstlisting}

The argument used with new: is not used by the object creation protocol
but only by the object initialization method. In later chapters we will see
how this feature can be used to automatically initialize objects.

The class Smalltalk provides protocol for the pseudo variable smalltalk. By passing messages to smalltalk, the user ~an discover and set
many attributes of the currently executing environment.
\begin{lstlisting}
    smaIitaIk date
Fri May 24 14:03: 16 1985
\end{lstlisting}

Another message, time:, requires a block as argument. The integer value
it returns represents the number ofseconds elapsed in evaluating the block.
\begin{lstlisting}
    smalltalk time:[ ( 1 to: 10000 ) do: [:x I] ]
104
\end{lstlisting}

Smalltalk is a subclass of Dictionary and thus responds to the messages at: and at:put:. Since smalltalk is accessible anywhere in any object,
it can be used to pass information from one object to another or to provide
global information used by a number of objects. Of course, it is the usees .
responsibility to insure that two objects do not try to store different information using the same key. With the exception of message passing, this
pseudo variable is the only means of communication between separate
objects. Although permitted, the use of the pseudo-variable in this manner
is at odds with the pure object-oriented philosophy of Little Smalltalk and
should be discouraged. The necessity for global variables is often the mark
of a poorly developed solution to a problem.

The pseudo variable smalltalk also provides a means to construct and
evaluate messages at run time by using the message perfonn:with
Arguments:. The first argument to this message must be a symbol indicating
the message to be processed. The second argument must be an array
representing the receiver and arguments to be processed. The second argument must be an array representing the receiver and arguments to be
used in evaluating the message. The response is the value returned by the
first argument of this array in response to the message, with the remainder
of the arguments in the second array as the argument values for the message. For example:
\begin{lstlisting}
    smalltalk perform: #between:and: withArguments: #(3 1.03.14)
True
\end{lstlisting}

An instance of class Process is used to represent a sequence of Smalltalk statements in execution. Processes cannot be created directly by
the user but are created by the system or by passing the message newProcess or fork to a block. Processes will be discussed in more detail in
Chapter 10.

\secrel{Abstract Superclasses}

We did not discuss the classes Collection, KeyedCollection,
SequenceableCollection, or ArrayedCollection in the last section, even
though they were listed as forms of "collection" in the beginning of this
chapter. These classes, along with such classes as Boolean, Magnitude,
or Number, are what are known as abstract superclasses. Instances of
abstract superclasses are seldom useful by themselves, but the classes are
important in providing methods that can be inherited by subclasses. For
example, while it is legal in Smalltalk to say:
\begin{lstlisting}
x Collection new
\end{lstlisting}
and the resulting variable x will indeed be an instance of class Collection,
the object is not particularly useful. It has no insertion or deletion protocol,
for example. An instance of class Set, however, is very useful, and the
messages defined in class Collection are important in providing functionality to objects of this class and to other subclasses of class Collection.

The selection and design of abstract superclasses is one of the more
important arts in Smalltalk programming. For example, if one were designing a system to manipulate banking accounts, a class Account might be
a useful abstract superclass for classes CheckingAccount and
SavingsAccount. The actions specific to the individual types of accounts
would be in the subclasses, whereas any common behavior, such as the
actions necessary for opening and closing an account or querying the
balance, might be implemented in the superclass.


\secly{EXERCISES}

\begin{enumerate}

\item Suppose you created a new instance of the class UndefinedObject, as
follows:
\begin{lstlisting}
    I~UndefinedObject new
\end{lstlisting}
How does i respond to print? To isNil or notNil? To the object comparison message = = with nil ? Is i nil? List facts to support your
answer.

\item Note that the messages arcSin and arcCos produce an object of type
Radian and not oftype Float. Furthermore, only objects oftype Radian
respond to sin and cos. An alternative would have been to eliminate
the class Radian and to permit all objects of class Float to respond
to the messages sin and cos. Discuss the advantages and disadvantages
of these two different arrangements.

\item Suppose you have a Bag containing numbers. How would you go about
producing an instance of the class List containing the numbers listed
in sorted order?

\item What is the class of Class? what is the superclasss of Class?

\item Many times, two ormore different sequences ofmessages to collections
will have the same effect. In each of the following, describe a sequence
of messages that will have the same effect ,\S the indicated message.
\begin{enumerate}
    \item implement reject: in terms of select:
    \item implement size in terms of inject:into:
    \item implement includes: in terms of inject:into:.
\end{enumerate}

\end{enumerate}

\secup


\secrel{Class Definition}\secdown

В этой главе представлен синтаксис, используемый для определения классов, и пример определения.

\bigskip

The last chapter introduced some of the standard classes in the Little
Smalltalk system. This chapter will show how the user can define new
classes to provide additional functionality for the Smalltalk language.

Class descriptions cannot be entered directly at the keyboard during
a Little Smalltalk session. Instead, class descriptions must be prepared
beforehand in a file, using a standard editor. There can be any number of
class descriptions in a file, although it is convenient to keep each class
description in a separate file for ease in program development. The textual
representation of a class description consists of two main parts, the class
Heading, followed by a sequence of one or more m.ethod descriptions. We
will discuss each portion in turn. The syntax for class descriptions is given
in detail in Appendix 2.

Figure 4.1 shows a prototypical class description, in this case the description for the class Set. The initial part of the description, the class
header, consists of the keyword Class (the first letter capitalized) followed
by a class name (also capitalized). Following the class name, the superclass
name preceded by a colon, is given. The superclass name is optional and,
if not given, the class Object will be assumed.

After the class and superclass names, the class description lists instance
variable names. Instaqce variables provide the local memory for instances
of the class. Each class instance is given its own set of instance variables,
and no other object can modify or change an instance variable in another
object. The list of instance variable names is surrounded by a pair of
vertical bars. Note that Smalltalk has no notion of variables being declared
to be of a specific type, thus any instance variable can contain any accessible
object or expression. Although the syntax is free form, it is conventional
to place the instance variables on a line separate from the class name.
Instance variables are initially given the value nil. Ifa class does not contain
any instance variables, the entire list, including the vertical bars, can be
omitted. Following the heading, a pair ofsquare braces surround the methods that comprise the protocol for the class

Each method in the class protocol defines how instances of the class
will respond.to a single message. The particular message being defined by
the method is given by the message pattern. The pattern is the first of the
three major portions ofa method; the other two portions are a list of
temporary variables and the message body.

\fig{language/fig_4_1.png}{height=.6\textheight}

A message pattern defines both the name of the message for which the
method is providing protocol and the names to be used when referring to
parameters associated with the message. There are three forms of message
pattern, corresponding to the three forms of messages (unary, binary, and
keyword). Example methods using two of those forms are given in Figure
4.2. These methods are from class Collection. Note that method descriptions are separated by vertical bars. The one exception to the free form
notation for class descriptions is that this vertical bar must appear in the
first column.

An optional list of temporary identifier names can appear following a
message pattern. For example, the message description for the message
do: in Figure 4.2 lists a temporary identifier named item. Temporary identifiers are created when the method is initiated in response to a message,
and they exist as long as the message body is being executed or there is a
block in existence that can refer to the temporary value. Like instance
variables, temporary variables are initially given the value nil.

A method body consists of a sequence of one or more Smalltalk expressions separated by periods. Note that the period is an expression separator,
not an expression terminator, and does not follow the final expression in
the method body. The final expression in a method body, or the final
expression in any block created within the method body, can be preceded
by an up arrow to indicate that the following expression is to be returned
as the response to the message. (On some terminals the up arrow looks
like a <lcaret" or".) If no explicit value is returned, the default action at the
end of a method is to return the receiver as the response to the messae.

\fig{language/fig_4_2.png}{width=\textwidth}

\clearpage\noindent
Note that the up arrow indicates an immediate return from the current
method description even ifthe arrow occurs within a block. This is different
from a block returning a value, which, as we saw in Chapter 2, is implicitly
the value of the last expression in the block.

Within a method body there are four types of variables that can be
accessed, namely instance variables, argument variables, temporary variables, and pseudo variables. In addition to the pseudo variables discussed
in the last chapter (true, false, nil and smalltalk), the pseudo variables
self, super, and selfProcess can be used. Both the variables self and super
refer to the receiver of the current message. When a message is passed to
self, the search for a method begins with the class description associated
with the class of the receiver. On the other hand, when a message is passed
to super, the search begins with the class description of the superclass of
the class in which the current message is defined. For example, if an
expression in class Set (Figure 4.1) passed a message to super, the search
for a matching method would begin in class Collection, the superclass of
class Set. Amessage passed to selfwould initiate a search for an associated
method in class Set.

To illustrate the actions of self, suppose variable "x" is an instance of
class Set. The methods shown in Figure 4.2 are from class Collection, a
superclass of class Set. In response to the message "x isEnzpty, " the method
isEmpty shown in Figure 4.2 is initiated. This method, in turn, passes the
message size to the pseudo variable self, which in this case represents x.
Thus the search for a nlethod matching the message size would begin in
class Set, the class of x. If, on the other hand, the method for isEmpty had
passed the message size to the pseudo variable super, the search for a
corresponding method would have begun in the class Object, the super
class of Collection.

Two messages are singled out for special treatment. If either of the
messages new or new: is defined in a class protocol, then, when an instance
of that class is created, the associated message (either new or new:) will
automatically be passed to the new instance before any further processing.
Thus, these messages can be used to provide automatic initialization of
instance variables. The next section describes in detail one class, the class
Set, and illustrates the use of this feature.

\secrel{An Illustrative Example}

Figure 4.3 gives the complete class description of the class Set. A set is a
collection of objects; each object occurs no more than once in the collection. (See Exercise 2.) The data structure used to implement the set will
be a list. Recall that a list is also a collection but one that maintains values
in order.

\fig{language/fig_4_3_a.png}{width=.7\textwidth}

\fig{language/fig_4_3_b.png}{width=.7\textwidth}

As we have already seen, the class Set maintains one instance variable,
list. Each instance of class Set will have a separate instance variable that
cannot be accessed by any other object. The method for message new will
automatically create an instance of class List in the variable list whenever
an instance of class Set is created.

When an element is added to a set, a test is first performed to see if
the element is already in the set. If so, no further processing is done; if
not, the element is added to the underlying Jist. Similarly, to remove an
element, the actions in the method in Set merely pass the removal message
on to the underlying list. Note that the long form of the remove: message
is defined; it includes a second argument, indicating what actions should
be taken if the given item is not found. As we saw in Figure 4.2, the class
Collection defines the short message (re1110ve:) in terms of this longer
message. Thus either form ofthe message can be used on a Set.

The messages first and next are used to produce generators. Generators
will be the topic of a later chapter; it is sufficient to say here that first and
next provide a means to enumerate all elements in a collection. The message first can be interpreted informally as flinitialize the enumeration of
items in the collection and return the first item, or nil if there are no items
in the collection." Similarly, next can be defined as "return the next item
in the collection, or nil if there are no remaining items." The protocol for
do: (Figure 4.2) illustrates how these messages can be used to construct a
loop to access every item in a collection. In Class Set, the generation of
items in response to these messages is provided by passing the same messages to the underlying list.

\secrel{Processing a Class Definition}

Once you have created a class description, the next step is to add the class
to the collection ofstandard classes provided by the Little Smalltalk system.
Suppose, for example, you have defined a class Test in the file test.st; to
add this class definition to a running Little Smalltalk execution you would
type the following command in place of a Smalltalk expression:
\begin{lstlisting}
    )i test.st
\end{lstlisting}

The i can be thought of as mnemonic for "include." At this point the
class description will be parsed, and, if there are syntax errors, a number
of descriptive error messages will be produced. Ifthere are no syntax errors,
the class definition will be added to the collection of classes known to the
Little Smalltalk system. Whether there are syntax errors or not, the cursor
should advance by one tab stop when the system is ready to accept the
next command.

If there are syntax errors, the class description can be edited without
leaving the Little Smalltalk system. To do this, type the following command:
\begin{lstlisting}
    )e test.st
\end{lstlisting}

The )e command will place the user in an editor mode to view the
designated file\note{On UNIX Systems the editor selected can be changed by modifying the EDITOR
environment variable.}. When the user exits the editor, another attempt will be
made to include the file, as if the )i command were typed.

Once a class definition has been successfully included, the defined class
can be used the same as any other class in the Little SmalltaIk system. For
example, a new instance can be created using the command new.
\begin{lstlisting}
    i~Te5t new
\end{lstlisting}

There are other system commands, similar to )i and k, described in
Appendix 1. These commands are provided with an alternative syntax,
rather than as messages passed to smalltalk, because they are used during
the bootstrapping process before any objects have been defined. The bootstrapping of the Little Smalltalk system will be discussed in the second
section of this book.


\secup

\secly{EXERCISES}

\begin{enumerate}

\item A Bag is similar to a Set; however, each entry may occur any number
of times. One way to implement the class Bag would be to use a
dictionary, similar to the way a List was used to implement the class
Set in Figure 4.3. The value contained in the dictionary for a given
entry would represent the number of times the entry occurs in the bag.
The framework for such an implementation is shown below. Change
the name of the class from Bag to MyBag, and complete the implementation of class Bag.
\begin{lstlisting}
    Class Bag :Colleetion
    I diet count I
    [
    new
    dict~Dictionarynew
    several missing methods
    first
    (count diet first) isNil ifTrue: [ t nil].
    count count - 1.
    t diet currentKey
    next
    [count notNil]whileTrue:
    [(eount>O)
    ifTrue:[eount~eount-1. t diet currentKey]
    ifFaIse: [eou nt~ iet next]].
    t nil
\end{lstlisting}
Keep in mind that instances of Dictionary respond to first and next
with values and not keys. However, the current key is accessible if you
use the message currentKey

\item The collections described in Chapter 3 were all linear, meaning they
could all be represented by a linearly written list. A common nonlinear
data structure in computer science is the binary tree. Implement the
class BinaryTree. Instances of BinaryTree contain a left subtree (or
nil), a right subtree (or nil), and a value. Instances of the class should
respond to the following messages:

\begin{tabular}{l p{8cm}}
    left & Return the left subtree, usually either nil or other instances of BinaryTree. \\
    left: & Set the left subtree to the argument value. \\
    right & Return the right subtree. \\
    right: & Set the right subtree to the argument value. \\
    value & Return the value of the current node. \\
    value: & Set the value of the current node to the argument. \\
\end{tabular}

\noindent
How should instances ofBinaryTree respond to the enumeration messages first and next? How about print or printString? What should be
.the superclass of BinaryTree?

\end{enumerate}

\secrel{A Simple Application}\secdown

This chapter illustrates the development of a simple application in
Smalltalk and describes how environments can be saved and restored.

\bigskip

This chapter depicts the development of a simple Smalltalk application.
The application chosen, a tool to help keep track of employee information
in a small business, is considerably less important than the design techniques being illustrated.

The first, and probably one of the most important steps, is deciding
exactly what you want to do. This involves not only stating the desired
functionality, but also describing the intended user interface (the set of
messages to which the application should respond).

Because programming in Smalitalk is so easy compared to many other
programming languages, a particularly attractive technique of developing
an application is rapid-prototyping. Using this technique, you first define
a minimal system that will still exhibit the important aspects of the desired
functionality. That is, you strive to find the core of the functions you want
at the expense of enhancements or features you might think ultimately
desirable. You then design the simplest, most straightforward implementation of this bare system, ignoring for the moment such features as user
friendliness and efficiency.

Once you have an executable system, you experiment with it. There is
a great psychological benefit to having an executing system, even one that
is very simple. High level logical mistakes are most easily exposed with
the aid of an executing system, and thus a working version helps considerably in getting more complex versions mnning. An important aspect of
software is its feel, a notion nearly impossible to define and almost as
difficult to predict before you have a working system.

The n~cessity for devoting time to a complete and thorough job of
specification and design cannot be understated. However, it is likely that
there will be many changes in the design stage of any realistic piece of
softwar:e. It js also likely (perhaps unfortunately) that the user's concept
of the problem at hand and the correct solution also will change as the
user's experience with the initial versions increases. Users often discover
that the set of enhancements they thought were important before a project
was implemented are not the same as those they want after experimenting
with the initial version. Thus time spent creating a complete system, beyond a minimal system for experimentation purposes may not be productive. (Even with the best planning it may be necessary to throwaway at
least one version, and oftentimes more, and start anew. This is not a sign
of poor programming practice. It is far better to throwaway the first
attempt at a program, after learning from the mistakes, and to rewrite it
than to take a poorly designed and less than adequate program and try to .
fix it by "patching.")

What is the minimal functionality we could desire for our employee
database? At the simplest, we must be able to add and delete information
from the records. But what information? Let us start with four fields:

\medskip
\begin{tabular}{l l}
    name        & a string giving the name of the employee \\
    idNumber    & a unique internal identification number \\
    position    & a symbol representing the employee job classification \\
    salary      & a number giving the salary of the employee \\
\end{tabular}

\medskip
In a realistic situation there would probably be many more fields one
would want to maintain (length of employment, social security number,
department, and so on) but the four fields listed are sufficient for our
examples. For a first approximation, we need not create any new classes
at all. We can merely use an instance of class Dictionary as our database
and an Array for each employee. The key for each dictionary entry can be
the employee number (since they must be distinct, and names may not
be) and the value field can be the rest of the information.

\begin{lstlisting}
    employeeDatabase Dictionary new
    employeeDatabase at: 14737 put: # ( 'David Smith ' # clerk 14000 )
    employeeDatabase at: 16432 put: # ( 'Roger Jonesl # clerk 13500)
    employeeDatabase at: 2431 put: #( 'Fred Brown' # president 68020 )    
\end{lstlisting}

Information on a particular employee can be extracted using at:

\begin{lstlisting}
    employeeDatabase at: 16432
# ( lRoger Jonesl # clerk 13500 )
\end{lstlisting}

Searches of various sorts can be performed by using select:

\begin{lstlisting}
        employeeDatabase select: [:x I (x at: 2) = = # president]
    Dictionary ( 2431 @ #( 'Fred Brown ' president 68020 ) )
        employeeDatabase select: [:x I (x at: 3) < 20000 ]
    Dictionary ( 14737 @ #( 'David Smithl #c1erk 14000 )
        16432 @ # ( 'Roger Jonesl # clerk 13500 ) )
\end{lstlisting}

Output which looks slightly better can be obtained by using do:

\begin{lstlisting}
    employeeDatabase do: [:x I x at: 3) < 20000) ifTrue: [ x print] ]
    #( 'David Smith' #c1erk 14000)
    #( 'Roger Jonesl
    #c1erk 13500)
\end{lstlisting}

A record can be updated by combinations of at: and at:put:

\begin{lstlisting}
    (employeeDatabase at: 16432) at: 3 put: 13750
    employeeDatabase at: 16432
    # ( 'Roger Jonesl
    # clerk 13750)
\end{lstlisting}

While it required almost no work to produce this first approximation,
it is obvious that this scheme has some deficiences. One ofthe most serious
deficiencies is that the user of the database must know the position of each
field in the employee record in order to understand or update the information. One mistake in updating (using the wrong field number, for example) can damage the record badly. Therefore our first improvement will
be to replace each employee record with an instanc~ of a new class, EmployeeRecord. Instances of EmployeeRecord will themselves contain
instance variables for each field of interest. A pair of messages is defined
for each field: one to set the value and one to return it. The following class
description shows one of these pairs.

\begin{lstlisting}
    Class EmployeeRecord
    I name idNumber position salary I
    [
    name: aString
    name aString
    name
    t name
    ...
    ]
\end{lstlisting}

We still use a Dictionary for the entire database but replace the entries
in the dictionary by instances of EmployeeRecord. In a certain sense we
have complicated matters since it is now necessary to initialize each field
separately. Also it is now possible to retrieve only a single field, not the
entire structure, of the employee record.

\begin{lstlisting}
    employeeDatabase Dictionary new
    employeeDatabase at: 2431 put: EmployeeDatabase new
    (employeeDatabase at: 2431) name: IFred Brown'
    (employeeDatabase at: 2431) position: #president
    (employeeDatabase at: 2431) salary: 68020
    employeeDatabase at: 2431
    EmployeeRecord
    (employeeDatabase at: 2431) position
    #president
\end{lstlisting}

Let us deal first with the problem of the difficulty in retrieving all
information at once. The message printString is used uniformly throughout
the Little Sn1alltalk system to produce a printable representation of an
object. If no method is provided for this message, the default method (in
class Object) produces the class name, as illustrated above. We can produce a more meaningful output, however, by concatenating the various
fields with strings showing the field names.

\begin{lstlisting}
    printString
    t Iname: II name,
    I position: I, position,
    Isalary: I I salary
\end{lstlisting}

Now when we try to print an individual record, the result is more helpful

\begin{lstlisting}
    employeeDatabase at: 2431
name: Fred Brown position: president salary: 68020
\end{lstlisting}

The solution to the problem of initialization is slightly more complicated. As a first step, we can create a new message initialize and use the
message getString with the pseudo variable smalltalk to return a string in
response to a prompt. In order for the prompt and the response to appear
on the same line, we use the printing command printNoReturn.

\begin{lstlisting}
    initialize
    Iname: I printNoReturn.
    name <E-- smalltalk getString.
    'position: I printNoReturn.
    position <E-- smalltalk getString asSymbol.
    'salary: I printNoReturn
    salary <E-- smalltalk getString aslnteger
\end{lstlisting}

Thus we can initialize all the fields at the same time with one command:
\begin{lstlisting}
    employeeDatabase <E-- Dictionary new
    employeeDatabase at: 2431 put: EmployeeRecord new
    (employeeDatabase at: 2431) initialize
    name: Fred Brown
    position: president
    salary: 68020
    employeeDatabase at: 2431
    name: Fred Brown position: president salary: 68020
\end{lstlisting}

There is a shortcut in Smalltalk to make the initialization of newly
created objects easier. If the class of an object defines a method for the
message new, then this message is passed to each new instance of that
class as it is created, before the new object is returned to the user. Thus,
we can define the following message in class EmployeeRecord:

\begin{lstlisting}
    new
        self initialize
\end{lstlisting}

The message initialize will then be sent automatically to each new instance
of EmployeeRecord.

\begin{lstlisting}
    employeeDatabase <E-- Dictionary new
    employeeDatabase at: 2431 put: EmployeeRecord new
    name: Fred Brown
    position: president
    salary: 68020
    employeeDatabase at: 2431
    name: Fred Brown position: president salary: 68020
\end{lstlisting}

One can update a record in the same manner as initialization. That is,
to update a record on a specific individual, you merely pass the message
initialize to the record for that-individual and retype the information, making changes where appropriate. We still can use the messages we defined
at the start to update single fields individually. Both of these alternatives
may be error prone. A better scheme would be to display each field as it
is currently contained in the database, then making any changes desired.
As with initialize, the system would prompt for each field, giving the current
value of the field. If the user merely typed return, the current value would·
be retained.

\begin{lstlisting}
    (employeeDatabase at: 2431) upDate
    name (Fred Brown):
    position (president):
    salary (68020): 72000
    employeeDatabase at: 2431
    name: Fred Brown position: president salary: 72000
\end{lstlisting}

\noindent
Here only the salary field was changed.

Note that we are performing the same actions for each field. Thus it
is easier to abstract the desired behavior into a lower level message, passing
messages to self for each individual field. For that lower level message, it
is necessary to print the prompt, including the current value; retrieve the
user's response, and, if blank, return the current value, or, if not blank,
return the user's response. Thus there are two essential pieces of information that must be passed as arguments; namely, the string with which
to prompt and the current value for the field. We can write upDate as
follows:

\begin{lstlisting}
    upDate
    name self promptString: Inamel currentValue: name
    position (self promptString: 'position' currentValue: position) asSymbol
    salary (self promptString: 'salari currentValue: salary) aslnteger
    I
    promptString: aString currentValue: aValue Ireply I
    ( promptString, 1( I, currentValue, I ):1) printNoReturn.
    reply smalltalk getString.
    (reply size = 0)
    ifTrue: [ i currentValue]
    ifFalse: [ t reply]
\end{lstlisting}

Now let us return once more to the representation of the database as
a whole. We have up to this point merely been using an instance of class
Dictionary and the insertion and deletion messages at: and at.·put:. Thus
the abstract actions, namely creating, updating, and listing employee entries, must be phrased (sometimes awkwardly) in terms of messages Dictionary understands. (Recall the messages required to list all employees
with salaries less than 20,000, for example.) A better scheme would be to
create a new class, like EmployeeDatabase, that would respond to messages more appropriate for our application. A set of messages might be
the following:

\begin{lstlisting}
newEmployee
\end{lstlisting}

create a new employee record.

\begin{lstlisting}
    upDate: aNumber
\end{lstlisting}

update the fields for the employee with the given number.

\begin{lstlisting}
    list: aBlock
\end{lstlisting}

print out information on all employees that satisfy the condition given
by the argument block.

A typical session using this class might appear as follows:
\begin{lstlisting}
    employeeDatabase EmployeeDatabase new
    employeeDatabase newEmployee
    id: 2431
    name: Fred Brown
    position: president
    salary: 68020
    employeeDatabase update: 2431
    name (Fred Brown):
    position (president):
    salary (68020): 72000
    employeeDatabase list: [:x Ix position = #president]
    name: Fred Brown position: president salary: 7200
\end{lstlisting}

The new database must still retain the information somewhere. One
way is to maintain an internal dictionary. This can be established at the
time an instance of the database class is created, using the special semantics of the new message.
\begin{lstlisting}
    Class EmployeeDatabase
    Irecords I
    [
    new
    records Dictionary new
    ...
    ]
\end{lstlisting}

The messages upDate: and list: merely pass appropriate commands to
the dictionary.
\begin{lstlisting}
    upDate: idNumber
    (records at: idNumber) upDate
    list: condition
    records do: [:x I(condition value: x) ifTrue: [ x print] ]
\end{lstlisting}

The message new Employee requires an additional prompt to return
the employee number.

\begin{lstlisting}
    newEmployee I newNumber I
    lid: I printNoReturn.
    newNumber smalltalk getString aslnteger.
    records at: newNumber put: EmployeeRecord new
\end{lstlisting}

The exercises propose various further extensions that could be made
to this application

\secrel{Saving Environments}

Updating an employee database is not something you do once and then
never again; rather it must be done periodically, as new employees are
hired or retire. One way to update an employee database is to record the
final values for the database on a slip of paper. Before the start of the next
session you initialize the database by initializing each entry. This may be
somewhat unsatisfactory; slips of paper tend to get lost. Since computers
usually have a much better memory for such things, a good alternative is
to use the Little Smalltalk system to save and restore environments. We
use the term environment to denote the set of all objects accessible at any
one time. The command

\begin{lstlisting}
    )s filename
\end{lstlisting}

\noindent
saves a representation of the current environmerh in the indicated file. I In
response to this command, a message indicating the number of bytes
written to the file will be printed. These files tend to be rather large; too
many of them can clutter your directory.

After saving an environment, you can then continue execution or exit
the Little Smalltalk system by typing control-D. A saved environment can
later be restored by typing the command
\begin{lstlisting}
    )1 filename
\end{lstlisting}

\noindent
This command loads the environment saved in the indicated file\note{The )s and )1 commands do not work on all machines on which the other portions
of the Little Smalltalk system will operate. Check with your system manager, or experiment
yourself, to see if they work on your system.}. Notice
that in doing so, it totally erases the environment that existed before the
)1 command was issued, replacing it with the restored environment. In
response to this command, a message indicating the number of bytes read
in will be produced. This figure should match the figure written after the
)s command. All names that denoted accessible objects before the )s COffi
mand are now valid, and the user can continue as if neither the )s nor the
)1 commands had been issued.

Note that the same environment can be loaded many times. Acommon
use for this feature is for users to save an environment containing their
favorite classes and objects that they have created. This environment can
then be quickly loaded, and the classes will be available without having to
issue )i commands for each one.


\secly{EXERCISES}

\begin{enumerate}

    \item Add messages to delete employees from the database.

    \item Is it necessary to have both the messages initialize and upDate in EmployeeRecord ? Can one be replaced with the other? What changes
    would the user notice?

    \item As it stands, the application is rather lax about error checking. Add
    checks to make sure a new employee record does not override an older
    one and that a request to update specifies a valid employee identification number.

    \item In the scheme described in this chapter, identification numbers are
    different from the other fields in the employee record. They are kept
    oilly as keys in the database and not as part of the record, for example.
    Change the class EmployeeRecord so that it maintains the identification number as part of the record. What changes are then required
    in EntployeeDatabase ?

    \item Instead of maintaining an instance of Dictionary in EmployeeDatabase, one could make it a subclass of Dictionary. How would
    this change the methods for this class? What are the advantages and
    disadvantages of both schemes?

\end{enumerate}

\secup


\secrel{Primitives, Cascades, and Coercions}\secdown

This chapter introduces the syntax for cascaded expressions and describes the notion of primitive expressions. It illustrates the use of
primitives by showing how primitives are used to produce the correct
results for mixed mode arithmetic operations.

\secrel{Cascades}
\secrel{Primitives}
\secrel{Numbers}

\secup

\secrel{A Simulation}\secdown

This chapter presents a simple simulation of an ice cream store, illustrating the ease with which simulations can be described in Smalltalk.

\secrel{The Ice Cream Store Simulation}
\secrel{Further Reading}

\secup

\secrel{Generators}\secdown

This chapter introduces the concept of generators and shows how
generators can be used in the solution of problems requiring goaldirected evaluation.

\secrel{Filters}
\secrel{Goal-Directed Evaluation}
\secrel{Operatkms on Generators}
\secrel{Further Reading}

\secup

\secrel{Graphics}\secdown

Although graphics are not fundamental to Little Smalltalk in the same
way that they are an intrinsic part of the Smalltalk-80 system, it is still
possible to describe some graphics functions using the language. This
chapter details three types of approaches to graphics.

\secrel{Character Graphics}
\secrel{Line Graphics}
\secrel{Bit-Mapped Graphics}

\secup

\secrel{Processes}\secdown

This chapter introduces the concepts of processes and semaphores. It
illustrates these concepts using the dining philosophers problem.

\secrel{Semaphores}
\secrel{Monitors}
\secrel{The Dining Philosophers Problem }
\secrel{Further Reading}

\secup

\secup
