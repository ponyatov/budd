\secly{Preface}
\subsecly{The Little Smalltalk System: Some History}
\subsecly{About A Little Smalltalk}
\subsecly{Acknowledgments}
\subsecly{Obtaining the Little Smalltalk System}

\secrel{The Language}\secdown

\secrel{Basics}\secdown

This chapter introduces the basic concepts of the Smalltalk language;
namely object, method, class, inheritance and overriding.

\secrel{Objects, Classes, and Inheritance}
\secrel{History, Background Reading}
\secup

\secrel{Syntax}\secdown

This chapter introduces the syntax for literal objects (such as numbers)
and the syntax for messages. It explains how to use the Little Smalltalk
system to evaluate expressions typed in directly at the keyboard and
how to use a few simple messages to discover information about different types of objects.

\secrel{Literal Constants}
\secrel{Identifiers}
\secrel{Messages}
\secrel{Getting Started}
\secrel{Finding Out About Objects}
\secrel{Blocks}
\secrel{Comments and Continuations}

\secup

\secrel{Basic Classes}\secdown

The basic classes included in the Little Smalltalk standard library are
explained in this chapter.

\secrel{Basic Objects}
\secrel{Collections}
\secrel{Control Structures}
\secrel{Class Management}
\secrel{Abstract Superclasses}

\secup

\secrel{Class Definition}\secdown

This chapter introduces the syntax used for defining classes. An example class definition is presented.

\secrel{An Illustrative Example}
\secrel{Processing a Class Definition}

\secup

\secrel{A Simple Application}\secdown

This chapter illustrates the development of a simple application in
Smalltalk and describes how environments can be saved and restored.

\secrel{Saving Environments}

\secup

\secrel{Primitives, Cascades, and Coercions}\secdown

This chapter introduces the syntax for cascaded expressions and describes the notion of primitive expressions. It illustrates the use of
primitives by showing how primitives are used to produce the correct
results for mixed mode arithmetic operations.

\secrel{Cascades}
\secrel{Primitives}
\secrel{Numbers}

\secup

\secrel{A Simulation}\secdown

This chapter presents a simple simulation of an ice cream store, illustrating the ease with which simulations can be described in Smalltalk.

\secrel{The Ice Cream Store Simulation}
\secrel{Further Reading}

\secup

\secrel{Generators}\secdown

This chapter introduces the concept of generators and shows how
generators can be used in the solution of problems requiring goaldirected evaluation.

\secrel{Filters}
\secrel{Goal-Directed Evaluation}
\secrel{Operatkms on Generators}
\secrel{Further Reading}

\secup

\secrel{Graphics}\secdown

Although graphics are not fundamental to Little Smalltalk in the same
way that they are an intrinsic part of the Smalltalk-80 system, it is still
possible to describe some graphics functions using the language. This
chapter details three types of approaches to graphics.

\secrel{Character Graphics}
\secrel{Line Graphics}
\secrel{Bit-Mapped Graphics}

\secup

\secrel{Processes}\secdown

This chapter introduces the concepts of processes and semaphores. It
illustrates these concepts using the dining philosophers problem.

\secrel{Semaphores}
\secrel{Monitors}
\secrel{The Dining Philosophers Problem }
\secrel{Further Reading}

\secup

\secup
