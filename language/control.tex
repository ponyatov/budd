\secrel{Control Structures}

One of the more surprising aspects of Smalltalk is the fact that control
structures are not provided as part ofthe basic syntax but rather are defined
using the message passing paradigm. The basic control structure in Smalltalk, as in most computer languages, is the conditional test: IF some condition is satisfied THEN perform some actions ELSE perform some other
actions. In Smalltalk this is accomplished by passing messages to instances
of class Boolean. The class True (a subclass of Boolean) defines methods
for the messages ifTrue: and ifFalse: (similar methods are defined for class
False). The arguments used with these messages are blocks. If the condition is satisfied (Le., the receiver is true and the message is ifTrue:, or
the receiver is false and the message is ifFalse:), the argument block is
evaluated, and the result it produces is returned. If the condition is not
satisfied, the value nil is returned.
\begin{lstlisting}
        (3 < 5) ifTrue: { 17 ]
    17
        (3 < 5) ifFalse: { 17 ]
    nil
\end{lstlisting}

The combined forms ifTrue:ifFalse: and ifFalse:ifTrue: are also recognized:
\begin{lstlisting}
        (3 < 5) ifTrue: { 17 ] ifFalse: { 23 ]
    17
        (3 > 5) ifTrue: { 17 ] ifFalse: { 23 ]
    23
\end{lstlisting}

The message and: and or: are similar to ifTrue: and ifFalse:. They are
also used with booleans and passed as arguments objects.of class Block.
And: i and or: provide "short circuit" evaluation of booleans; that is, the
argument block is evaluated only if necessary to determine the result of
the boolean expression.
\begin{lstlisting}
((i < 10) and: { (b at: i) = 4] ifTrue: { i print]
\end{lstlisting}
In this example, the expression "(b at:i) =4" will be evaluated only if the
expression <10<10)" is true. If the first expression returns false, the argument block used with the and: message is not 
evaluated\note{In actual fact the Little Smalltalk parser will, for efficiency, often optimize conditions
to remove the message passing overhead. Nevertheless, the underlying paradigm holds true
and will, in fact, be used under some conditions (for example, when the arguments are not
a block).}.' Notice that the
relational < returns either true (an instance of class True) or false (an
instance of class False) and that the message and: is implemented in class
Boolean, a superclass of both True and False. Various other boolean
operations, such as not are also defined in this class.

Next to conditional tests, the most common control structure is a loop.
A loop is produced by passing the timesRepeat: message with a parameterless bock as an argument to an integer. The value of the integer is the
number of times to execute the loop. For example;
\begin{lstlisting}
        5 timesRepeat:{ 8 print]
    8
    8
    8
    8
    8
\end{lstlisting}
will print the number 8 five times.

A more general loop is used to produce numbers in arithmetic progression. The messages to: and to:by:, when passed to a number, produce
an instance of class Interval. As we noted in the last section, an interval
is a collection of values in arithmetic progression. We can then use the
message do: to enumerate the elements in the progression. For example:
\begin{lstlisting}
        (2 to: 9 by: 3) do: [ :x I x print]
    2
    5
    8
\end{lstlisting}

A more general form of loop is the while loop. A while loop is formed
using blocks as both receiver and argument. The result of the receiver
block must be a boolean. The actions performed by the block are to evaluate
the receiver block and, as long as it returns true, to evaluate the argument
block. For example, by using an additional variable the previous loop could
have been written:
\begin{lstlisting}
        i~2.
        [ i < = 9 ] whileTrue: [ i print. i i + 3 ]
    2
    5
    8
\end{lstlisting}
Since both the receiver and the argument block can contain any number
of expressions (the value of a block is always the value ofthe last expression,
regardless of how many expressions the block contains), sometimes no
argument block is necessary. The unary message whileTrue (or whileFalse) can then be used. For example:
\begin{lstlisting}
        i~2.
        [i print. i i + 3 . i < = 9] whileTrue.
    2
    5
    8
\end{lstlisting}
