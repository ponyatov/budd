\secrel{Generators}\secdown

This chapter introduces the concept of generators and shows how generators can
be used in the solution of problems requiring goaldirected evaluation.

\bigskip

In Little Smalltalk, the term generator describes any object that represents a
collection of other objects and that responds to the following two messages:
\begin{description}
    \item{first} The response should be an element of the collection, or the
    special value nil if there are no elements in the collection.
    \item{next} The response should be another element in the collection, or nil
    if there are no more elements in the collection.
\end{description}

For example, instances of the standard data structures, such as Array, String,
or Bag, can all be considered to be generators. Instances of Array or String
return the element stored in their first subscript position (if they have at
least one subscript position) in response to the message first. On subsequent
next messages they respond with the remaining elements in order. This
functionality is provided by class ArrayedCollection, using an instance variable
current (Figure 8.1.)

\fig{language/fig_8_1.png}{height=.7\textheight}

\fig{language/fig_8_2.png}{height=\textheight}

Some data structures, such as instances ofthe class Bag, do not possess a
"natural" ordering, and thus the order in which elements are produced in
response to first and next messages is not defined, other than that all elements
are eventually produced and no element is produced more than once.

Notice that nothing is said about how a generator produces the object to be
yielded in response to one of these messages. Some objects, such as instances of
Bag or Array, maintain their collections in memory, and thus the response to
first and next is merely to enumerate their elements. Instances of File are
similar, only the values are retrieved from an external disk as required. Other
generators, such as instances ofInterval, maintain only the information
necessary for generating each new element as required, and that recompute each
new element on demand (Figure 8.2). Indeed, the list of elements represented by
instances of class Random can be considered to be infinite in length, and thus
cannot be stored entirely in memory.

From the point of view of the message passing interface, there is no distinction
between classes that iterate over their elements in memory and classes that
produce new elements on demand. Even in cases where the sequence to be produced
in response to first and next is finite, there may be advantages to computing
elements as needed rather than all at once when the object is defined.

An example will illustrate how generators assist problem solving in Smalltalk.
Consider the problem of producing prime numbers. By definition, a prime number
is a value having only two divisors, itself and 1. A generator for prime numbers
will produce the first prime value (namely 2) when offered the message first,
and successive prime numbers in response to each next message.

If a number n divides a number 111, then the prime factors of n must also divide
nt. Thus, to tell if a number nl is a prime, we need not test all values less
than 111, only those values that are prime. Therefore a simple generator for
primes can be constructed by merely retaining the previously generated primes in
a Set. As each new value is requested, an object representing the last prime
produced is incremented and tested until a value having no factors is found. The
new value is then inserted into the set and returned.

\begin{lstlisting}
    Class Primes
    I prevPrimes lastPrime I
    [
    first
    prevPrimes Set new.
    prevPrimes add: (IastPrime 2).
    l' lastPrime
    next
    [ lastPrime lastPrime + 1.
    self testNumber: lastPrime ] whileFalse.
    prevPrimes add: lastPrime.
    l' lastPrime
    testNumber: n
    prevPrimes do: [:x I(n " "x = 0) ifTrue: [ l' false] ].
    l' true    
\end{lstlisting}

A few simple observations will improve the efficiency ofthis algorithm and will
also illustrate the proper choice of data structures. The loop in the method for
testNumber: haIts and returns as soon as a previous prime is shown to be a
factor of the number under consideration. Two is a factor of exactly one half of
all numbers. Similarly three is a factor of one third of all numbers, and so on.
If we could arrange to test previous primes in numeric order (that is, in the
order in which they were generated) we would on average remove nonprimes much
more quickly than the more or less random order given to us by a Set. The
appropriate data structure for an ordered collection without keys is a List.
Thus we rewrite the algorithm to use a List and the insertion method addLast:,
which adds elements to the end of the list, rather than add:, which would add to
the front of the list. In fact, keeping the previous primes in order allows yet
another improvement in the algorithm since we can terminate the search of
previous primes as soon as a value larger than vn is reached where n is the
value being tested.

\begin{lstlisting}
    Class Primes
I prevPrimes lastPrime I
[
first
prevPrimes List new.
prevPri mes add: (IastPri me 2).
l' lastPrime
next
[ lastPrime lastPrime + 1.
self testNumber: lastPrime l whileFalse.
prevPrimes addLast: lastPrime.
t lastPrime
testNumber: n .
prevPrimes do: [:x I
(x squared> n) ifTrue: [ t true l.
(n "" "" x = 0) ifTrue: [ t false l l
\end{lstlisting}

An obvious problem with both of these prime number generators is that they
require an ever-increasing amount of storage to maintain the list of previous
prime numbers. If you were constructing a long list of prime values, the size of
this storage could easily become a problem. An alternative, which trades
slightly longer computation time for reduced storage, is a recursive generator.
This is analogous to a recursive procedure in programming languages such as
Pascal. The following program does not maintain the list of previous primes but
instead regenerates the list each time a new number is to be tested.

\begin{lstlisting}
    Class Primes
IlastPrime I
[
first
t lastPrime 2
next
[ lastPrime lastPrime + 1.
self testNumber: lastPrime l whileFalse.
t lastPrime
testNumber: n
(Primes new) do: [:x I
(x squared> n) ifTrue: [ t true l.
(n "" "" x = 0) ifTrue: [ i false l l
\end{lstlisting}

You may have noted that the message do: is being passed to an instance of class
Primes, which does not contain a method for this message. The method for do: is
inherited from class Object and is defined in terms of first and next.

\begin{lstlisting}
    do: aBlock Iitem I
item self first.
[ item notNii l whileTrue:
[aBlock value: item. item self next l.
i nil
\end{lstlisting}

The fact that do: is in class Object and therefore provides functionality for
all objects illustrates the peIVasive nature of generators in Little Smalltalk.
Any object can be manipulated as a generator merely by providing methods for the
messages first and next.

\secrel{Filters}

An entirely different program can solve the same task as the prime number
generators described in the last section. It uses another programing technique,
filters, that is frequently useful in conjunction with generators.
Extemally(that is, examining only the messages to which an object responds) a
filter looks just like a generator. Unlike a "true" generator, however, a filter
does not produce new· values in.response to first or next but takes values
produced by a previously defined generator and modifies them or filters but
values.

The class FactorFilter exemplifies some of the essential features of a filter.
Instances of FactorFilter are initialized by giving them a generator and a
specific nonnegative value. In .response to next (the message first is in this
case replaced by the initialization protocol), values from the underlying
generator are requested andretumed, except values for which the given number is
a factor are repressed. Thus the sequence returned by an instance of
FactorFiiteris exactly the same as that given by the underlying generator, with
the exception that values for which the given number is a factor are filtered
out.

\begin{lstlisting}
    Class FactorFilter
I myFactor generator I
[
remove: factorValue from: generatorValue
myFactor factorValue.
generator generatorValue
next I possible I
[ (possible generator next) notNil ]
whileTrue:
[ (possible" " myFactor -=0)
ifTrue: [ t possible]].
i nil
\end{lstlisting}

Using FactorFilter, you can construct a simple generator for prime numbers.
First an instance of Interval that will generate all numbers from 2 to some
fixed limit is constructed. As each value is removed, a filter is inserted in
front of the generator to insure that all subsequent multiples of the value will
be eliminated. A new value is then requested from the updated generator.

\begin{lstlisting}
    Class Primes
    I primeGenerator lastFactor I
    [
    first
    primeGenerator 2 to: 100.
    lastFactor primeGenerator first.
    i lastFactor
    next
    primeGenerator (FactorFilter new;
    remove: lastFactor
    from: primeGenerator).
    i lastFactor primeGenerator next    
\end{lstlisting}

Pictorially, the underlying generator constructed by the first occurrence of the
message next can be viewed as follows:

\fig{language/filter_1.png}{height=.2\textheight}

When asked for the next prime, the generator is modified by adding a filter,
this time for the last prime value returned, the number 3.

\fig{language/filter_2.png}{height=.2\textheight}

The program continues. Each time a new prime is requested, a filter is
constructed to remove all factors of the previous prime. in this fashion, all
the primes are eventually generated.

\fig{language/filter_3.png}{width=\textwidth}

Of course, like the first two programs in the last section, the storage required
for the chain of filters is proportional to the number of primes generated so
far. Despite this, timings of actual programs show that the filter program is
the fastest of the prime number generating programs described in this chapter.

\secrel{Goal-Directed Evaluation}

A useful programming technique when used in conjunction with generators is
goal-directed evaluation. By this technique, a generator is repeatedly queried
for values until some condition is satisfied. In a certain sense the notion of
filters we have just described represents a simple form of goaldirected
evaluation. The goal of instances of FactorFilter, for example, is to find a
value from the underlying generator for which the given number is not a factor.
In the more general case of goal directed evaluation, the condition frequently
involves the outcome of several generators acting together. An example will
illustrate this Il1ethod.

Consider the problem of placing eight queens on a chess board in such a way that
no queen can attack any other queen (Figure 8.3). In this section we will
describe how such a problem can be formulated and solved using generators,
filters, and goal directed evaluation.

We first observe that in any solution, no two queens can occupy the same column,
and that no column can be empty. We can therefore assign a specific column to
each queen at the start, and reduce the problem to finding a correct row
assignment for each of the eight queens.

In general terms, our approach will be to place queens from left to right (the
order in which we assign numbers to columns). An acceptable solution for column
n is one in which no queen in columns 1 through n can attack any other queen in
those columns. Once we have found an acceptable solution in column 8 we are
finished. Before that, however, we can formulate the problem of finding an
acceptable solution in column n recursively, as follows:

\begin{enumerate}
    \item Find an acceptable solution for column n - 1. If there is none, return
    nil, there is no acceptable solution. Otherwise, place the queen for column
    n in row 1. Go to step 2.
    \item Test to see if any queen in columns 1 through n - 1 can attack the
    queen in column n. If not, then an acceptable solution has been found. if
    the queen can be attacked, then go to step 3.
    \item If the queen for column n is in row 8, go to step 4, otherwise advance
    the queen by one row and go back to step 2.
    \item Find the next acceptable solution for column n - 1. If there is none,
    return nil, otherwise, place the queen for column n in row 1 and go to step
    2.
\end{enumerate}

\fig{language/fig_8_3.png}{height=.7\textheight}

\secrel{Operatkms on Generators}

\secrel{Further Reading}


\secup
