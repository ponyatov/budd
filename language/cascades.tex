\secrel{Cascades}

Suppose you wanted to send a number of messages to the same object. For example,
suppose you wanted to create a new instance of class Bag and to initialize it
with the values 0 and 1. One way to do this would be to use a temporary variable
as in:
\begin{lstlisting}
i~Bag new
i add: 0
i add: 1
\end{lstlisting}

Another way is to use a cascade. A cascade is a type of Smalltalk
expression\note{Note that although similar concepts appear in both the Little
Smalltalk system and the Xerox Smalltalk-80 programming environment, the syntax
and meaning of cascades and primitives are different in the two systems.
Appendix 5 describes how cascades and primitives are handled in the Xerox
system.}. It is written as an expression followed by a semicolon followed by
another expression without a receiver. When a cascade is evaluated, the first
expression is computed, and then used as the receiver for the second expression.
The result of a cascade is not the result ofthe second expression but the result
ofthe first expression. In other words, the result ofthe second expression is
ignored and discarded. Since a cascade is an expression, it can be used to form
another cascade, and so on, to any desired level. The result of a cascade of any
depth is always the result of the first expression. Thus, for example, you can
create a new Bag and initialize it all with one expression, as in:
\begin{lstlisting}
Bag new; add: 0 ; add: 1
\end{lstlisting}
The receiver for both the messages add: a and add: 1 is the result of the
expression Bag new, as is the result of the entire expression.

There are several advantages to using cascaded expressions. Since
there is no necessity to name the intermediate object, a cascade obviates
the need for a great many temporary variables. Also, since the entire sequence is an expression, it can be used anywhere expressions are legal,
for example, as an argument. Finally a cascade can often be used in place
of parentheses to separate two keyword messages or to apply one message
to another of lower precedence. For example, to print the result yielded
by a keyword message you could write:
\begin{lstlisting}
Integer respondsTo: #+ ; print
\end{lstlisting}
in place of
\begin{lstlisting}
(Integer respondsTo: # +)print
\end{lstlisting}
