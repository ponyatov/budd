\secrel{An Illustrative Example}

Figure 4.3 gives the complete class description of the class Set. A set is a
collection of objects; each object occurs no more than once in the collection. (See Exercise 2.) The data structure used to implement the set will
be a list. Recall that a list is also a collection but one that maintains values
in order.

\fig{language/fig_4_3_a.png}{width=.7\textwidth}

\fig{language/fig_4_3_b.png}{width=.7\textwidth}

As we have already seen, the class Set maintains one instance variable,
list. Each instance of class Set will have a separate instance variable that
cannot be accessed by any other object. The method for message new will
automatically create an instance of class List in the variable list whenever
an instance of class Set is created.

When an element is added to a set, a test is first performed to see if
the element is already in the set. If so, no further processing is done; if
not, the element is added to the underlying Jist. Similarly, to remove an
element, the actions in the method in Set merely pass the removal message
on to the underlying list. Note that the long form of the remove: message
is defined; it includes a second argument, indicating what actions should
be taken if the given item is not found. As we saw in Figure 4.2, the class
Collection defines the short message (re1110ve:) in terms of this longer
message. Thus either form ofthe message can be used on a Set.

The messages first and next are used to produce generators. Generators
will be the topic of a later chapter; it is sufficient to say here that first and
next provide a means to enumerate all elements in a collection. The message first can be interpreted informally as flinitialize the enumeration of
items in the collection and return the first item, or nil if there are no items
in the collection." Similarly, next can be defined as "return the next item
in the collection, or nil if there are no remaining items." The protocol for
do: (Figure 4.2) illustrates how these messages can be used to construct a
loop to access every item in a collection. In Class Set, the generation of
items in response to these messages is provided by passing the same messages to the underlying list.
