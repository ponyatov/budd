\secrel{A Simple Application}\secdown

This chapter illustrates the development of a simple application in
Smalltalk and describes how environments can be saved and restored.

\bigskip

This chapter depicts the development of a simple Smalltalk application.
The application chosen, a tool to help keep track of employee information
in a small business, is considerably less important than the design techniques being illustrated.

The first, and probably one of the most important steps, is deciding
exactly what you want to do. This involves not only stating the desired
functionality, but also describing the intended user interface (the set of
messages to which the application should respond).

Because programming in Smalitalk is so easy compared to many other
programming languages, a particularly attractive technique of developing
an application is rapid-prototyping. Using this technique, you first define
a minimal system that will still exhibit the important aspects of the desired
functionality. That is, you strive to find the core of the functions you want
at the expense of enhancements or features you might think ultimately
desirable. You then design the simplest, most straightforward implementation of this bare system, ignoring for the moment such features as user
friendliness and efficiency.

Once you have an executable system, you experiment with it. There is
a great psychological benefit to having an executing system, even one that
is very simple. High level logical mistakes are most easily exposed with
the aid of an executing system, and thus a working version helps considerably in getting more complex versions mnning. An important aspect of
software is its feel, a notion nearly impossible to define and almost as
difficult to predict before you have a working system.

The n~cessity for devoting time to a complete and thorough job of
specification and design cannot be understated. However, it is likely that
there will be many changes in the design stage of any realistic piece of
softwar:e. It js also likely (perhaps unfortunately) that the user's concept
of the problem at hand and the correct solution also will change as the
user's experience with the initial versions increases. Users often discover
that the set of enhancements they thought were important before a project
was implemented are not the same as those they want after experimenting
with the initial version. Thus time spent creating a complete system, beyond a minimal system for experimentation purposes may not be productive. (Even with the best planning it may be necessary to throwaway at
least one version, and oftentimes more, and start anew. This is not a sign
of poor programming practice. It is far better to throwaway the first
attempt at a program, after learning from the mistakes, and to rewrite it
than to take a poorly designed and less than adequate program and try to .
fix it by "patching.")

What is the minimal functionality we could desire for our employee
database? At the simplest, we must be able to add and delete information
from the records. But what information? Let us start with four fields:

\medskip
\begin{tabular}{l l}
    name        & a string giving the name of the employee \\
    idNumber    & a unique internal identification number \\
    position    & a symbol representing the employee job classification \\
    salary      & a number giving the salary of the employee \\
\end{tabular}

\medskip
In a realistic situation there would probably be many more fields one
would want to maintain (length of employment, social security number,
department, and so on) but the four fields listed are sufficient for our
examples. For a first approximation, we need not create any new classes
at all. We can merely use an instance of class Dictionary as our database
and an Array for each employee. The key for each dictionary entry can be
the employee number (since they must be distinct, and names may not
be) and the value field can be the rest of the information.

\begin{lstlisting}
    employeeDatabase Dictionary new
    employeeDatabase at: 14737 put: # ( 'David Smith ' # clerk 14000 )
    employeeDatabase at: 16432 put: # ( 'Roger Jonesl # clerk 13500)
    employeeDatabase at: 2431 put: #( 'Fred Brown' # president 68020 )    
\end{lstlisting}

Information on a particular employee can be extracted using at:

\begin{lstlisting}
    employeeDatabase at: 16432
# ( lRoger Jonesl # clerk 13500 )
\end{lstlisting}

Searches of various sorts can be performed by using select:

\begin{lstlisting}
        employeeDatabase select: [:x I (x at: 2) = = # president]
    Dictionary ( 2431 @ #( 'Fred Brown ' president 68020 ) )
        employeeDatabase select: [:x I (x at: 3) < 20000 ]
    Dictionary ( 14737 @ #( 'David Smithl #c1erk 14000 )
        16432 @ # ( 'Roger Jonesl # clerk 13500 ) )
\end{lstlisting}

Output which looks slightly better can be obtained by using do:

\begin{lstlisting}
    employeeDatabase do: [:x I x at: 3) < 20000) ifTrue: [ x print] ]
    #( 'David Smith' #c1erk 14000)
    #( 'Roger Jonesl
    #c1erk 13500)
\end{lstlisting}

A record can be updated by combinations of at: and at:put:

\begin{lstlisting}
    (employeeDatabase at: 16432) at: 3 put: 13750
    employeeDatabase at: 16432
    # ( 'Roger Jonesl
    # clerk 13750)
\end{lstlisting}

While it required almost no work to produce this first approximation,
it is obvious that this scheme has some deficiences. One ofthe most serious
deficiencies is that the user of the database must know the position of each
field in the employee record in order to understand or update the information. One mistake in updating (using the wrong field number, for example) can damage the record badly. Therefore our first improvement will
be to replace each employee record with an instanc~ of a new class, EmployeeRecord. Instances of EmployeeRecord will themselves contain
instance variables for each field of interest. A pair of messages is defined
for each field: one to set the value and one to return it. The following class
description shows one of these pairs.

\begin{lstlisting}
    Class EmployeeRecord
    I name idNumber position salary I
    [
    name: aString
    name aString
    name
    t name
    ...
    ]
\end{lstlisting}

We still use a Dictionary for the entire database but replace the entries
in the dictionary by instances of EmployeeRecord. In a certain sense we
have complicated matters since it is now necessary to initialize each field
separately. Also it is now possible to retrieve only a single field, not the
entire structure, of the employee record.

\begin{lstlisting}
    employeeDatabase Dictionary new
    employeeDatabase at: 2431 put: EmployeeDatabase new
    (employeeDatabase at: 2431) name: IFred Brown'
    (employeeDatabase at: 2431) position: #president
    (employeeDatabase at: 2431) salary: 68020
    employeeDatabase at: 2431
    EmployeeRecord
    (employeeDatabase at: 2431) position
    #president
\end{lstlisting}

Let us deal first with the problem of the difficulty in retrieving all
information at once. The message printString is used uniformly throughout
the Little Sn1alltalk system to produce a printable representation of an
object. If no method is provided for this message, the default method (in
class Object) produces the class name, as illustrated above. We can produce a more meaningful output, however, by concatenating the various
fields with strings showing the field names.

\begin{lstlisting}
    printString
    t Iname: II name,
    I position: I, position,
    Isalary: I I salary
\end{lstlisting}

Now when we try to print an individual record, the result is more helpful

\begin{lstlisting}
    employeeDatabase at: 2431
name: Fred Brown position: president salary: 68020
\end{lstlisting}

The solution to the problem of initialization is slightly more complicated. As a first step, we can create a new message initialize and use the
message getString with the pseudo variable smalltalk to return a string in
response to a prompt. In order for the prompt and the response to appear
on the same line, we use the printing command printNoReturn.

\begin{lstlisting}
    initialize
    Iname: I printNoReturn.
    name <E-- smalltalk getString.
    'position: I printNoReturn.
    position <E-- smalltalk getString asSymbol.
    'salary: I printNoReturn
    salary <E-- smalltalk getString aslnteger
\end{lstlisting}

Thus we can initialize all the fields at the same time with one command:
\begin{lstlisting}
    employeeDatabase <E-- Dictionary new
    employeeDatabase at: 2431 put: EmployeeRecord new
    (employeeDatabase at: 2431) initialize
    name: Fred Brown
    position: president
    salary: 68020
    employeeDatabase at: 2431
    name: Fred Brown position: president salary: 68020
\end{lstlisting}

There is a shortcut in Smalltalk to make the initialization of newly
created objects easier. If the class of an object defines a method for the
message new, then this message is passed to each new instance of that
class as it is created, before the new object is returned to the user. Thus,
we can define the following message in class EmployeeRecord:

\begin{lstlisting}
    new
        self initialize
\end{lstlisting}

The message initialize will then be sent automatically to each new instance
of EmployeeRecord.

\begin{lstlisting}
    employeeDatabase <E-- Dictionary new
    employeeDatabase at: 2431 put: EmployeeRecord new
    name: Fred Brown
    position: president
    salary: 68020
    employeeDatabase at: 2431
    name: Fred Brown position: president salary: 68020
\end{lstlisting}

One can update a record in the same manner as initialization. That is,
to update a record on a specific individual, you merely pass the message
initialize to the record for that-individual and retype the information, making changes where appropriate. We still can use the messages we defined
at the start to update single fields individually. Both of these alternatives
may be error prone. A better scheme would be to display each field as it
is currently contained in the database, then making any changes desired.
As with initialize, the system would prompt for each field, giving the current
value of the field. If the user merely typed return, the current value would·
be retained.

\begin{lstlisting}
    (employeeDatabase at: 2431) upDate
    name (Fred Brown):
    position (president):
    salary (68020): 72000
    employeeDatabase at: 2431
    name: Fred Brown position: president salary: 72000
\end{lstlisting}

\noindent
Here only the salary field was changed.

Note that we are performing the same actions for each field. Thus it
is easier to abstract the desired behavior into a lower level message, passing
messages to self for each individual field. For that lower level message, it
is necessary to print the prompt, including the current value; retrieve the
user's response, and, if blank, return the current value, or, if not blank,
return the user's response. Thus there are two essential pieces of information that must be passed as arguments; namely, the string with which
to prompt and the current value for the field. We can write upDate as
follows:

\begin{lstlisting}
    upDate
    name self promptString: Inamel currentValue: name
    position (self promptString: 'position' currentValue: position) asSymbol
    salary (self promptString: 'salari currentValue: salary) aslnteger
    I
    promptString: aString currentValue: aValue Ireply I
    ( promptString, 1( I, currentValue, I ):1) printNoReturn.
    reply smalltalk getString.
    (reply size = 0)
    ifTrue: [ i currentValue]
    ifFalse: [ t reply]
\end{lstlisting}

Now let us return once more to the representation of the database as
a whole. We have up to this point merely been using an instance of class
Dictionary and the insertion and deletion messages at: and at.·put:. Thus
the abstract actions, namely creating, updating, and listing employee entries, must be phrased (sometimes awkwardly) in terms of messages Dictionary understands. (Recall the messages required to list all employees
with salaries less than 20,000, for example.) A better scheme would be to
create a new class, like EmployeeDatabase, that would respond to messages more appropriate for our application. A set of messages might be
the following:

\begin{lstlisting}
newEmployee
\end{lstlisting}

create a new employee record.

\begin{lstlisting}
    upDate: aNumber
\end{lstlisting}

update the fields for the employee with the given number.

\begin{lstlisting}
    list: aBlock
\end{lstlisting}

print out information on all employees that satisfy the condition given
by the argument block.

A typical session using this class might appear as follows:
\begin{lstlisting}
    employeeDatabase EmployeeDatabase new
    employeeDatabase newEmployee
    id: 2431
    name: Fred Brown
    position: president
    salary: 68020
    employeeDatabase update: 2431
    name (Fred Brown):
    position (president):
    salary (68020): 72000
    employeeDatabase list: [:x Ix position = #president]
    name: Fred Brown position: president salary: 7200
\end{lstlisting}

The new database must still retain the information somewhere. One
way is to maintain an internal dictionary. This can be established at the
time an instance of the database class is created, using the special semantics of the new message.
\begin{lstlisting}
    Class EmployeeDatabase
    Irecords I
    [
    new
    records Dictionary new
    ...
    ]
\end{lstlisting}

The messages upDate: and list: merely pass appropriate commands to
the dictionary.
\begin{lstlisting}
    upDate: idNumber
    (records at: idNumber) upDate
    list: condition
    records do: [:x I(condition value: x) ifTrue: [ x print] ]
\end{lstlisting}

The message new Employee requires an additional prompt to return
the employee number.

\begin{lstlisting}
    newEmployee I newNumber I
    lid: I printNoReturn.
    newNumber smalltalk getString aslnteger.
    records at: newNumber put: EmployeeRecord new
\end{lstlisting}

The exercises propose various further extensions that could be made
to this application

\secrel{Saving Environments}

Updating an employee database is not something you do once and then
never again; rather it must be done periodically, as new employees are
hired or retire. One way to update an employee database is to record the
final values for the database on a slip of paper. Before the start of the next
session you initialize the database by initializing each entry. This may be
somewhat unsatisfactory; slips of paper tend to get lost. Since computers
usually have a much better memory for such things, a good alternative is
to use the Little Smalltalk system to save and restore environments. We
use the term environment to denote the set of all objects accessible at any
one time. The command

\begin{lstlisting}
    )s filename
\end{lstlisting}

\noindent
saves a representation of the current environmerh in the indicated file. I In
response to this command, a message indicating the number of bytes
written to the file will be printed. These files tend to be rather large; too
many of them can clutter your directory.

After saving an environment, you can then continue execution or exit
the Little Smalltalk system by typing control-D. A saved environment can
later be restored by typing the command
\begin{lstlisting}
    )1 filename
\end{lstlisting}

\noindent
This command loads the environment saved in the indicated file\note{The )s and )1 commands do not work on all machines on which the other portions
of the Little Smalltalk system will operate. Check with your system manager, or experiment
yourself, to see if they work on your system.}. Notice
that in doing so, it totally erases the environment that existed before the
)1 command was issued, replacing it with the restored environment. In
response to this command, a message indicating the number of bytes read
in will be produced. This figure should match the figure written after the
)s command. All names that denoted accessible objects before the )s COffi
mand are now valid, and the user can continue as if neither the )s nor the
)1 commands had been issued.

Note that the same environment can be loaded many times. Acommon
use for this feature is for users to save an environment containing their
favorite classes and objects that they have created. This environment can
then be quickly loaded, and the classes will be available without having to
issue )i commands for each one.


\secly{EXERCISES}

\begin{enumerate}

    \item Add messages to delete employees from the database.

    \item Is it necessary to have both the messages initialize and upDate in EmployeeRecord ? Can one be replaced with the other? What changes
    would the user notice?

    \item As it stands, the application is rather lax about error checking. Add
    checks to make sure a new employee record does not override an older
    one and that a request to update specifies a valid employee identification number.

    \item In the scheme described in this chapter, identification numbers are
    different from the other fields in the employee record. They are kept
    oilly as keys in the database and not as part of the record, for example.
    Change the class EmployeeRecord so that it maintains the identification number as part of the record. What changes are then required
    in EntployeeDatabase ?

    \item Instead of maintaining an instance of Dictionary in EmployeeDatabase, one could make it a subclass of Dictionary. How would
    this change the methods for this class? What are the advantages and
    disadvantages of both schemes?

\end{enumerate}

\secup
