\secrel{A Simulation}\secdown

This chapter presents a simple simulation of an ice cream store, illustrating
the ease with which simulations can be described in Smalltalk.

\bigskip

The programming paradigm employed in Smalltalk, that of a large number of
independent objects communicating via message passing, is particularly suitable
for constructing simulations of processes that can be described as the
interaction of a finite number of events. Each object can represent some object
in the simulation model, with the local memory of the object maintaining
information about the state of the model object. Messages correspond to
interaction between the simulation objects. Indeed, the language was developed
largely with just such applications in mind. This chapter will illustrate the
construction of a simple simulation describing the operation of an ice cream
store. The reader interested in more extensive simulation techniques can consult
some of the references listed at the end of this chapter.

\secrel{The Ice Cream Store Simulation}

In writing a simulation, such as our simulation of an ice cream store, the first
question to ask is what the different objects in the simulation should represent
and what functionality they should possess. At its simplest, our simulation must
involve at least two classes of objects: a class representing the actions of the
various customers that come to the store and a class representing the actions of
the store itself in response to the requests of the customers.

The application we will describe is known as a discrete, event-driven
simulation. This means that the actions of the simulation will revolve around
the recording and processing of a finite number of individual events. An event
is simply the occurrence of some action that is important in the context of the
simulation. Each event is marked with a time at which the event should take
place. A "clock" records the current "time," which need not be related in any
way to conventional time; a simple counter will suffice. As time progresses, the
occurrence of each event controls the sequence of further actions in the
simulation.

Our first, and very simple, attempt at simulating an ice cream store will
illustrate these concepts. A pending event is defined as any event that has. not
yet taken place. Assume, first, that there is never more than one pending event.
At the moment we need not concern ourselves with how each event is to be
encoded, but only that it can be represented by an object. We can isolate the
actions of the simulation that are independent of any particular application in
an abstract superclass called Simulation. Particular simulations, such as in the
ice cream store, will then be subclasses of Simulation.

The first version of the class Simulation is shown in Figure 7.1. The class
maintains the current "time" in the variable currentTime. The message proceed
will indicate that the next action should take place. Since the actual
interpretation of an event will differ from one simulation to the next, proceed
leaves the task of interpreting the event to a subclass by passing the message
processEvent: to the pseudo variable self. The message processEvent: must
therefore be recognized in each subclass.

\fig{language/fig_7_1.png}{height=\textheight}

Having described the framework for keeping track of the simulation
"bookkeeping," we can go on to describe the actions of our ice cream store. We
start with a simple model in which customers arrive, order some number of scoops
of ice cream, and leave. Each scoop of ice cream produces some amount of profit
for the store. At the end of the simulation period we will want to know the
profit.

It is, however, not sufficient to merely describe what actions take place in the
store; we must also describe how often those events should take place. For
example, how often do customers arrive, how many scoops of ice cream will each
customer order, and so on. We could make a simple assumption, such as a new
customer will arrive every two minutes and order three scoops of ice cream. Such
a deterministic assumption, however, defeats the necessity for the simulation at
all since we can easily predict the outcome: in 15 minutes seven customers will
arrive and order 21 scoops of ice cream. Instead, it is more interesting (to say
nothing of being more realistic) to define the outcome of an event in terms of a
random value chosen from a selected distribution.

For example, let us suppose that at each instance the next customer will appear
at a time uniformly chosen from the numbers 1 to 5. The phrase "uniformly
chosen" means that each outcome (or each time) is equally likely, and thus we
can use a random number generator to select a value in this range to represent
the time for the next arrival. Letting IceCreamStore be the class representing
our simulation of the store, this could be computed as shown in the method for
the message scheduleArrival in Figure 7.2. Note that the instance variable rand
is initialized to be an instance of the random number generator, and that during
initialization the first customer is scheduled. The message randomize is used to
insure that a new random sequence is generated each time we invoke the
simulation.

\fig{language/fig_7_2.png}{height=\textheight}

Notice we have chosen to represent events by an instance of class Customer. In
our first simulation very little functionality is required of the customer. Each
customer must arrive and upon arrival decide the number of scoops of ice cream
he or she will order. Thus the class shown in Figure 7.3 suffices for customers.

The only remaining part of our simulation is deciding how the class
IceCreamStore should respond to the message processEvent:. Since the event is an
instance of class Customer, all that is required is determining the amount of
ice cream to be dispensed (and thus the profit to be made) and scheduling the
next customer. The latter requirement may not be obvious, but is important. For
our simulation to work, each event must generate some number of other events.
The decision about where and when each event should schedule the next is not
always obvious, but must be made someplace.

\begin{lstlisting}
    Class Customer
    Irand I
    [
    new
    rand Random new.
    rand randomize
    numberOfScoops I number I
    number rand randlnteger: 3.
    ('customer has " number I I scoops I) print.
    l' number    
\end{lstlisting}

Let us assume that each scoop of ice cream generates seventeen cents profit for
the store. The response to the message processEvent: can then be represented as
follows:
\begin{lstlisting}
    processEvent: event
    ('customer received at I, self time) print.
    profit profit + (event numberOfScoops * 0.17 ).
    self scheduleArrival    
\end{lstlisting}
To determine how much profit might be produced in 15 minutes, we could then run
our simulation as follows:
\begin{lstlisting}
    store IceCreamStore new
    [store time < 15] whileTrue: [store proceed]
    customer received at 1
    customer has 1 scoops
    customer received at 2
    customer has 3 scoops
    customer received at 5
    customer has 1 scoops
    customer received at 8
    customer has 1 scoops
    customer received at 11
    customer has 1 scoops
    customer received at 16
    customer has 2 scoops
    store reportProfits
    profits are 1.53    
\end{lstlisting}

Now, having produced our first simulation, let us go back and examine some of
the assumptions we made to decide if they are really accurate. Do customers
really arrive with a uniform distribution? Do they always arrive individually?
Do they really order some number of scoops with a uniform probability?

Let us tackle the last question first. Suppose we observe a real ice cream store
for some period of time and note that 65\% of the time customers will order one
scoop, 25\% of the time they order two scoops, and only 10\% of the time do they
order three scoops. This is certainly a far different distribution from the one
given by our assumption that all three outcomes are equally likely. In order to
simulate this behavior, we must generate a random integer that returns one 65%
of the time, two 25\% of the time, and three 10\% of the time. A distribution of
values such as we have described is known as a weighted discrete probability.
One way to generate a random value satisfying our requirements is to generate a
uniform random number between 1 and 100. Ifthis uniform value is less than or
equal to 65, we return 1; if less than or equal to 90, we return 2; otherwise,
we return 3. We can generalize this to work for any weighted distribution we
like, producing the class DiscreteProbability shown in Figure 7.4. In response
to the message next, an instance of this class will return a value between 1 and
the size of the weights array, distributed according to the weights given.

\fig{language/fig_7_3.png}{height=\textheight}

The distribution of ice cream orders was obtained by obsenring a large body of
customers. So we can argue whether the number of scoops an individual will order
should be part of the protocolfor the Customer class (since the customer is
issuing the order) or for the IceCreamStore class (since the distribution is
taken from observations of ice cream store customers as a group). The last
simulation illustrated the first Variation. The following simulation will
illustrate the second variation

Let us alter the assumption that customers arrive one by one; since it is a
social process, people tend to eat ice cream in groups. Each instance of class
Customer will be changed, therefore, to represent a group of individuals. Upon
creation, each instance will determine its group size, which will thereafter be
returned in response to a request via the message groupSize. Given these
changes, our second simulation can be given as shown in Figure 7.5

Suppose we complicate things now by adding an inside dining area to our ice
cream store. There are several more factors to consider. Whereas formerly we
assumed we could accommodate as many customers as would arrive in any particular
time period, now we can only accommodate those customers who can find chairs. In
this new situation, the sequence of events affecting a single customer or a
group of customers is now more complicated than the previous case, where the
only event of importance from the customer's point ofview was the receipt ofthe
ice cream. Now the following sequence will take place:
\begin{enumerate}
    \item Some group of customers will arrive. If there are not enough chairs,
    they will immediately leave. Otherwise, they will take seats and start to
    look at the menu.
    \item Later the group of customers will place an order and receive their ice
    cream.
    \item Still later, the group will have finished their ice cream and will
    leave.
\end{enumerate}

Furthermore. we can now have several events happening simultaneously. One group
of customers can be eating their ice cream, while another is ordering, and a
third is just arriving. Thus, the basic assumption that there is always just one
pending event is no longer valid. Now the class Simulation must be altered to
keep a queue of pending events. In response to the message proceed, the event
with the next smallest time marker is removed from the queue and initiated. One
convenient data structure to maintain both the pending events and their time for
initiation is a dictionary using the time as a key and the event as the value.
However, two events can happen at the same time (for instance, one group can
order while another arrives). Therefore the value of the dictionary cannot be
simply a single event, but must be a set of events. Our revised class Simulation
is shown in Figure 7.6.

Events are now more complicated. We must .remember not only a group of customers
but also what state they are in: whether they have just arrived, are about to
order, or are about to leave. One way to do this is to store as an event a block
which when evaluated will move the customer to the next state. Recall that a
block evaluates in the context in which it IS defined and not until passed the
message value. Thus, for example, the protocol for scheduleArrival might be
given as follows:
\begin{lstlisting}
    scheduleArrival I newCustomer I
newCustomer Customer new.
self addEvent: [self customerArrival: newCustomer]
at: (self time + (rand randlnteger: 5
\end{lstlisting}

\begin{lstlisting}
    Class IceCreamStore :Simulation
    I profit rand scoopDistribution I
    [
    new
    profit O.
    rand Random new.
    scoopDistribution DiscreteProbability new
    scoopDistribution defineWeights: #(65 25 10).
    self scheduleArrival
    scheduleArrivaI
    self addEvent: Customer new
    at: (self time + (rand randlnteger: 5
    processEvent: event
    Ccustomer received at I, self time) print.
    profit profit + selfscoopsFor: event groupSize) * 0.17).
    self scheduleArrival
    scoopsFor: group I number I
    number O.
    group timesRepeat:
    [number number + scoopDistribution next].
    ('group of I, group, I have I I number I I scoops I) print.
    t number
    
    reportProfits
('profits are I, profit) print
Class Customer
1 groupSize I
[
new
groupSize <E- (Random new randomize) randlnteger: 8
groupSize
t groupSize
\end{lstlisting}

A new customer is created and placed in the temporary variable new Customer. A
block is then installed in the event queue. Since this block references the
temporary variable newCustomer, the temporary will be retained (i.e., the
storage it uses ·will not be reclaimed) as long as the block exists. However,
each time the message scheduleArrival is received, a new instance of Customer
will be created. The processing indicated by the block will not take place until
the block is evaluated using the message value. This takes place when the event
is processed:
\begin{lstlisting}
    processEvent: event
event value.
self scheduleArrival
\end{lstlisting}

\begin{lstlisting}
    Class Simulation
I currentTime eventQueue I
[
new
eventQueue Dictionary new.
currentTime 0
time
t currentTime
addEvent: event at: eventTime

(eventQueue inciudesKey: eventTime)
ifTrue: [(eventQueue at: eventTime) add: event]
ifFalse: [eventQueue at: eventTime
put: (Set new; add: event)]
addEvent: event next: timelncrement
self addEvent: event at: currentTime + timelncrement
proceed I minTime eventset event I
minTime ~99999.
eventQueue keysDo:
[:x I(x < minTime) ifTrue: [minTime xl].
currentTime minTime.
eventset eventQueue at: minTime ifAbsent: [t nil].
event eventset first.
eventset remove: event.
(eventset isEmpty) ifTrue: [eventQueue removeKey: minTime].
self processEvent: event
\end{lstlisting}

When the block created in scheduleArrival is evaluated, it will send the message
customerArrival: to the pseudo variable self, that is, to the simulation object.
Upon arrival, if there are a sufficient number of chairs, the customers sit down
and order, otherwise they leave. Let us use an instance variable remainingChairs
to represent the number of free chairs at any point. The time between arriving
and ordering will be a random value from one to three. The protocol for
customerArrival: can then be given as follows:

\begin{lstlisting}
    customerArrival: customer I size I
size customer groupSize.
('group of size 1, size, I arrives') print.
(size < remainingChairs)
ifTrue: [remainingChairs remainingChairs - size.
'take chairs, schedules order' print.
self addEvent: [self customerOrder: customer]
next: (rand rand Integer: 3).
]
ifFalse: ['finds no chairs, leaves' print]
\end{lstlisting}

Notice that again a block has been used to represent the next event. When
evaluated, this block will pass the message custonlerOrder:. The protocol for
this message is as follows:

\begin{lstlisting}
    customerOrder: customer I size numScoops I
size customer groupSize.
numScoops O.
size timesRepeat:
[numScoops numScoops + scoopDistribution next].
('group of size I, size, I orders I , numScoops , I SCOOpSl) print.
profit profit + (numScoops * 0.17).
self addEvent:
[self customerLeave: customer]
next: (rand rand Integer: 5)
\end{lstlisting}

Once more the time between a group of customers ordering and leaving
is determined by a random value chosen between 1 and 5. When the
customers finally do leave, they relinquish their chairs.

\begin{lstlisting}
    customerLeave: customer I size I
size customer groupSize.
('group of size I, size, I leavesl
) print.
remainingChairs remainingChairs + customer groupSize
\end{lstlisting}

We will make one final change to illustrate how our simulation can be made even
more realistic. In practice, few random events ever occur with uniform
probability. More often, other distributions, such as a Bernoulli distribution
or a Poisson distribution, are observed to model a process. One very common form
is the Normal distribution, which is characterized by values clustering around a
mean, with the chances of a value decreasing exponentially the farther they move
from the mean. Figure 7.7 shows one class that can be used for generating random
values with a normal distribution. No attempt is made to motivate the algorithm
used in the method for the message next; an interested reader can refer to the
end of this chapter fDr additional literature.

Given the ability to produce random values from a normal distribution, we can
change our assumption about customers arrivals to be more realistic. For
example, we could assume customers arrive in a normal distribution with a mean
of 3 minutes and a standard deviation of 1 minute. Figure 7.8 shows the class
header and the protocol for the initialization message new and the message
scheduleArrival incorporating these changes. In the method for scheduleAnival we
have also incorporated a "closing time" by adding events corresponding only to
customers who arrive before some fixed limit. After closing time, no new
customers will arrive, but the customers waiting to order and waiting to leave
will still be processed. This is so that the event queue can be flushed out and
the simulation terminate in a clean fashion.

Other random values used in the simulation could be modified to use a different
distribution by making changes such as the ones we have illustrated for the
arrival time distribution.

We end with an example session of our final simulation:
\begin{lstlisting}
    Class Normal :Random
I mean deviation I
[
new
self setMean: 1.0 deviation: 0.5
setmean: m deviation: s
mean m.
deviation s
next I v1 v2 sui
s~l
[s > = I] whileTrue:
[v1 (2 * super next) - 1
v2 (2 * super next) - 1.
s v1 squared + v2 squared ].
u (- 2.0 * s In / s) sqrt.
t mean + (deviation * v1 * u)
\end{lstlisting}

\begin{lstlisting}
    Class IceCreamStore :Simulation
I profit arrivalDistribution rand scoopDistribution remainingChairs I
[
new
profit O.
remainingChairs 15.
rand Random new.
~arrivalDistribution Normal new:
setMean: 3.0 deviation: 10.
scoopDistribution DiscreteProbability new:
defineWeights: #(65 26 10).
self scheduleArrival
scheduleArrival I newCustomer time I
newCustomer Customer new.
time self time + (arrivalDistribution next).
(time < 15) ifTrue: [ self addEvent: [ self customerArrival: newCustomer]
c:lt: time]
\end{lstlisting}

\begin{lstlisting}
    store lceCreamStore new
[stQre time < 60] whileTrue: [store proceed]
event received at 3.46877
group of size 8 arrives
takes chairs, schedules order
event re<;eived at 5.81336
group of size 8 arrives
finds no chairs, leaves
event received at 6.46877
group of size 8 orders 11 scoops
event received at 146877
group of size 8 leaves
event received at 8.91228
group of size 1 arrives
takes chairs, schedules order
event received at 10.9123
group of size 1 orders 1 scoops
event received at 10.9499
group of size 7 arrives
takes chairs, schedules order
event received at 11.8463
group of size 5 arrives
takes chairs, schedules order
event received at 12.0194
group of size 2 arrives
finds no chairs, leaves
event received at 12.8463
group of size 5 orgers 6 scoops
event received at 12.9123
group of size 1 leaves
event received at 12.9499
group of size 7 orders 13 scoops
event received at 13.8077
group of size 7 arrives
finds no chairs, leaves
event received at 14.6301
group of size 5 arrives
finds no chairs, leaves
event received at 16.9499
group of size 7 leaves
event received at 17.8463
group of size 5 leaves
store reportProfits
profits are 5.27
\end{lstlisting}

Since Smalltalk objects can be created, removed, and in general placed in a
one-to-one correspondence with objects in the abstract model being simulated
(the objects representing the customer groups, for example). It is relatively
easy to take any model expressed in the discrete event-driven form and enunciate
it in Smalltalk. Similarly the development of new simulations is simplified by
the inheritance ofcommon behavior from class Simulation. Finally, once a
simulation has been developed, it is easy to modify, for example, replacing a
value generated with a normal distribution with one generated according to a
Bernoulli distribution.

\secrel{Further Reading}

A good introduction to the concepts of simulation can be found in (Maryanski
80). A more theoretical treatment is given in (Zeigler 76). The definitive
description of the Smalltalk-80 language (Goldberg 83) illustrates many more
extensive simulations in Smalltalk. Other programming languages designed
forsimulation include Simula (Birtwistle 73), Demos (Birtwistle 79), and GPSS
(Greenberg 72). The algorithm for computing the normal distribution is taken
from Knuth (Knuth 81, Vol. 2). A large collection of references to computer
games and simulation exercises is found in (Gibbs 74).


\secly{Excersises}

\begin{enumerate}
    
    \item Deciding when to use subclassing and when to use an instance variable
    is not always easy. An argument can be made that DiscreteProbability should
    really be a subclass of Random, such as the following:
    \begin{lstlisting}
        Class DiscreteProbability :Random
I weights max I
[
    defineWeights: anArray
weights anArray.
max anArray inject: 0 into: [:x :y I x + y]
next Iindex value I
value super randlnteger: max.
index 1.
[value> (weights at: index)]
whileTrue: [value value - (weights at: index).
index index + 1].
t index
    \end{lstlisting}
    Look at the class description for Random, in particular the response . to
    the message randlnteger:, and then describe why this class description will
    not produce the desired result.

    \item An alternative method of defining a discrete probability is to provide
    the actual sample space in a collection. For example, suppose a group of
    boys are observed to have heights represented by the array \#(60 54 60 62
    50). We can then ask for the height of a randomly selected boy. The
    following shows how a class SampleSpace might be used for this purpose:
    \begin{lstlisting}
        sample SampleSpace new; define: #(60 54 60 62 50)
        sample first
    60
    \end{lstlisting}
    Producy a class description for SampleSpace.

    \item If written in a manner analogous to DiscreteProbability, the class
    SampleSpace defined in the last exercise is said to provide Ilrandom
    selection with replacement.II The alternative, random selection without
    replacement, is frequently more useful. For example, a sample space might
    represent a deck of cards, and a random selection, the choosing of a card.
    This card should then be thought of as being removed from the deck and not
    available for return in subsequent selections. Describe how to modify the
    class SampleSpace to provide random selection without replacement.
    
\end{enumerate}

\secup
