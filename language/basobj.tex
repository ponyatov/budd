\secrel{Basic Objects}

The class Object is a superclass of all classes in the system and is used to
provide a consistent basic functionality and default behavior. For example,
the message = = is defined in class Object and is thus accepted by all
objects in the Little Smalltalk system. This message tests to see if the
expressions for the receiver and the argument represent the same object.
Another message defined in class Object is the message class, which returns the object representing the class of the receiver.

The last chapter introduced the classes associated with literal objects.
other types of objects are also basic to many applications. For example,
instances of the class Radian are used to represent radians. A radian is a
unit of measurement, independent of other numbers. Only radians will
respond to trigonometric functions such as sin and cos. Numbers can be
converted into radians by passing them the message radians. Similarly,
radians can be converted into numbers by sending them the message
asFloat. Only a limited range of arithmetic operations on Radians, such
as scaling by a numeric quantity or taking the difference of two radians,
are permitted. Radians are normalized by adding or subtracting multiples
of 2'11" from their value.

The class Point is used to represent ordered pairs of quantities. Ordered pairs are useful in the solution of many problems, such as storing
coordinate pairs in graphics applications. In fact, the class Number provides a convenient method for constructing points. All instances of class
Number will respond to the message @ by producing a point consisting
of the receiver and the argument. Thus 10 @ 12 generates a point representing the ordered pair (l0,12). The first value is known as the x-value
and will be returned in response to the message x. The second value is the
y-value and is returned in response to the message y

The class String provides messages useful in manipulating arrays of
characters. One important property of this class is that its instances are
the only objects in the Little Smalltalk system that can be displayed on an
output device such as a terminal or printer. Any object to be displayed
must first be converted into an instance of class String. The behavior
defined in class Object for the message print is to convert the object into
a string (using the message printString) and then to print that string (by
passing the message print to it).

The message printString is uniformly interpreted throughout the Little
Smalltalk system as "produce a string representation ofyour value." Classes
for which this makes sense (such as Integer) must define a method for
this message that will produce the appropriate string. By default (that is,
by a method in class Object that will be invoked unless overridden), a
string containing the name of the class of the object is produced. In subsequent chapters we will see several examples of how different classes
respond to the message printString.
