\secrel{Syntax}\secdown

В этой главе представлен синтаксис для литеральных объектов (таких как числа) 
и синтаксис для сообщений. В нем объясняется, как использовать систему Little Smalltalk 
для вычисления выражений, набираемых непосредственно с клавиатуры, и как использовать 
несколько простых сообщений для получения информации о различных типах объектов.

\bigskip

В этой главе будет описано, как объекты представляются и 
управляются в Little Smalltalk. Как мы отмечали в главе 1, 
все в \st\ --- это объект. Обсуждение синтаксиса начинается 
с описания того, как объекты представлены.

\secrel{Литеральные константы}

Некоторые объекты, \term{литералы}, отличаются тем, что их имя однозначно идентифицирует 
класс и значение объекта, независимо от контекста, и тем фактом, что их не нужно 
объявлять перед использованием. Например, символ 7, независимо от того, где он 
появляется, всегда обозначает один и тот же объект. В Algol-подобных языках 
такой символ, как 7, обычно обозначает «значение», а не идентификатор. В
\st\ это различие гораздо менее отчетливо. Все объекты, включая числа, являются 
объектами, а объекты характеризуются сообщениями, которые они принимают, и их 
ответами на них. Таким образом, 7 обозначает объект так же, как идентификатор, 
такой как \var{x} (в надлежащем контексте), может обозначать объект.

Числа, пожалуй, самые распространенные литеральные объекты. Существует два класса 
чисел, которые могут быть записаны как литеральные объекты, а именно целые числа и 
значения с плавающей точкой. Числа отвечают на различные арифметические сообщения 
(унаследованные от класса \class{Number}) и сообщения отношений (унаследованные от класса 
\class{Magnitude}). Экземпляр класса \class{Integer} состоит из необязательного знака, за которым 
следует любое количество цифр. Число с плавающей запятой состоит из целого числа, 
за которым следует точка (десятичная точка) и другого целого числа без знака 
(дробная часть) и/или буквы \verb|е| и целого числа со знаком (экспоненциальная часть). 
Любому числу может предшествовать основание системы счисления, которое представляет собой 
положительное целое число, за которым следует буква \verb|r|. Для оснований больше 10 
буквы от A до Z интерпретируются как цифры. Примеры чисел:

\begin{lstlisting}
7
16rFF
-3.1415926
2e32
2.4e-32
15rC.ABC
\end{lstlisting}
    
\noindent
Основание системы счисления в основном используется просто ради удобства и внешнего вида. 
Число 16rFF совпадает с числом 10r255 или просто 255.
    
Класс \class{Char} предоставляет возможности для работы со значениями букв. Буквы отличаются 
от цифр. Поскольку символы имеют порядок, заданный последовательностью сортировки, 
их можно сравнивать и, следовательно, они являются подклассом класса \class{Magnitude}. 
Символ написан в виде знака доллара, за которым следует буква (или цифра). 
Ниже приведены примеры экземпляров этого класса:
    
\begin{lstlisting}
$A
$7
$
$$
\end{lstlisting}
    
Экземпляр класса \class{String} представлен последовательностью букв между одинарными кавычками. 
Встраивание кавычки в строку требует двух соседних кавычек. Строка похожа на массив; 
фактически класс \class{String} является подклассом \class{ArrayedCollection}, как и класс \class{Array}. 
И строки, и массивы могут быть объединены вместе для формирования больших строк 
с помощью оператора \term{конкатенации} строк\ --- запятой (,). Примеры строк:
    
\begin{lstlisting}
'a String'
'a String with an '' embedded quote mark'
\end{lstlisting}
    
Массив \class{Array} записывается в виде знака фунта (\verb|#|), за которым следует список элементов 
массива в скобках. Элементами массива являются литеральные объекты (числа или символы), 
строки или другие массивы. В списке массивов ведущий знак фунта на символах 
и массивах может быть исключен. Примеры массивов:

\begin{lstlisting}
#(this is an array of symbols)
#(12 'abc' (another array))
\end{lstlisting}

\noindent
Массивы и строки используют сообщения \verb|at:| и \verb|at:put:| 
для выбора и изменения определенных элементов в их коллекциях.

Класс \class{Symbol}\ --- это еще один литеральный класс. Символ записывается в виде 
знака фунта (\verb|#|), за которым следует любая последовательность букв. 
Пробелы между символами не допускаются. В отличие от строки (которая также 
является последовательностью букв) символ не может быть разбит на более 
мелкие части. Кроме того, одна и та же последовательность букв, используемая 
в разных местах, всегда будет обозначать один и тот же объект. В отличие от 
чисел, символов или букв, символы не имеют порядка и не могут сравниваться 
(за исключением, конечно, равенства объектов). Примеры символов:

\begin{lstlisting}
#aSymbol
#AndAnother
#+++
#very.long.symbol.with.periods
\end{lstlisting}

\secrel{Identifiers}

Identifiers in Little Smalltalk can be divided into three categories: instance
variables, class names, and pseudo-variables. An identifier beginning with
a capital letter is always a class name, whereas an identifier beginning
with a lowercase letter must represent either a pseudo variable or an
instance variable.

At the command level, new instance variables can be defined merely
by assigning a value to a name. The assignment arrow is formed as a twocharacter sequence consisting of a less than sign and a minus sign
\note{From now on the text will use the symbol $\leftarrow$ to represent this two-character sequence}:
\begin{lstlisting}
newname <- 17
\end{lstlisting}

Instance variables defined at the command level are known only at the
command level and cannot be used within a method for any class. As we
will see in a later chapter, instance variables within a class must all be
declared.

Class identifiers respond to a variety of messages that can be used to
discover information concerning the class the object represents. For example, the message respondTo, when passed to an object representing a
class, will cause the class to print a list of the messages to which instances
of the class will respond.

Pseudo variables look like normal identifiers (that is, they are named
by a sequence of letters beginning with a lower case letter), but unlike
identifiers they need not be declared. There are several pseudo variables:
self, super, selfProcess
\note{The pseudo-variables selfprocess and smalltalk are unique to Little Smalltalk and are
not part of the \st-80 system, where different techniques are used to obtain the currently executing process or to obtain information about the current environment. See Appendix 5 for an overView of the differences between Little Smalltalk and the Smalltalk-80
programming environment.}
, true, false, nil, and smalltalk. Arguments for
a method (to be discussed shortly) are also considered to be. pseudo-variables. Of the seven, self, super, and selfProcess are farthest from being
literal objects because their meaning depends entirely upon context. We
will discuss these in more detail when we describe class methods and
processes. The next three, true, false, and nil, are defined to be instances
(usually the only instances) of the classes True, False, and UndefinedObject, respectively. We will discuss these three in more detail when
we outline the behavior of different classes. The final pseudo variable,
smalltalk, is an instance of class Smalltalk and is used to centralize several
pieces of information concerning the currently executing environment.

Other types of objects in the Little Smalltalk system, such as blocks
and instances of user defined classes, will be discussed in later sections.

\secrel{Messages}

As noted in Chapter I, all actions in Smalltalk are produced by sending
messages to objects. This section begins by describing the syntax used to
produce messages.

Any message can be divided into three parts; a receiver, a message
selector, and zero or more arguments. The receiver and argument portions
of a message can be specified by other message expressions, or they may
be specified by a single token, such as an identifier or a literal.

The first type of message selector requires no arguments and is called
a unary message. A unary message selector consists of an identifier, the
first letter of which must be lowercase. For example:
\begin{lstlisting}
7 sign
\end{lstlisting}
illustrates the message sign being passed to the number 7. Unary messages,
like all messages, elicit a response, which is simply another object. The
response to sign is an integer, either -1,0, or 1, depending upon the sign
of the object the message was sent to (the receiver). Unary messages parse
left to right, so, for example:
\begin{lstlisting}
7 factorial sqrt
\end{lstlisting}
returns $\sqrt{7!}$, or approximately 70.993.    

The second form of message, called a binary message, takes one argument. 
A binary message is formed from one or two adjacent nonalphabetic characters\note{Some 
characters, such as braces, parenthesis or periods, cannot be used to form
binary messages. See the description in Appendix 2 for a more complete description of the
restrictions.}.
Binary messages tend to be used for arithmetic
operations, although this is not enforced by the system and there are
notable exceptions. An example of a binary message is arithmetic addition:
\begin{lstlisting}
7 + 4
\end{lstlisting}
At first the fact that this is interpreted as "send the message + with
argument 4 to-the object 7"may seem strange; however, soon the uniform
treatment of objects and message passing in Smalltalk makes this seem
natural.

Binary messages, like unary messages, parse left to right. Thus
\begin{lstlisting}
7 + 4 * 3
\end{lstlisting}
results in 33, not 19. \emph{Unary messages have a higher precedence than binary
messages}, thus
\begin{lstlisting}
7 + 17 sqrt
\end{lstlisting}
evaluates as $7 + (\sqrt{17})$, not $\sqrt{(7 + 17)}$.

The most general type of message is a keyword message. The selector
for a keyword message consists of one or more keywords. Each keyword
is followed by an argument. A keyword is simply an identifier (again, the
first character must be lower case) followed by a colon. The argument can
be any expression, although if the expression is formed using a keyword
message, it must be placed in parentheses to avoid ambiguity. Example
keyword expressions are:
\begin{lstlisting}
7 max: 14.
7 between: 2 and: 24
\end{lstlisting}

When we wish to express the name of the message being requested by
a keyword message, we catenate the keyword tokens. Thus we say the
message selector being expressed in the second example above is between:and:. There can be any number of keywords in a keyword message,
although in practice few messages have more than three.

Keyword messages have lower precedence than either binary or unary
messages. Thus
\begin{lstlisting}
7 between: 2 sqrt and: 4 + 2
\end{lstlisting}
\begin{lstlisting}
7 between: (2 sqrt) and: (4 + 2)
\end{lstlisting}
            
\secrel{Getting Started}

You now have enough information to try getting some hands-on experience
using the Little Smalltalk system. After logging on, type the command st.
After a moment, the message "Little Smalltalk" should appear, and the
cursor should be indented by a small amount on the next line. If, at this
point, you type in a Smalltalk expression and hit the return key, the expression will be evaluated and the result printed. Try typing "3 + 4" and see
what happens. The result should be a 7, produced at the left margin. The
cursor then should advance to the next line and once more tab over several
spaces. Try typing "5 + 4 sqrt." Can you explain the outcome? Try "(5 +
4) sqrt."

Try typing
\begin{lstlisting}
i <- 3
\end{lstlisting}
Notice that, since assignment expressions do not have a value, no value
was printed. However, if you now type
\begin{lstlisting}
i
\end{lstlisting}
the most recent object assigned to the name will be produced.

The name last always contains the value of the last expression computed. Try typing
\begin{lstlisting}
27 + 3 sqrt
\end{lstlisting}
followed by
\begin{lstlisting}
last
\end{lstlisting}

\secrel{Finding Out About Objects}

There are various messages that can be used to discover facts about an
object. The message class, for example, will tell you the class of an object.
Try typing
\begin{lstlisting}
7 class
\end{lstlisting}
The message superClass, when passed to an instance of Class, will return
the immediate superclass of that class. Try typing
\begin{lstlisting}
Integer superClass
7 class superClass
\end{lstlisting}
What is the superclass of Object?

The keyword message respondsTo: can be used to discover if an object
will respond to a particular message. The argument must be a symbol,
representing the message. Try typing
\begin{lstlisting}
 7 respondsTo: #+
$A respondsTo: #between:and:
$A respondsTo: #sqrt
\end{lstlisting}
When passed to a ciass, the message respondTo: inquires whetherinstances
of the class respond to the given message. For example,
\begin{lstlisting}
Integer respondsTo: #+
\end{lstlisting}

You can discover if two objects are the same using the binary message
==. The message \verb|~~| is the logical inverse of ==. Try typing
\begin{lstlisting}
i <- 17
i == 17
17~~17
\end{lstlisting}

One way to tell if an object is an instance of a particular class is to
connect the unary message class and the binary message ==. Try typing
\begin{lstlisting}
i class == Integer
\end{lstlisting}

A simple abbreviation for this is the message isMemberOf:. For example, the last expression given is equivalent to
\begin{lstlisting}
i isMemberOf: Integer
\end{lstlisting}

Suppose we want to tell if an object is a Number, but we don't care if
it is any particular kind of number (Integer or Float). We could use the
boolean OR bar (I), which is recognized by the boolean values true and
false:
\begin{lstlisting}
(i isMemberOf: Integer) | (i isMemberOf: Float)
\end{lstlisting}
A simplier method is to use the message isKindOf;. This message asks
whether the class of the object, or any of superclasses, is the same as the
argument. Try typing
\begin{lstlisting}
i isKindOf: Number
\end{lstlisting}

\secrel{Blocks}

An interesting feature of Smalltalk is the ability to encapsulate a sequence
of actions and then to perform those actions at a later time, perhaps even
in a different context. This feature is called a block (an instance of class
Block) and is formed by surrounding a sequence of Smalltalk statements
with square braces, as in:
\begin{lstlisting}
[ i <= i+1. i print ]
\end{lstlisting}

Within a block (and, as we will see in the next chapter, in general
within a method) a period is used as a statement separator. Since a block
is an object, it can be assigned to an identifier or passed as an argument
with a message or used in any other manner in which objects may be used.
In response to the unary message value, a block will execute in the context
in which it was defined, regardless of whether this is the current context
or not. That is, when the block given above is evaluated, the identifier i
will refer to the binding of the identifier i that was known at the time the
block was defined. Even if the block is passed as an argument into a class
in which there is a different instance variable i and then evaluated, the i
in the block will refer to the i in the context in which the block was defined.
Thus a block when used as a parameter is similar to the Algol-60 call-byname notion of a thunk.

The value returned by a block is the value of the last expression inside
that block. Frequently a block will contain a single expression, and the
value resulting from that block will be the value of the expression.

One way to think about blocks is as a type of in-line procedure declaration. Like procedures, a block can also take a number of arguments.
Parameters are denoted by colon-variables at the beginning of the block,
followed by a vertical bar and then the statements composing the block.
For example,
\begin{lstlisting}
[:x :y | (x + y) print]
\end{lstlisting}
is known as a two-parameter block (sometimes two-argument block). The
message value: is used to evaluate a block with parameters the number of
value: keywords given matches the number of arguments in the block. So,
for the example given above, the evaluating message would be value:value:.

\secrel{Comments and Continuations}

A pair of double quote marks (II) are used to enclose a comment. One
must be careful not to confuse the double quote mark with two adjacent
single quote marks CI), which look very similar. The text of the comment
can be arbitrary and is ignored by the Little SmaIItalk system.

The Little Smalltalk system assumes that each line typed at the terminal
is a complete SmaIItalk expression. Should it be necessary to continue a
long expression on two or more lines, a special ineJication must be given
to the Little SmaIItaik system to prevent it from misinterpreting the partial
expression on the first line and generating an unintentional error message.
This special indication is a backwards slash C""-) as the last character on
all intermediate lines, for example:
\begin{lstlisting}
    2 +     \
    3 * 7   \
    + 5     \
40
\end{lstlisting}


\secly{EXERCISES}

\begin{enumerate}

    \item Show the order of evaluation for the subexpressions in the following
    expression:
    \begin{lstlisting}
        7/2 between: 7 + 17 sqrt and: 3 * 5
    \end{lstlisting}

    \item Type the following expressions:
    \begin{lstlisting}
        7 = = 7
        label = = label
        #abe = = #abc
    \end{lstlisting}
    How do you explain this behavior?
    
    \item What values will be printed in place of the question marks in the
    following sequence:
    \begin{lstlisting}
            i <- 17
            j<-[i< +1]
            j print
        ??
            j print
        ??
            i < - 23
            i print
        ??
            j value print
        ??
            i print
        ??
            j value print
        ??
            i print
        ??
    \end{lstlisting}

\end{enumerate}

\secup
