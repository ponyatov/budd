\secrel{Saving Environments}

Updating an employee database is not something you do once and then
never again; rather it must be done periodically, as new employees are
hired or retire. One way to update an employee database is to record the
final values for the database on a slip of paper. Before the start of the next
session you initialize the database by initializing each entry. This may be
somewhat unsatisfactory; slips of paper tend to get lost. Since computers
usually have a much better memory for such things, a good alternative is
to use the Little Smalltalk system to save and restore environments. We
use the term environment to denote the set of all objects accessible at any
one time. The command

\begin{lstlisting}
    )s filename
\end{lstlisting}

\noindent
saves a representation of the current environmerh in the indicated file. I In
response to this command, a message indicating the number of bytes
written to the file will be printed. These files tend to be rather large; too
many of them can clutter your directory.

After saving an environment, you can then continue execution or exit
the Little Smalltalk system by typing control-D. A saved environment can
later be restored by typing the command
\begin{lstlisting}
    )1 filename
\end{lstlisting}

\noindent
This command loads the environment saved in the indicated file\note{The )s and )1 commands do not work on all machines on which the other portions
of the Little Smalltalk system will operate. Check with your system manager, or experiment
yourself, to see if they work on your system.}. Notice
that in doing so, it totally erases the environment that existed before the
)1 command was issued, replacing it with the restored environment. In
response to this command, a message indicating the number of bytes read
in will be produced. This figure should match the figure written after the
)s command. All names that denoted accessible objects before the )s COffi
mand are now valid, and the user can continue as if neither the )s nor the
)1 commands had been issued.

Note that the same environment can be loaded many times. Acommon
use for this feature is for users to save an environment containing their
favorite classes and objects that they have created. This environment can
then be quickly loaded, and the classes will be available without having to
issue )i commands for each one.
