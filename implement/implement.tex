\secrel{The Implementation}\secdown

\secrel{Implementation Overview}\secdown

This chapter describes the features that make an interpreter for the
Smalltalk language different from, say, a Pascal compiler. Provides a
high-level description of the major components in the Little Smalltalk
system.

\secrel{Identifier Typelessness}
\secrel{Unscoped Lifetimes}
\secrel{An Interactive System}
\secrel{A Multi-Processing Language}
\secrel{System Overview}

\secup

\secrel{1he Representation ofObjects}\secdown

The internal representation of objects in the Little Smalltalk system is
described in this chapter, which also overviews the memory management algorithms. 
The chapter ends with a discussion of several optimizations used to 
improve the speed of the Little Smalltalk system.

\secrel{Special Objects}
\secrel{Memory Management}
\secrel{Optimizations}

\secup

\secrel{Bytecodes}\secdown

The techniques used to represent methods internally in the Little Smalltalk system are described in this chapter.

\secrel{The Representation of Methods}
\secrel{Optimizations}
\secrel{Dynamic Optimizations }

\secup

\secrel{The Process Manager}\secdown

This chapter presents a more detailed view of the central component
of the Little Smalltalk system, the process manager. It then goes on to
describe the driver, the process that reads commands from the user
terminal and schedules them for execution. The chapter ends by describing the class parser and the internal representation of classes.

\secrel{The Driver}
\secrel{The Class Parser}

\secup

\secrel{The Interpreter}\secdown

This chapter describes the actions of the interpreter and the courier
in executing bytecodes and passing messages. It ends by describing
the primitive handler and the manipulation of special objects.

\secrel{Push Opcodes}
\secrel{Pop Opcodes}
\secrel{Message-Sending Opcodes}
\secrel{Block Creation}
\secrel{Special Instructions}
\secrel{The Courier}
\secrel{The Primitive Handler}
\secrel{Blocks}

\secup

\secup
