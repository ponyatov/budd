\secrel{Время жизни не связано с областью видимости}

В Паскале, как и во многих других языках, память для локальных переменных в
процедуре выделяется при вызове процедуры, и может быть освобождена при возврате
из процедуры. Если, например, процедура P вызывает процедуру Q, память для Q
будет выделена после памяти для P, и может быть освобождена до памяти для P
(поскольку возврат из Q должен быть раньше чем может быть возврат из P). Таким
образом, можно использовать стек для выделения области памяти, размещаемой на
вершине стека при входе в каждую процедуру (рисунок 11.1).

В Smalltalk, как мы уже отмечали, объекты могут быть созданы в любое время, и
могут сохраняться в течение неопределенного периода времени. Это требует более
сложного протокола выделения памяти. Поскольку физическая память в большинстве
систем ограничена, важно, чтобы память для объектов, к которым больше не
обращались, использовалась повторно для новых объектов. Таким образом, в любое
время память можно рассматривать как последовательность областей, некоторые из
которых используются, а другие не используются (рисунок 11.2).

Таким образом, диспетчер памяти является важным компонентом системы Little
Smalltalk. Диспетчер памяти обрабатывает все запросы по памяти и отмечает, когда
память больше не используется. Эта важная часть системы будет более подробно
описана в главе 12.

\fig{implement/fig_11_1.png}{height=\textheight}
\fig{implement/fig_11_2.png}{height=\textheight}
