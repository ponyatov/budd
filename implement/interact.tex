\secrel{Интерактивная система}

Внутреннее представление объектов в Little Smalltalk является компромиссом между
двумя конкурирующими целями. С одной стороны, представление должно быть
достаточно гибким, чтобы обеспечить простоту создания, изменения и удаления
объектов. С другой стороны, оно не может быть настолько общим, чтобы значительно
снизить эффективность. Например, если методы были бы сохранены в их
первоначальном текстовом виде, их можно было бы легко изменить. Это, однако,
серьезно замедлит работу интерпретатора, так как потребует повторного анализа
кода перед каждым выполнением.

Внутреннее представление объектов было кратко обсуждено в предыдущем разделе, и
будет объяснено более подробно в следующей главе. Однако важным частным случаем
является представление объектов класса \class{Class}, которое должно включать
представление методов класса. Обратите внимание, что интерактивная система,
такая как система Little Smalltalk, должна хранить гораздо больше информации чем
пакетная система типа традиционного компилятора. В то время как компилятор может
игнорировать и удалять текстовое представление, как только будет создана
адекватная внутренняя форма, интерактивная система должна быть в состоянии
восстановить исходный текст, если это необходимо. Это наиболее очевидно в случае
описания классов, где должны быть представлены три варианта. Первый вариант\ ---
заново сгенерировать текстовое описание класса из внутренней формы, если это
необходимо, например, для редактирования описания класса. Другой вариант\ ---
сохранить как внутреннее представление, так и исходное текстовое представление
кода в памяти. Это, однако, потребует слишком много памяти для небольших машин.
Эффективный, хотя и несколько менее общий вариант\ --- сохранить как часть
внутреннего формата класса объекта только позицию и имя файла, из которого было
прочитано описание класса. Единственным ограничением является то, что описания
классов не могут быть созданы, но должны существовать в каком-то файле. Если
пользователь желает редактировать описание класса, файл открывается и
редактируется с использованием обычного редактора.

Как и большинство интерпретирующих систем, внутреннее представление процедурной
(или исполняемой) части класса, а именно методов класса, можно рассматривать как
язык ассемблера для виртуальной машины специального назначения. В то время как
реальная машина, на которой выполняется система, имеет дело с ресурсами, такими
как байты и слова, виртуальная машина может иметь дело с понятиями более
высокого уровня, такими как стеки, объекты и символы. Язык ассемблера для этой
виртуальной машины называется \term{байткод} и обсуждается в главе 13.
Интерпретатор байт-кода описан в главе 14.
