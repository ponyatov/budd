\secrel{Обзор реализации}\secdown

В этой главе описываются функции, которые отличают интерпретатор языка Smalltalk
от, скажем, компилятора Pascal. Она предоставляет общее описание основных
компонентов системы Little Smalltalk.

\bigskip

Чтобы лучше понять причины многих архитектурных особенностей системы Little
Smalltalk, важно сначала рассмотреть, какие особенности языка Smalltalk
заставляют его реализацию отличаться, скажем, от компилятора Pascal или
интерпретатора BASIC. К наиболее важным аспектам языка, с точки зрения
разработчика, относятся следующие:
\begin{description}

    \item{\st\ язык безтиповый (с динамической типизацией)}\ \\
    В \st\ нет понятия «объявление типа идентификатора». Любой
    идентификатор может использоваться для ссылки на объекты любого типа, и
    может быть изменен в любое время для ссылки на объекты другого типа.

    \item{Объекты имеют неограниченное время жизни}\ \\
    В языке, подобном Algol, таком как Pascal, переменные являются глобальными
    или локальными. Все глобальные идентификаторы существуют во всё время
    выполнения, и поэтому им могут быть назначены статические области памяти.
    Локальные идентификаторы существуют только до тех пор, пока активна
    процедура, в которой они объявлены. Поскольку процедуры активируются и
    деактивируются по принципу стека, стек (иногда называемый «стек записей
    активации») может использоваться для предоставления локальных областей
    памяти. В \st, с другой стороны, объекты существуют вне вызова
    процедуры (если мы примем что передача сообщения \st\ эквивалентна
    вызова процедуры, что не верно) и могут существовать в течение
    неопределенных периодов времени. Таким образом, в \st\ стековая схема
    распределения не подходит, и должна использоваться другая политика выделения
    памяти.

    \item{\st\ интерактивен}\ \\
    Как и во многих других языках программирования, таких как APL, B, Prolog или
    SETL, Little Smalltalk представляет собой интерактивную и динамическую
    систему. Это означает, что пользователь может не только создавать или
    изменять идентификаторы во время выполнения, но и такие базовые вещи, как
    описания классов, могут динамически изменяться во время выполнения. Таким
    образом, если скорость выполнения в рантайме должна быть достаточно
    постоянной, ни одна часть системы не может быть слишком сильно привязана к
    какой-либо конкретной фиче (такой как описание класса), которая впоследствии
    может быть изменена.

    \item{\st\ язык многозадачный}\ \\
    Как мы видели в предыдущей главе, пользователь может указать множество
    различных процессов и выполнять их одновременно. Таким образом, система
    Little SmaIItalk должна облегчать передачу управления от одного процесса
    другому.

\end{description}

В следующих разделах будут описаны некоторые из наиболее важных способов,
которыми проект Little SmaIItalk обеспечивает эти функции. В остальных главах
речь пойдет о реализации более подробно.

\secrel{Отсутствие типов у идентификаторов}

В Algol-подобном языке, таком как Pascal, все идентификаторы должны иметь
объявленный тип, известный во время синтаксического анализа программы при
компиляции. Таким образом, поскольку области памяти отводятся отдельно, либо во
время загрузки, либо во время выполнения, необходимо выделять пространство
только для значений, поскольку информация о типе известна компилятору, и код
может быть сгенерирован соответствующим образом.

\fig{implement/val.png}{height=.12\textheight}

В динамическом языке, таком как Smalltalk, тип идентификатора обычно не может
быть определен во время анализа программы (или описания класса). Традиционное
решение состоит в том, чтобы с памятью для каждого идентификатора ассоциировать
небольшой тег, который указывает тип объекта, хранимого в поле значения.

\fig{implement/tagval.png}{height=.1\textheight}

В языках, где количество типов данных является фиксированным и относительно
небольшим (например, во многих реализациях Lisp), это поле тега может быть также
небольшим, например, восемь бит. В Smalltalk, с другой стороны, единственное
понятие, вообще сопоставимое с понятием типа, это класс объекта, указанный
идентификатором; количество различных классов, которые можно определить,
практически безгранично. К счастью, для каждого класса существует уникальный
объект, содержащий информацию о классе, а именно объект класса. Таким образом,
тип каждого объект в системе Little Smalltalk может быть помечен указателем на
соответствующий объект класса.

\fig{implement/classval.png}{height=.12\textheight}

Чтобы определить, подходит ли какая-либо операция (сообщение) для какого-либо
объекта, система использует указатель класса для анализа класса (и, через другой
указатель в объекте класса, любых суперклассов) при поиске подходящего метода. В
следующей главе будет более подробно объяснена внутренняя структура объектов
Little Smalltalk.


\secrel{Время жизни не связано с областью видимости}

В Паскале, как и во многих других языках, память для локальных переменных в
процедуре выделяется при вызове процедуры, и может быть освобождена при возврате
из процедуры. Если, например, процедура P вызывает процедуру Q, память для Q
будет выделена после памяти для P, и может быть освобождена до памяти для P
(поскольку возврат из Q должен быть раньше чем может быть возврат из P). Таким
образом, можно использовать стек для выделения области памяти, размещаемой на
вершине стека при входе в каждую процедуру (рисунок 11.1).

В Smalltalk, как мы уже отмечали, объекты могут быть созданы в любое время, и
могут сохраняться в течение неопределенного периода времени. Это требует более
сложного протокола выделения памяти. Поскольку физическая память в большинстве
систем ограничена, важно, чтобы память для объектов, к которым больше не
обращались, использовалась повторно для новых объектов. Таким образом, в любое
время память можно рассматривать как последовательность областей, некоторые из
которых используются, а другие не используются (рисунок 11.2).

Таким образом, диспетчер памяти является важным компонентом системы Little
Smalltalk. Диспетчер памяти обрабатывает все запросы по памяти и отмечает, когда
память больше не используется. Эта важная часть системы будет более подробно
описана в главе 12.

\fig{implement/fig_11_1.png}{height=\textheight}
\fig{implement/fig_11_2.png}{height=\textheight}


\secrel{Интерактивная система}

Внутреннее представление объектов в Little Smalltalk является компромиссом между
двумя конкурирующими целями. С одной стороны, представление должно быть
достаточно гибким, чтобы обеспечить простоту создания, изменения и удаления
объектов. С другой стороны, оно не может быть настолько общим, чтобы значительно
снизить эффективность. Например, если методы были бы сохранены в их
первоначальном текстовом виде, их можно было бы легко изменить. Это, однако,
серьезно замедлит работу интерпретатора, так как потребует повторного анализа
кода перед каждым выполнением.

Внутреннее представление объектов было кратко обсуждено в предыдущем разделе, и
будет объяснено более подробно в следующей главе. Однако важным частным случаем
является представление объектов класса \class{Class}, которое должно включать
представление методов класса. Обратите внимание, что интерактивная система,
такая как система Little Smalltalk, должна хранить гораздо больше информации чем
пакетная система типа традиционного компилятора. В то время как компилятор может
игнорировать и удалять текстовое представление, как только будет создана
адекватная внутренняя форма, интерактивная система должна быть в состоянии
восстановить исходный текст, если это необходимо. Это наиболее очевидно в случае
описания классов, где должны быть представлены три варианта. Первый вариант\ ---
заново сгенерировать текстовое описание класса из внутренней формы, если это
необходимо, например, для редактирования описания класса. Другой вариант\ ---
сохранить как внутреннее представление, так и исходное текстовое представление
кода в памяти. Это, однако, потребует слишком много памяти для небольших машин.
Эффективный, хотя и несколько менее общий вариант\ --- сохранить как часть
внутреннего формата класса объекта только позицию и имя файла, из которого было
прочитано описание класса. Единственным ограничением является то, что описания
классов не могут быть созданы, но должны существовать в каком-то файле. Если
пользователь желает редактировать описание класса, файл открывается и
редактируется с использованием обычного редактора.

Как и большинство интерпретирующих систем, внутреннее представление процедурной
(или исполняемой) части класса, а именно методов класса, можно рассматривать как
язык ассемблера для виртуальной машины специального назначения. В то время как
реальная машина, на которой выполняется система, имеет дело с ресурсами, такими
как байты и слова, виртуальная машина может иметь дело с понятиями более
высокого уровня, такими как стеки, объекты и символы. Язык ассемблера для этой
виртуальной машины называется \term{байткод} и обсуждается в главе 13.
Интерпретатор байт-кода описан в главе 14.


\secrel{Многозадачный язык}

Тот факт, что \st\ является многозадачным языком, создает ряд трудностей.
Можно подумать, что если реализация Smalltalk не разрешает несколько процессов,
даже если объекты могут сохраняться бесконечно долго, по крайней мере протокол
передачи сообщений будет демонстрировать поведение, подобное стеку. Например,
если сообщение \var{one} передается объекту \var{a}, а метод, связанный с этим
сообщением, передает второе сообщение \var{two} другому объекту \var{b}, то
второе сообщение должно сделать возврат до того, как отработает первое
сообщение. Таким образом, хранилище, возникающее как часть механизма передачи
сообщений, такое как хранилище для аргументов или временных переменных, может
быть выделено и освобождено стековым способом, подобным записям активации в
обычном языке.

К сожалению, это видение слишком упрощенное. Даже без многозадачности реализация
блоков вызывает проблемы. Для правильного выполнения блок должен иметь доступ к
среде (включая аргументы и временные переменные), в которой он был определен
(замыкание). Кроме того, блок может быть возвращен как результат обработки
сообщения, или назначен идентификатору и, таким образом, переживает сообщение, в
котором он был определен. Даже в случае одного процесса временные и аргументные
переменные не обязательно появляются и умирают в порядке стека.

Решение состоит в том, чтобы унифицированно применять алгоритмы диспетчера
памяти к тем объектам, соответствующим значениям, которые может видеть
пользователь, таким как идентификаторы, а также к внутренне сгенерированным
объектам, таким как те, которые соответствуют традиционным записям активации.
Когда сообщение должно быть отправлено, создается объект класса \class{Context}.
Контекст (рисунок 11.3)\ --- это объект похожий на массив, который указывает на
получателя сообщения, и на объекты-аргументы, переданные вместе с сообщением.
Контекст также предоставляет пространство для любых временных идентификаторов
или внутренних параметров (например, для блоков), которые понадобятся сообщению.

\fig{implement/fig_11_3.png}{height=.75\textheight}

Затем производится поиск описаний классов, чтобы найти фрагмент байт-кода,
соответствующему методу, который обрабатывает сообщение. Как только байт-код
найден, создается объект второго типа, называемый \class{Interpreter}. Имя
\class{Interpreter} является несколько неправильным; лучше его можно было бы
назвать \class{ИнтерпретируемыйМетодГотовыйКИсполнению}. Экземпляры класса
\class{Interpreter} указывают на байт-код, который они будут выполнять, на
контекст, который они будут использовать во время выполнения, и на стек
интерпретатора, используемый виртуальной машиной, которая выполняет байт-код
(рисунок 11.4).

\fig{implement/fig_11_4.png}{height=\textheight}

Когда интерпретатор выполняет инструкцию байт-кода, которая вызывает отправку
нового сообщения, создается новый экземпляр интерпретатора, который связывается
с существующим интерпретатором, который в свою очередь становится неактивным,
пока не вернется ответ на сообщение:

\fig{implement/fig_11_x.png}{height=.2\textheight}

\class{Process}\ --- это просто указатель на активный интерпретатор. Поскольку
может быть много процессов, все активные процессы связаны друг с другом (рисунок
11.5). Структура обработчика процесса будет обсуждаться в главе 14.

\fig{implement/fig_11_5.png}{height=\textheight}


\secrel{Обзор системы}

Реализация системы Little Smalltalk, как и практически любой крупной программной
системы, представляет собой набор взаимодействующих компонентов. В этом разделе
в общих чертах будут описаны различные компоненты и их взаимосвязи.

На рисунке 11.6 показаны основные компоненты системы Little Smalltalk и
взаимосвязи между их потоками управления. Центральным элементом всей системы
является \term{менеджер процессов}. Как мы видели в главе 10, процесс - это
последовательность выражений Little Smalltalk плюс контекстная информация,
необходимая для их правильной интерпретации. Диспетчер процессов поддерживает
очередь активных процессов (напомним, что несколько процессов могут быть созданы
с помощью сообщения \var{fork} или \var{newProcess}), обеспечивая каждому
процессу достаточную долю времени выполнения. Одним из специальных процессов
является \term{driver}. Драйвер считывает команды, введенные пользователем на
терминале, создает процесс для выполнения каждой из них, и помещает его в
очередь, управляемую диспетчером процессов. Подчиненным драйверу является
специальный модуль для чтения и перевода описаний классов в форму, используемую
внутри системы Little Smalltalk.

\fig{implement/fig_11_6.png}{height=\textheight}

Выражения Little Smalltalk хранятся во внутренней форме, называемой
\term{байткодом}. \term{Интерпретатор} отвечает за выполнение байткода и
обновление контекста для каждого процесса соответствующим образом. Байт-коды,
представляющие примитивные операции, обрабатываются специальным модулем,
называемым \term{обработчик примитивов}. Обработчик примитивов является основным
интерфейсом между системой Little Smalltalk и нижележащей операционной системой.
Кроме того, обработчик примитивов манипулирует объектами, такими как целые
числа, вещественные числа, строки и символы, для которых внутреннее
представление отличается от представления "нормальных" объектов Little
Smalltalk. Для этого менеджер примитивов использует набор специальных процедур
для объектов, каждый из которых относится к объекту другого типа.

Одной из наиболее распространенных задач интерпретатора является отправка
сообщения от одного объекта другому. Для этого \term{курьеру} передаются
инструкции, описывающие отправляемое сообщение. Курьер создает интерпретатор для
вычисления сообщения, и помещает его в очередь диспетчера процессов,
приостанавливая отправляющий интерпретатор до тех пор, пока принимающий
интерпретатор не вернет значение и не прекратит работу.

В основе всех частей системы лежит всепроникающий \term{менеджер памяти}.
Менеджер памяти отвечает за создание объектов и отслеживание того, какие объекты
используются в настоящее время, и, что более важно, какие объекты больше не
используются и, следовательно, их память может быть освобождена для создания
последующих объектов.

Наконец, \term{модуль инициализации и завершения} является первой процедурой,
которой передается управление при запуске системы Little Smalltalk. Задача этого
модуля\ --- установить значения определенных глобальных переменных в правильные
начальные состояния, включая загрузку стандартной библиотеке классов Little
Smalltalk, создание процесса драйвера и помещение его в очередь диспетчера
процессов, и запуск выполнения системы. Когда пользователь указывает, что
выполнение должно быть прервано, эта процедура зачищает различные ссылки на
объекты, хранящиеся в глобальных переменных, и, при необходимости, создает
статистику использования памяти.

В последующих главах будут подробно описаны дизайн и реализация каждого из этих
компонентов.


\secup
