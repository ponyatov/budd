\secrel{Обзор системы}

Реализация системы Little Smalltalk, как и практически любой крупной программной
системы, представляет собой набор взаимодействующих компонентов. В этом разделе
в общих чертах будут описаны различные компоненты и их взаимосвязи.

На рисунке 11.6 показаны основные компоненты системы Little Smalltalk и
взаимосвязи между их потоками управления. Центральным элементом всей системы
является \term{менеджер процессов}. Как мы видели в главе 10, процесс - это
последовательность выражений Little Smalltalk плюс контекстная информация,
необходимая для их правильной интерпретации. Диспетчер процессов поддерживает
очередь активных процессов (напомним, что несколько процессов могут быть созданы
с помощью сообщения \var{fork} или \var{newProcess}), обеспечивая каждому
процессу достаточную долю времени выполнения. Одним из специальных процессов
является \term{driver}. Драйвер считывает команды, введенные пользователем на
терминале, создает процесс для выполнения каждой из них, и помещает его в
очередь, управляемую диспетчером процессов. Подчиненным драйверу является
специальный модуль для чтения и перевода описаний классов в форму, используемую
внутри системы Little Smalltalk.

\fig{implement/fig_11_6.png}{height=\textheight}

Выражения Little Smalltalk хранятся во внутренней форме, называемой
\term{байткодом}. \term{Интерпретатор} отвечает за выполнение байткода и
обновление контекста для каждого процесса соответствующим образом. Байт-коды,
представляющие примитивные операции, обрабатываются специальным модулем,
называемым \term{обработчик примитивов}. Обработчик примитивов является основным
интерфейсом между системой Little Smalltalk и нижележащей операционной системой.
Кроме того, обработчик примитивов манипулирует объектами, такими как целые
числа, вещественные числа, строки и символы, для которых внутреннее
представление отличается от представления "нормальных" объектов Little
Smalltalk. Для этого менеджер примитивов использует набор специальных процедур
для объектов, каждый из которых относится к объекту другого типа.

Одной из наиболее распространенных задач интерпретатора является отправка
сообщения от одного объекта другому. Для этого \term{курьеру} передаются
инструкции, описывающие отправляемое сообщение. Курьер создает интерпретатор для
вычисления сообщения, и помещает его в очередь диспетчера процессов,
приостанавливая отправляющий интерпретатор до тех пор, пока принимающий
интерпретатор не вернет значение и не прекратит работу.

В основе всех частей системы лежит всепроникающий \term{менеджер памяти}.
Менеджер памяти отвечает за создание объектов и отслеживание того, какие объекты
используются в настоящее время, и, что более важно, какие объекты больше не
используются и, следовательно, их память может быть освобождена для создания
последующих объектов.

Наконец, \term{модуль инициализации и завершения} является первой процедурой,
которой передается управление при запуске системы Little Smalltalk. Задача этого
модуля\ --- установить значения определенных глобальных переменных в правильные
начальные состояния, включая загрузку стандартной библиотеке классов Little
Smalltalk, создание процесса драйвера и помещение его в очередь диспетчера
процессов, и запуск выполнения системы. Когда пользователь указывает, что
выполнение должно быть прервано, эта процедура зачищает различные ссылки на
объекты, хранящиеся в глобальных переменных, и, при необходимости, создает
статистику использования памяти.

В последующих главах будут подробно описаны дизайн и реализация каждого из этих
компонентов.
