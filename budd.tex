% e-book
\documentclass[oneside,10pt]{book}
%% screen paper layout (A5 landscape)
\usepackage[paperwidth=118.8mm,paperheight=68.2mm,margin=2mm]{geometry}
%% font setup for screen reading
\renewcommand{\familydefault}{\sfdefault}\normalfont
%% hyperlinks pdf style
\usepackage[unicode,colorlinks=true]{hyperref}
%% fix heading styles for tiny paper
\usepackage{titlesec}
\titleformat{\chapter}{\Large\bfseries}{\thechapter.}{1em}{}
\titleformat{\section}{\large\bfseries}{\thesection.}{1em}{}

% graphics
\usepackage[pdftex]{graphicx}
\newcommand{\fig}[2]{\noindent\includegraphics[#2]{#1}}

% xcolor fixes
\usepackage{xcolor}
\definecolor{red}{rgb}{0.7,0,0}		% R
\definecolor{green}{rgb}{0,0.6,0}	% G
\definecolor{blue}{rgb}{0,0,0.7}	% B
\definecolor{darkblue}{rgb}{0,0,0.3}	% DB

% Cyrillization
%% \usepackage[T1,T2A]{fontenc}
\usepackage[utf8]{inputenc}
%% \usepackage[cp1251]{inputenc}
\usepackage[english,russian]{babel}
\usepackage{indentfirst}

% relative sectioning
\usepackage{ifthen}
\newcounter{secdepth}\setcounter{secdepth}{0}
\newcommand{\secup}{\addtocounter{secdepth}{1}}
\newcommand{\secdown}{\addtocounter{secdepth}{-1}}
\newcommand{\secrel}[1]{
\ifthenelse{\equal{\value{secdepth}}{0}}{\part{#1}}{}
\ifthenelse{\equal{\value{secdepth}}{-1}}{\chapter{#1}}{}
\ifthenelse{\equal{\value{secdepth}}{-2}}{\section{#1}}{}
\ifthenelse{\equal{\value{secdepth}}{-3}}{\subsection{#1}}{}
\ifthenelse{\equal{\value{secdepth}}{-4}}{\subsubsection{#1}}{}
}
\newcommand{\secly}[1]{\section*{#1} \addcontentsline{toc}{section}{#1} \refstepcounter{section}}
\newcommand{\subsecly}[1]{\subsection*{#1} \addcontentsline{toc}{subsection}{#1}  \refstepcounter{subsection}}



\title{A Little Smalltalk}
\author{Timothy Budd}

\begin{document}

\maketitle
\tableofcontents

\secly{Preface}
\subsecly{The Little Smalltalk System: Some History}
\subsecly{About A Little Smalltalk}
\subsecly{Acknowledgments}
\subsecly{Obtaining the Little Smalltalk System}

\secrel{The Language}\secdown

\secrel{Basics}\secdown

This chapter introduces the basic concepts of the Smalltalk language;
namely object, method, class, inheritance and overriding.

\secrel{Objects, Classes, and Inheritance}
\secrel{History, Background Reading}
\secup

\secrel{Syntax}\secdown

This chapter introduces the syntax for literal objects (such as numbers)
and the syntax for messages. It explains how to use the Little Smalltalk
system to evaluate expressions typed in directly at the keyboard and
how to use a few simple messages to discover information about different types of objects.

\secrel{Literal Constants}
\secrel{Identifiers}
\secrel{Messages}
\secrel{Getting Started}
\secrel{Finding Out About Objects}
\secrel{Blocks}
\secrel{Comments and Continuations}

\secup

\secrel{Basic Classes}\secdown

The basic classes included in the Little Smalltalk standard library are
explained in this chapter.

\secrel{Basic Objects}
\secrel{Collections}
\secrel{Control Structures}
\secrel{Class Management}
\secrel{Abstract Superclasses}

\secup

\secrel{Class Definition}\secdown

This chapter introduces the syntax used for defining classes. An example class definition is presented.

\secrel{An Illustrative Example}
\secrel{Processing a Class Definition}

\secup

\secrel{A Simple Application}\secdown

This chapter illustrates the development of a simple application in
Smalltalk and describes how environments can be saved and restored.

\secrel{Saving Environments}

\secup

\secrel{Primitives, Cascades, and Coercions}\secdown

This chapter introduces the syntax for cascaded expressions and describes the notion of primitive expressions. It illustrates the use of
primitives by showing how primitives are used to produce the correct
results for mixed mode arithmetic operations.

\secrel{Cascades}
\secrel{Primitives}
\secrel{Numbers}

\secup

\secrel{A Simulation}\secdown

This chapter presents a simple simulation of an ice cream store, illustrating the ease with which simulations can be described in Smalltalk.

\secrel{The Ice Cream Store Simulation}
\secrel{Further Reading}

\secup

\secrel{Generators}\secdown

This chapter introduces the concept of generators and shows how
generators can be used in the solution of problems requiring goaldirected evaluation.

\secrel{Filters}
\secrel{Goal-Directed Evaluation}
\secrel{Operatkms on Generators}
\secrel{Further Reading}

\secup

\secrel{Graphics}\secdown

Although graphics are not fundamental to Little Smalltalk in the same
way that they are an intrinsic part of the Smalltalk-80 system, it is still
possible to describe some graphics functions using the language. This
chapter details three types of approaches to graphics.

\secrel{Character Graphics}
\secrel{Line Graphics}
\secrel{Bit-Mapped Graphics}

\secup

\secrel{Processes}\secdown

This chapter introduces the concepts of processes and semaphores. It
illustrates these concepts using the dining philosophers problem.

\secrel{Semaphores}
\secrel{Monitors}
\secrel{The Dining Philosophers Problem }
\secrel{Further Reading}

\secup

\secup

\secrel{The Implementation}\secdown

\secrel{Implementation Overview}\secdown

This chapter describes the features that make an interpreter for the
Smalltalk language different from, say, a Pascal compiler. Provides a
high-level description of the major components in the Little Smalltalk
system.

\secrel{Identifier Typelessness}
\secrel{Unscoped Lifetimes}
\secrel{An Interactive System}
\secrel{A Multi-Processing Language}
\secrel{System Overview}

\secup

\secrel{1he Representation ofObjects}\secdown

The internal representation of objects in the Little Smalltalk system is
described in this chapter, which also overviews the memory management algorithms. 
The chapter ends with a discussion of several optimizations used to 
improve the speed of the Little Smalltalk system.

\secrel{Special Objects}
\secrel{Memory Management}
\secrel{Optimizations}

\secup

\secrel{Bytecodes}\secdown

The techniques used to represent methods internally in the Little Smalltalk system are described in this chapter.

\secrel{The Representation of Methods}
\secrel{Optimizations}
\secrel{Dynamic Optimizations }

\secup

\secrel{The Process Manager}\secdown

This chapter presents a more detailed view of the central component
of the Little Smalltalk system, the process manager. It then goes on to
describe the driver, the process that reads commands from the user
terminal and schedules them for execution. The chapter ends by describing the class parser and the internal representation of classes.

\secrel{The Driver}
\secrel{The Class Parser}

\secup

\secrel{The Interpreter}\secdown

This chapter describes the actions of the interpreter and the courier
in executing bytecodes and passing messages. It ends by describing
the primitive handler and the manipulation of special objects.

\secrel{Push Opcodes}
\secrel{Pop Opcodes}
\secrel{Message-Sending Opcodes}
\secrel{Block Creation}
\secrel{Special Instructions}
\secrel{The Courier}
\secrel{The Primitive Handler}
\secrel{Blocks}

\secup

\secup

\secly{References}

An annotated bibliography of references related to the Little Smalltalk system.

\secly{Projects}

\secly{Appendices}

\subsecly{Running Little Smalltalk}

\subsecly{Syntax Charts}

\subsecly{Class Descriptions}

\subsecly{Primitives}

\subsecly{Differences Between Little Smalltalk\\and the Smalltalk-80 Programming System}

\end{document}